% This is LLNCS.DEM the demonstration file of
% the LaTeX macro package from Springer-Verlag
% for Lecture Notes in Computer Science,
% version 2.4 for LaTeX2e as of 16. April 2010
%
\documentclass{llncs}
%

\usepackage[utf8]{inputenc}



\begin{document}

\title{On Logic Embeddings and G\"odel's God}

\author{Christoph Benzm\"uller\inst{1}\thanks{This work has been supported by
    the German Research Foundation DFG under grants BE2501/9-1 \&
    BE2501/11-1.} \and Bruno Woltzenlogel Paleo\inst{2}
}

\institute{%Department of Mathematics and Computer Science \\ 
 Freie Universit\"at Berlin, Germany, \url{c.benzmueller@fu-berlin.de}
\and
 Vienna Technical University, Austria, \url{bruno@logic.at}}

\maketitle            



% \begin{abstract}
%   Logic embeddings provide an elegant means to formalize sophisticated
%   non-classical logics in classical higher-order logic. In recent work
%   we have applied the logic embeddings approach to verify and automate
%   a prominent philosophical argument that has fascinated philosophers
%   and theologists for about 1000 years: the ontological argument for
%   the existence of God. In our work we have concentrated on Kurt
%   G\"odel's (respectively, Dana Scott's) modern version of this
%   argument, which employs a second-order modal logic.  In our ongoing
%   computer-assisted study of G\"odel's proof script, the automated
%   reasoning tools have made some interesting observations, some of
%   which were unknown so far.

%   Our work attests the maturity of contemporary interactive and
%   automated deduction tools for classical higher-order logic and
%   demonstrates the elegance and practical relevance of the
%   embeddings-based approach. Most importantly, our work opens new
%   perspectives towards a computational metaphysics.

%   In addition to the above, I will also briefly discuss the relation
%   of the logic embeddings approach to axiomatic and algebraic
%   specifications

%   The study of G\"odel's ontological proof of God's existence is joint
%   work with Bruno Woltzenlogel Paleo.
% \end{abstract}
%
% \section{Introduction}
%

Logic embeddings provide an elegant means to formalize sophisticated
non-classical logics in classical higher-order logic (HOL, Church's
simple type theory \cite{Church40}). In previous work (cf.~\cite{C35}
and the references therein) the embeddings approach has been
successfully applied to automate object-level and meta-level reasoning
for a range of logics and logic combinations with off-the-shelf HOL
theorem provers. This also includes quantified modal logics (QML)
\cite{J23} and quantified conditional logics (QCL) \cite{C37}.  For
many of the embedded logics few or none automated theorem provers did
exist before. HOL is exploited in this approach to encode the
semantics of the logics to be embedded, for example, Kripke semantics
for QMLs \cite{fitting98} or selection function semantics for QCLs
\cite{Stalnaker68}.  The embeddings approach is related to labelled
deductive systems \cite{gabbay96}, which employ meta-level
(world-)labeling techniques for the modeling and implementation of
non-classical proof systems. In our embeddings approach such labels
are instead encoded in the HOL logic.

In recent work \cite{C40,J28} we have applied the embeddings
approach to verify and automate a philosophical argument that has
fascinated philosophers and theologists for about 1000 years: the
ontological argument for the existence of
God~\cite{sobel2004logic}. We have thereby concentrated on
G\"odel's \cite{GoedelNotes}, respectively  Scott's \cite{ScottNotes},
modern
version of this argument, which employs a second-order modal logic, for
which, until now, no theorem provers were available.  In our
computer-assisted study of the argument, the HOL provers LEO-II
\cite{C26} and Satallax \cite{Satallax} have made some interesting
observations, some of which were unknown so far.

Ongoing and future work concentrates on the systematic study of
G\"odel's and Scott's proofs. We have also begun to study more recent
variants of the argument
\cite{anderson90:_some_emend_of_goedel_ontol_proof,AndersonGettings,bjordal99,fuhrmann05:_exist_notwen,fitting02:_types_tableaus_god,Hajek2002,Hajek2008},
which claim to remedy some fundamental problem of G\"odel's and Scott's proofs, known as the modal collapse. The long-term goal
is to work out a  landscape of the detailled logic parameters (e.g., constant
vs. varying domains, rigid vs. non-rigid terms, logics KB vs.
S4 vs.  S5, etc.) under which the proposed variants of the 
modern ontological argument hold or fail.

There is little related work~\cite{oppenheimera11,rushby13}, and this focuses solely on the comparably simpler,
original ontological argument by Anselm of Canterbury.  % These works do not achieve the close
% correspondence between the original formulations and the formal
% encodings that can be found in our approach and they do also not reach
% the same of degree of proof automation.

Our work attests the maturity of contemporary interactive and
automated deduction tools for HOL and
demonstrates the elegance and practical relevance of the
embeddings-based approach. Most importantly, our work opens new
perspectives towards a computational metaphysics.


% \paragraph{Acknowledgement:}
%   The study of G\"odel's ontological proof of God's existence is joint
%   work with Bruno Woltzenlogel Paleo.

\bibliographystyle{plain}
%\bibliography{chris,Bibliography}


\begin{thebibliography}{10}

\bibitem{AndersonGettings}
A.C. Anderson and M.~Gettings.
\newblock G\"odel ontological proof revisited.
\newblock In {\em {G\"odel'96: Logical Foundations of Mathematics, Computer
  Science, and Physics: Lecture Notes in Logic 6}}, pages 167--172. {Springer},
  1996.

\bibitem{anderson90:_some_emend_of_goedel_ontol_proof}
C.A.~Anderson.
\newblock Some emendations of {G{\"o}del's} ontological proof.
\newblock {\em Faith and Philosophy}, 7(3), 1990.

\bibitem{C37}
C. Benzm{\"u}ller.
\newblock Automating quantified conditional logics in {HOL}.
\newblock In F.~Rossi, editor, Proc. of {\em IJCAI 2013}, pages 746--753, Beijing, China, 2013.

\bibitem{C35}
C. Benzm{\"u}ller.
\newblock A top-down approach to combining logics.
\newblock In {\em Proc. of ICAART 2013}, pages 346--351, Barcelona, Spain, 2013.
  SciTePress Digital Library.

\bibitem{J28}
C. Benzm\"uller and B.~Woltzenlogel Paleo.
\newblock {G{\"o}del's God in Isabelle/HOL}.
\newblock {\em Archive of Formal Proofs}, 2013.

\bibitem{C40}
C. Benzm{\"u}ller and B.~Woltzenlogel Paleo.
\newblock Automating {G\"{o}del's} ontological proof of god's existence with
  higher-order automated theorem provers.
\newblock In 
  {\em ECAI 2014}, volume 263 of {\em Frontiers in Artificial Intelligence and
  Applications}, pages 163 -- 168. IOS Press, 2014.

\bibitem{J23}
C. Benzm{\"u}ller and L. Paulson.
\newblock Quantified multimodal logics in simple type theory.
\newblock {\em Logica Universalis (Special Issue on Multimodal Logics)},
  7(1):7--20, 2013.

\bibitem{C26}
C. Benzm{\"u}ller, F. Theiss, L. Paulson, and A. Fietzke.
\newblock {LEO-II} - a cooperative automatic theorem prover for higher-order
  logic (system description).
\newblock In 
  {\em Proc. of IJCAR 2008}, volume 5195 of {\em
  LNCS}, pages 162--170. Springer, 2008.

\bibitem{bjordal99}
F. Bjørdal.
\newblock Understanding gödel’s ontological argument.
\newblock In T. Childers, editor, {\em The Logica Yearbook 1998}.
  Filosofia, 1999.


\bibitem{Satallax}
C.E. Brown.
\newblock Satallax: An automated higher-order prover.
\newblock In {\em Proc. of IJCAR 2012}, number 7364 in LNAI, pages 111 -- 117.
  Springer, 2012.


\bibitem{Church40}
A.~Church.
\newblock A formulation of the simple theory of types.
\newblock {\em Journal of Symbolic Logic}, 5:56--68, 1940.

\bibitem{fitting98}
M.~Fitting and R.L. Mendelsohn.
\newblock {\em First-Order Modal Logic}, volume 277 of {\em Synthese Library}.
\newblock Kluwer, 1998.

\bibitem{fitting02:_types_tableaus_god}
M.~Fitting.
\newblock {\em Types, Tableaus, and {G}{\"o}del's God}.
\newblock Kluwer, 2002.

\bibitem{fuhrmann05:_exist_notwen}
A.~Fuhrmann.
\newblock {Existenz und Notwendigkeit --- Kurt G\"odels axiomatische
  Theologie}.
\newblock In W.~Spohn et~al., editor, {\em Logik in der Philosophie}.
  Heidelberg (Synchron), 2005.

\bibitem{gabbay96}
D.M. Gabbay.
\newblock {\em Labelled Deductive Systems}.
\newblock Clarendon Press, 1996.

\bibitem{GoedelNotes}
K.~G\"odel.
\newblock {\em Appx.A: Notes in Kurt G\"odel's Hand}, pages 144--145.
\newblock In  \cite{sobel2004logic}, 2004.

\bibitem{Hajek2002}
P.~Hajek.
\newblock A new small emendation of g\"odel's ontological proof.
\newblock {\em Studia Logica: An International Journal for Symbolic Logic},
  71(2):pp. 149--164, 2002.

\bibitem{Hajek2008}
P.~Hajek.
\newblock Ontological proofs of existence and non-existence.
\newblock {\em Studia Logica: An International Journal for Symbolic Logic},
  90(2):pp. 257--262, 2008.

\bibitem{oppenheimera11}
P.E. Oppenheimera and E.N. Zalta.
\newblock A computationally-discovered simplification of the ontological
  argument.
\newblock {\em Australasian J. of Philosophy}, 89(2):333--349, 2011.

\bibitem{rushby13}
J.~Rushby.
\newblock The ontological argument in {PVS}.
\newblock In {\em Proc.~of CAV Workshop ``Fun With Formal Methods''}, St.
  Petersburg, Russia,, 2013.

\bibitem{ScottNotes}
D.~Scott.
\newblock {\em Appx.B: Notes in Dana Scott's Hand}, pages 145--146.
\newblock In  \cite{sobel2004logic}, 2004.

\bibitem{sobel2004logic}
J.H. Sobel.
\newblock {\em Logic and Theism: Arguments for and Against Beliefs in God}.
\newblock Cambridge U. Press, 2004.

\bibitem{Stalnaker68}
R.C.~Stalnaker.
\newblock A theory of conditionals.
\newblock In {\em Studies in Logical Theory}, pages 98--112. Blackwell, 1968.

\end{thebibliography}



\end{document}

