

\typeout{Towards Artificial Intelligence for Metaphysics: \\ the Case of the Inconsistency in G\"odel's Ontological Argument}


\documentclass{article}
\usepackage{ijcai15}

% Use the postscript times font!
\usepackage{times}


% the following package is optional:
\usepackage{latexsym} 

\usepackage{amsmath}
\usepackage{xspace}
\usepackage{modallogics}
\usepackage{graphicx,url}
\usepackage{txfonts} % needed for \Diamondblack
\usepackage{color}
\usepackage{xcolor}

\newcommand{\imp}{\rightarrow}
\newcommand{\biimp}{\leftrightarrow}
\newcommand{\allq}{\forall}
\newcommand{\exq}{\exists}
\newcommand{\seq}{\vdash}

\newcommand{\Dia}{\Diamond} % possibly
\newcommand{\BlackBox}{\blacksquare}
\newcommand{\BlackDia}{\Diamondblack}

\newcommand{\NE}{\mathit{NE}}
\newcommand{\ess}[2]{#1 \mathit{ess} #2}
\newcommand{\nec}{\Box}
\newcommand{\pos}{\Dia}


\title{Towards Artificial Intelligence for Metaphysics: \\ the Case of the Inconsistency in G\"odel's Ontological Argument}
\author{Christoph Benzm\"uller\thanks{German Research Foundation DFG \ldots} and Bruno Woltzenlogel Paleo}
\author{}

\begin{document}

\maketitle

\begin{abstract}
  This paper discusses the discovery of the inconsistency in G\"odel's ontological argument as a success story for artificial intelligence. Despite the popularity of the argument since the appearance of G\"odel's manuscript in 1970, the inconsistency of the axioms used in the argument remained unnoticed until 2013, when it was detected automatically by the higher-order theorem prover LEO-II. Understanding and verifying the refutation generated by the prover turned out to be a time-consuming task, and its completion, as reported here, required the reconstruction of the refutation in the Isabelle proof assistant. The development of an improved syntactical hiding for the embedding technique, utilizing Isabelle's binding notation mechanism, allows the refutation to be presented in a human-friendly way, suitable for non-experts in the technicalities of higher-order theorem proving. This brings us a step closer to wider adoption of logic-based artificial intelligence tools by philosophers.
\end{abstract}


\section{Introduction}
Without exaggeration Kurt G\"{o}del's ontological
argument for the existence of God \cite{GoedelNotes,ScottNotes} is
amongst the most discussed formal proofs in modern literature. A rich
body of publications -- including very recent ones -- present,
discuss, assess, criticise, modify and improve G\"{o}del's original
work (see e.g.~Sobel~\shortcite{Sobel} and Oppy~\shortcite{sep-ontological-arguments} and the
references therein).  In philosophy lectures at universities the
argument is regularly presented as a masterpiece argument in
metaphysics. Since 2013, when Benzm\"uller and Woltzenlogel-Paleo~\shortcite{J30,C40} first
reported their successful initial computer-assisted
analysis of G\"odel's proof and Scott's variant,
their work has received a media repercussion on a global scale\footnote{A
  small collection of news articles is available at {\scriptsize
    \url{https://github.com/FormalTheology/GoedelGod/blob/master/Press/LinksToNews.md}}},
and numerous bloggers commented on the proof
\cite{fuhrmann15:_blogg_goedel}.

The in-depth computational analysis presented here substantially
extends previous computer-assisted studies of G\"odel's ontological
argument. Similar to the related work \cite{J30,C40} the analysis has
been conducted with automated theorem provers for classical
higher-order logic (HOL) even though G\"odel's proof is actually
formulated in \emph{modal} higher-order logic. To bridge between the two
logics we utilise and further improve the logic embedding
approach \cite{J23,C40}, which has already been employed succesfully in preceding related work.


The main novel contribution reported in this paper is a detailed analysis (in various modal logics) 
of the inconsistency of G\"{o}del's original version of the axioms used in his proof 
\shortcite{GoedelNotes}. The extraction, reconstruction and verification of an informal, 
human intuitive argument has been an open problem since the first detection of this inconsistency 
by Benzm\"uller and Woltzenlogel-Paleo \shortcite{C40} with the LEO-II prover. 
The verified refutation (discussed in Section \ref{ToDo}) displays a surprisingly accessible
explanation of the inconsistency, which is philosophically
profound and never presented in the
literature. The detection of this inconsistency in combination with
the work reported here thus demonstrates that artificial intelligence systems 
-- particularly higher-order automated theorem provers -- 
are capable of assisting in the discovery and elucidation of
\emph{new} and philosophically relevant knowledge. 

On the technical side, the quest for constructing a compelling refutation, 
capable of convincing also human non-experts, led us to improve the 
syntax of the embedding of modal logics in
Isabelle/HOL (as discussed in Section  \ref{ToDo}). With the new syntax, a nearly perfect match
between the original pen and paper presentations and our encoding in
Isabelle/HOl is feasible. A more user-friendly syntax, as reported here, is clearly an important prerequisite
for promoting the theorem proving technology employed here to a wider community of
philosophers, who are not necessarily experts in automated reasoning or higher-order logic.

Another novel contribution reported here (in Section \ref{ToDo}) is the implementation of an alternative embedding for the only slightly stronger modal logic \SFiveU. Our experiments have shown that the new embedding is more efficient, as the following two open problems can now be solved:
\begin{itemize}
\item Proving the final theorem T3 \textit{``Necessarily, there
    exists God''} fully automatically and directly from the axioms alone, without relying on the argument's lemmas.
\item Verifying automatically the proof of the modal
  collapse \cite{Sobel}, which is one of the most strongly criticized
  `side-effects' of the argument's axioms.
\end{itemize}


\subsection{Related Work}

First successful applications of theorem proving technology in
metaphysics were reported by Oppenheimer and
Zalta~\shortcite{oppenheimera11}, who coined the term \textit{Computational Metaphysics} for this new research area and employed the first-order
\textsc{Prover9} \cite{Prover9} in their experiments. Later on, Rushby~\shortcite{rushby13} used the proof assistant \textsc{PVS} \cite{PVS}. Common to both
works is a significant amount of proof-hand-coding work as well as their
focus on a non-modal formalization of St. Anselm's~\shortcite{Anselm} simpler 
and older ontological argument. 
None of these previous works formalises and automates variants of \emph{higher-order} and \emph{modal} logics, which are, however, crucial
for G\"{o}del's more complex ontological argument.


\section{A Brief History of the Argument}

St. Anselm's ontological argument \cite{Proslogion} can be regarded as the ancestor of modern ontological arguments such as G\"odel's. In the millenium between Anselm and G\"odel, many philosophers modified and arguably improved Anselm's argument. Of particular importance to G\"odel was the work of Leibniz \cite{ToDo}. 
G\"odel's manuscript (Figure \ref{GoedelScript}) can be considered a translation of Leibniz's presentation of the argument into modern modal logic. G\"odel shared his manuscript with Scott, who shared a slightly different version with a larger public. Scott's version of the axioms and definitions, formalized in Isabelle, is shown in Figure \ref{ScottS5}. The main difference to G\"odel's version is the addition of a conjunct in the definition of \emph{essence}. G\"odel's different definition of essence can be seen either in his manuscript (Figure \ref{GoedelScript}) or, in more modern notation, in the Isabelle formalization shown in Figure \ref{nconsistencyIsabelleK}. For Scott, an essential property of an individual must be possessed by him/her. For G\"odel, this is not required. This difference has been considered inessential and merely an oversight by many. For instance, Hazen 1998, p.365 \cite[p.365]{Hazen1998} states that ``G\"odel left this clause out in [11], but this appears to have been an oversight--it is included in related manuscripts''. However, the omitted conjunct is in fact crucial. Without it, G\"odel's original axioms are inconsistent. With it, Scott's axioms are consistent\footnote{In personal communication, Dana Scott confirmed that he was unaware G\"odel's axioms were inconsistent.}.


% There is reason to believe that the omission was more than just an oversight. As pointed out by Fuhrmann \cite{Fuhrmann2005}, ``G\"odel vermerkt diese Konsequenz [dass aus der Definition unmittelbar folgt, da\ss alle wesentlichen Eigenschaften notwendig äquivalent sind] in einer Fußnote zur Definition. Die Definition selbst aber l\"a{\ss}t im Definiens das Konjunkt $Xx$ aus. Ohne das Konjunkt folgt jedoch die Konsequenz nicht. Es ist deshalb naheliegend anzunehmen, da\ss G\"odel die Definition so beabsichtigte, wie sie hier notiert ist und wie G\"odel sie selbst in fr\"uheren Notizbucheintragungen formuliert hat.''\footnote{Our translation: ``G\"odel remarks this consequence [that the definition of essence entails that all essences are necessarily equivalent] in a footnote to the definition. The definition, however, omits the conjunct. ... ToDo''} 
% \marginpar{ToDo: check Fuhrmann's claim. He may be pointing out a second mistake by Gödel.}

For more than four decades, the serious consequences of G\"odel's mistake remained unnoticed, despite intense activity focused on criticizing his argument. Especially since the discovery by Sobel \cite{Sobel} that a modal collapse is entailed by G\"odel's (or also Scott's) axioms, several variants have been proposed \cite{Anderson,AndersonGettings1968,Hajek,Hajek2,Hajek3,Bjordal} attempting to avoid the modal collapse. Many of these variants omit the crucial conjunct in the definition of essence as well\footnote{As these variants also change other axioms, on which the inconsistency of G\"odel's axioms depends, it is not necessarily the case that these variants are also inconsistent.}. Opponents of the argument (e.g. Oppy \cite[p.226/227]{Oppy1996} \cite[p.364]{Oppy2000} \cite[p.1068]{Oppy2008}) have also proposed parodies and other criticisms, referring to variants where the conjunct is omitted.





%  Particularly interesting
% from the perspective of this paper is that many literature
% contributions unfortunately work with or refer to G\"odel's original
% definition of essence
% $$ todo $$
% This original version avoids the conjunct added by Scott expressing
% that essential properties of an individual should be possessed by the
% individual. 

% We give here a list of examples:
% \begin{itemize}
% \item Anderson and Getting's 1996, p. 168 \cite[p.168]{AndersonGettings1968}
%   use essence without conjunct.
% \item  
% \item Look???, p. 514
% % \item Oppy 1996, p.226-227 \cite[p.226/227]{Oppy1996}, Oppy 2000, p. 364
% %   \cite[p.364]{Oppy200}, Oppy 2008, p. 1068 \cite[p.1068]{Oppy2008}:
% %   Oppy uses: ``A is an essence of x iff for every property B, x has B
% %   neces- sarily iff A entails B'' (this is from Anderson's
% %   emendation). Check is this leads to inconsistency as well. In this
% %   case we may write something like: 

% %  Oppy presents an critical
% %   assessment in order to conclude that G\"odel's argument fails to
% %   convince. Oppy apparently does not see that he is assessing an
% %   inconsistent axiom system (and he in fact creates adaptations that
% %   suffer the same problems).
% % \item 
% \end{itemize}








\section{Automating HOML in HOL}

Logic textbooks \cite{ToDo: which} commonly utilize classical higher-order logic (HOL)
\cite{Church} as a meta-language to introduce the syntax and the
semantics of object logics of interest, in which reasoning
problems in concrete application domains can be modeled and solved
with pen and paper. In fact, this approach can also be followed for even very challenging
object logics to
enable interactive and automated theorem proving with existing theorem provers for classical
higher-order logic.

% Just as commonly the case in logic textbooks, in the embeddings
% approach classical higher-order logic is utilised as a meta-language
% to encode the syntax and the semantics of object logics in which the
% proof problems of interest are then modeled in. Modulo the embedding
% state of the art automated theorem provers for classical higher-order
% logic can then be utilised as object logic reasoning tools.

For a computational analysis of G\"odel's ontological argument in this
approach the embedding of higher-order modal logics (HOML) such as K,
KB and S5 with various domain conditions (possibilist and actualist)
is required. This idea has been successfully followed in related work
\cite{C40}. The embedding of higher-order modal logic is in fact
straightforward. Formulas in HMOL are \emph{lifted}, translated to predicates
over worlds, which are themselves explicitly represented as
terms. The logical constants of HMOL are translated to HOL terms in such a way that, for instance, 
%$\neg \varphi$, $\varphi\vee\psi$
$\Box \varphi$ and $\Diamond \varphi$ are mapped, respectively, to the HOL formulas $\forall w. (r w_0 w) \imp (\varphi w)$ and $\exists w. (r w_0 w) \wedge (\varphi w)$. This form of embedding is precisely the well-known standard translation \cite{Ohlbach,ModalLogicPatrickBlackburn},
which is here (intra-logically) realized in HOL by stating a set of
equations defining the logical constants (Fig. ~\ref{QMLS5}). The resulting object logic is the HOML K with rigid terms and constant domains (possibilist quantifiers). Other logics (e.g. KB, S5) can be embedded by adding axioms that restrict the accessibility relation $r$. Varying domains and actualist quantifiers can be simulated by means of an existence predicate, which can be used to guard the quantifiers. Therefore, the embedding approach is very flexible.


\subsection{Improved Embedding for \SFiveU}

The modal logic \SFive requires that the accessibility relation be reflexive, symmetric and transitive. The usual approach to embed \SFive would be to use the standard translation for K described above and state these additional conditions as three HOL axioms: 
\begin{itemize}
\item Reflexivity: $\forall x. (r~x~x)$
\item Symmetry: $\forall x. \forall y. (r~x~y) \rightarrow (r~y~x)$ 
\item Transitivity: $\forall x. \forall y, \forall z. (r~x~y) \wedge (r~y~z) \rightarrow (r~x~z)$
\end{itemize}

We consider here a modal logic that we call \SFiveU, which is characterized by the following condition on the accessibility relation:
\begin{itemize}
\item Universality: $\forall x. \forall y. (r~x~y)$
\end{itemize}

It is easy to see that \SFiveU is at least as strong as \SFive: universality entails reflexivity, symmetry and transitivity. \SFiveU is, in fact, \emph{strictly stronger} than \SFive. \SFiveU only admits complete\footnote{A graph is \emph{complete} iff there is a directed edge connecting every ordered pair of vertices.} frames, whereas \SFive admits non-complete frames as long as all their components are complete.

For \SFiveU, an improved embedding is possible. Universality implies that the guarding predicates in the definitions of $\Box$ and $\Diamond$ always hold. Therefore, they can be omitted and the accessibility relation can be dispensed altogether. The modal operators can then be defined merely as:
\begin{itemize}
\item $\BlackBox \varphi \equiv \forall w. (\varphi w)$ 
\item $\BlackDia \varphi \equiv \exists w. (\varphi w)$
\end{itemize}

The improved embedding of \SFiveU in Isabelle is shown in Fig.~\ref{QMLS5}.
\begin{figure}
\centerline{\includegraphics[width=\columnwidth]{./Images/QMLS5.png}}
\caption{Improved Embedding of \SFiveU} \label{QMLS5}
\end{figure}

\SFiveU is only slightly stronger than \SFive. Most importantly, $\vDash_{\SFive} \varphi$ iff $\vDash_{\SFiveU} \varphi^{U}$, where $\varphi^U$ is obtained from $\varphi$ by replacing all $\Box$ and $\Dia$ by, respectively, $\BlackBox$ and $\BlackDia$. Therefore, \SFiveU can be considered as adequate for metaphysics as \SFive (cf. \cite{Mattey}, \cite[p. 127]{WilliamsonModalLogicAsMetaphysics}, \cite[ToDo]{DunnHardegree}). 


With the improved embedding for \SFiveU, the final theorem T3 (\textit{``Necessarily, there
exists God''}) can be derived from Scott's version of the axioms fully automatically. 
The fully automatic proof has been generated by the theorem prover \textsc{Leo-II}~\cite{C26} 
and subsequently verified in the proof assistant Isabelle/HOL~\cite{NPW02}, as shown in Fig.~\ref{ScottS5}. 
On the other hand, with the embedding used in \shortcite{C40}, 
the provers had to be given the intermediate theorem T2 and the corollary C in order to manage to prove T3. 

Another evidence that the new embedding provides a significant performance boost 
is the successful automatic verification in Isabelle/HOL (with its automatic tactic Meson) of the modal
collapse \cite{Sobel}, which is one of the most strongly criticized
`side-effects' of G\"odel's and Scott's variants of the proof. In previous work \shortcite{C40}
the modal collapse has been proven by the higher-order provers
\textsc{Satallax} \cite{Satallax} and \textsc{Leo-II}, but a fully automatic
verification in the highly trusted Isabelle/HOL still failed \cite{ArchiveFormalProofsGoedelGod}. The success with the new embedding can be seen in Fig.~\ref{ScottS5}.

\begin{figure}
\centerline{\includegraphics[width=\columnwidth]{./Images/ScottS5.png}}
\caption{Full Automation of T3 and MC in \SFiveU } \label{ScottS5}
\end{figure}
\marginpar{ToDo: this picture has to be replaced by one without Isabelle errors}


\subsection{Improved Syntax in Isabelle}

Wider adoption of higher-order theorem proving technology for reasoning about and within embedded object logics, especially among non-expert users, is still hindered by the gap between the syntax used by people, when they write logical formulas with pen and paper, and the syntax used by higher-order logic theorem provers. Even when the syntax of the underlying higher-order system is elegant (as is the case in Isabelle/HOL), the embedding of HOML into HOL may easily expose details of HOL that may be uncommon to the user, disturbing his/her experience while using the system. To illustrate this point, Fig.~\ref{UglyEssence} shows how the definition of essence looked like in previous work \cite{ArchiveFormalProofsGoedelGod}, where advanced syntax-sugaring features were not used. It looks notably higher-order, and its style differs significantly from the common style seen in works on modal logics and the ontological argument. The following specific issues can be enumerated:
\begin{enumerate}
\item Lambda abstractions (which are a typical higher-order feature) appear explicitly in places where it didn't need to appear in a pure HOML formulation (cf. G\"odel's manuscript, Fig.~\ref{GoedelScript}).
\item Quantifiers appear as higher-order defined constants, and not as binders. This forces the user to read (and write) formulas of the form $\forall (\lambda x. A(x))$ instead of the more common $\forall x. A(x)$.
\item The lifted modal connectives are represented by prefixing the letter ``m'' (e.g. $m\wedge$ and $m\imp$). The prefix disturbs the user, as it constantly reminds him/her that there is something unusual about the modal connectives.
\item Higher-order parenthesis conventions for the application of a predicate to a term are used. 
Instead of reading $\psi(y)$, as he would expect, he has to read $(\psi y)$. Outside niche areas in computer science, the former syntax is more widely known than the latter.
\end{enumerate}


\begin{figure}
\centerline{\includegraphics[width=\columnwidth]{./Images/UglyEssence.png}}
\caption{ Definition of Essence using Old Syntax } \label{UglyEssence}
\end{figure}

In the embedding presented here, in Fig.~\ref{QMLS5}, advanced syntax-sugaring effects provided by Isabelle were used to prevent the issues enumerated above. The possibility to define boldface connectives allow us to drop the prefix; ``\texttt{binder}'' annotations enable modal quantifiers to be used in the standard binding way and reduce the need for explicit lambda abstractions; and a careful choice of priorities for infix connectives gives the parenthesis conventions that are more familiar to the user. As desired, the definition of essence in Fig.~\ref{QMLS5} is undeniably more immediately recognizable and comprehensible than the definition in Fig.~\ref{UglyEssence}. The embedding technique is now completely transparent to the user.

We hope that the syntax improvements described here will render the computer-assisted analysis of ontological arguments (as well as future works utilizing the embedding of modal logics into HOL) accessible to a wider audience and encourage philosophers to adopt logic-based artificial intelligence tools in order to conduct similar experiments in computational metaphysics.


\section{Intuitive Inconsistency Argument}

In the typical workflow of trying to prove a conjecture with a theorem prover, it is customary to check the consistency of the axioms first. For if the axioms are inconsistent, anything (including the conjecture) would be trivially derivable in classical logic (\emph{ex falso quodlibet}). Surprisingly, when this routine check was performed on G\"odel's axioms \cite{ECAI}, the LEO-II prover claimed that the axioms were inconsistent. Unfortunately, the refutation generated by LEO-II was barely human-readable. The text file was 153 lines\footnote{Long lines with an average of 184 characters per line.} long and machine-oriented proof calculus (higher-order resolution \cite{}) and syntax (TPTP THF \cite{}) were used. Part of the file is displayed in Fig.~\ref{LEO-Proof}.

\begin{figure*}
\centerline{\includegraphics[width=\textwidth]{./Images/LEO-Proof.png}}
\caption{Lines 115--120 of LEO-II's Refutation. Primitive Substitutions with Empty Property are Highlighted.} \label{LEO-Proof}
\end{figure*}

Although LEO-II's resolution refutation is not easy to read for humans, it did contain
relevant hints to the importance of the empty property (cf. Fig.~\ref{LEO-Proof}).


\subsection{Informal Refutation}

Based on the hints found in LEO-II's refutation, we conceived the following informal explanation for the inconsistency of G\"odel's axioms:

\begin{enumerate}
\item From G\"odel's definition of essence 
(${\ess{\phi}{x} \biimp {\allq \psi} (\psi(x)
\imp {\nec} \allq y (\phi(y) \imp \psi(y)))}$) it follows that the empty property is an essence of every individual (\textbf{Empty Essence Lemma}): 
$$\allq x\,(\ess{\emptyset}{x})$$. 

\item From theorem T1 (positive Properties are possibly
  exemplified: ${\allq \phi} [P(\phi) \imp {\pos}  \exq x
  \phi(x)]$) and axiom A5 (``necessary existence'' is a positive property: $P(\NE)$ ), it follows that $\NE$ is possibly exemplified:
  $$
  \pos \exq x [\NE(x)]
  $$
 
\item Expanding the definition of ``necessary existence'' ( 
  ${\NE(x) \equiv \allq
  \phi [\ess{\phi}{x} \imp \nec
  \exq y \phi(y)]}$), the following is obtained:
  $$
  \pos \exq x [\allq \varphi [ \ess{\varphi}{x} \imp \nec \exq y [\varphi(y)] ] ]
  $$

\item The sentence above holds for all $\varphi$ and thus, in particular, for the empty property:
$$
\pos \exq x [ \ess{\emptyset}{x} \imp \nec \exq y [\emptyset(y)] ]
$$

\item By the Empty Essence Lemma, the antecedent of the implication above is valid. Therefore, the sentence above entails:
$$
\pos \exq x [ \nec \exq y [\emptyset(y)] ]
$$ 

\item By definition of $\emptyset$: 
$$
\pos \exq x [ \nec \bot ]
$$

\item As the existential quantifier is binding no variable within its scope, the sentence is equi-valid with:
$$\pos \nec \bot $$

\item To see that the sentence above is contradictory, we may reason semantically, in terms of possible worlds. If $w_0$ is the arbitrary current world, the $\pos$ operator forces the existence of a world $w$ accessible from $w_0$ such that $\nec \bot$ is true in $w$. But $\nec \bot$ can only be true in $w$, if there is no world $w'$ accessible from $w$. In logics with a reflexive or symmetric accessibility relation (e.g. \KB), it is easy to see that there must be a world $w'$ accessible from $w$: either $w'$ itself, in case of a reflexive relation, or $w_0$, in case of a symmetric relation. In fact, even in \K, with no accessibility condition, there must be a world $w'$ accessible from $w$. The reason is that $\pos \nec \bot$ should be \emph{valid} (true in all worlds). Therefore, it is true in $w$ as well, where the existence of an accessible world $w'$ is forced by the $\pos$ operator. As a model for $\pos \nec \bot$ (which is a consequence of G\"odel's axioms) cannot be built, G\"odel's axioms are inconsistent.
\end{enumerate}

Interestingly, the refutation automatically generated by LEO-II uses a symmetric accessibility relation, and thus requires the modal logic KB. The informal refutation described above, on the other hand, requires only the weaker modal logic K.


\subsection{Argument Reconstruction in Isabelle}

To verify the correctness of the informal argument explained above, it was reconstructed in Isabelle/HOL, using Metis to automate the inessential parts. The essential use of the Empty Essence Lemma, on the other hand, is explicitly stated, to ensure that Isabelle is reconstructing the same argument. In fact, without the help of this lemma, Metis is not strong enough to refute G\"odel's axioms. \marginpar{ToDo: check this claim about Metis.}


\begin{figure}
\centerline{\includegraphics[width=\columnwidth]{./Images/InconsistencyIsabelleK.png}}
\caption{Refutation of G\"odel's Axioms in Isabelle/HOl..} \label{InconsistencyIsabelleK}
\end{figure}





\section{Conclusion}


Good improvement (technical sense), Sledgehammer can still be improved
(LEO-II finds inconsistency when call directly in THF Syntax, but not
from within Isabelle). Without independent experiments directly in THF 
the inconsistency would thus not have been detected.


As we show in this paper using this notion of essence
leads to an inconsistent axiom system, which in a classical setting of
course deserves no further attention, since from an inconsistent
system everything can be concluded, including God's existence and
non-existence simultaneously. In other words, these literature
contributions discuss variants of the argument which from a classical
logical perspective do not deserve any further attention.

We also point to several technical challenges which, at least for the
moment, still restrain non-expert users of these systems from
independently conducting similar experiments in computational
metaphysics as reported here and in previous work.
% Technical issues detected within Isabelle-Sledgehammer; room for improvements



\section*{Acknowledgments}

Will be added at a later point.

%German Research Foundation DFG, Chad Brown


\begin{figure*}
\centerline{\includegraphics[width=\textwidth]{./Images/Manuscript2.png}}
\caption{G\"{o}del's manuscript, with mutually inconsistent axioms and definitions highlighted.} \label{GoedelScript}
\end{figure*}


\appendix


%% The file named.bst is a bibliography style file for BibTeX 0.99c
\bibliographystyle{named}
\bibliography{Bibliography}

\end{document}

