%%%% ijcai15.tex

\typeout{Automating the Ontological Argument}

% These are the instructions for authors for IJCAI-15.
% They are the same as the ones for IJCAI-11 with superficical wording
%   changes only.

\documentclass{article}
% The file ijcai15.sty is the style file for IJCAI-15 (same as ijcai07.sty).
\usepackage{ijcai15}

% Use the postscript times font!
\usepackage{times}

% the following package is optional:
%\usepackage{latexsym} 

% Following comment is from ijcai97-submit.tex:
% The preparation of these files was supported by Schlumberger Palo Alto
% Research, AT\&T Bell Laboratories, and Morgan Kaufmann Publishers.
% Shirley Jowell, of Morgan Kaufmann Publishers, and Peter F.
% Patel-Schneider, of AT\&T Bell Laboratories collaborated on their
% preparation.

% These instructions can be modified and used in other conferences as long
% as credit to the authors and supporting agencies is retained, this notice
% is not changed, and further modification or reuse is not restricted.
% Neither Shirley Jowell nor Peter F. Patel-Schneider can be listed as
% contacts for providing assistance without their prior permission.

% To use for other conferences, change references to files and the
% conference appropriate and use other authors, contacts, publishers, and
% organizations.
% Also change the deadline and address for returning papers and the length and
% page charge instructions.
% Put where the files are available in the appropriate places.

\usepackage{graphicx}


\title{Automating the Ontological Argument}
\author{Christoph Benzmüller\thanks{German Research Foundation DFG \ldots} and Bruno Woltzenlogel Paleo}
\author{}

\begin{document}

\maketitle

\begin{abstract}
  Higher-order automated theorem provers have revealed some
  philosophically profound new knowledge in metaphysics.
\end{abstract}

\section{Introduction}
G\"{o}del's ontological argument for the existence of God is frequently
presented as a masterpiece argument in metaphysics. A large body of
literature is available which discusses G\"{o}del's original work and/or
subsequent variants of it.

This paper presents an in depth analysis of G\"{o}del's original version
of the ontological argument.   

The analysis has been conducted with automated theorem provers for
classical higher-order logic (HOL). However, since higher-order modal
logic is actually required for the encoding of G\"{o}del's argument, we
utilize an semantic embedding of HOML in HOL.


First applications of theorem proving technology in metaphysics
appeared in 199? in the work Zalta and Oppenheimer.  Further work also
includes Rushby \ldots. Different to what is presented here, none of
these previous formalises and automates variants of HOML, which is
however crucial for G\"{o}del's ontological argument.


The work presented here substantially extends previous work \cite{}.  
The novel contributions of this paper include:

\begin{itemize}
\item Detailed analysis of the inconsistency of G\"{o}del's axioms. The
inconsistency was detected before, but for a long time we were not
able to extract an intuitive argument from the proof of the ATPs.
Further experiments now led to discovery of a surprisingly accessible
understanding of this inconsistency which is philosophically profound
and which has not been presented yet in the respective literature.

\item Fully automated proof of Scott's theorem T3 from the axioms, that
   is, without intermediate argumentation steps as used in previous
   work.

\item Direct verification of modal collapse with Meson in Isabelle/HOL,
which did not work for the previous.

\item On a technical level we show that the novel embedding of S5 is more
    effective than the previous one.

\item Much improved syntax presentation.

\item Technical issues detected within Isabelle-Sledgehammer; room for improvements
\end{itemize}

These novel contributions are presented in Sections ? and ?.

In Section ?, novel and previous findings are summarised and discussed 
with regard to their relevance in metaphysics. In this section also 
some further details and improvements on earlier 
findings are presented.

Section ? concludes the paper.



\section{Variants of the Ontological Argument}
Short history, Anselm, Leibniz, etc.
\subsection{G\"{o}del's Proof Script}
\subsection{Scott's Variant}
\subsection{Recent Variants}
Dozens of contributions were published in the last four decades which
present variants of G\"odel's/Scott's proofs. In many cases the
motivation has been to remedy the modal collapse. Other authors are
simply interested in a critical assessment. Particularly interesting
from the perspective of this paper is that many literature
contributions unfortunately work with or refer to G''odel's original
definition of essence
$$ todo $$
This original version avoids the conjunct added by Scott expressing
that essential properties of an individual should be possessed by the
individual. As we show in this paper using this notion of essence
leads to an inconsistent axiom system, which in a classical setting of
course deserves no further attention, since from an inconsistent
system everything can be concluded, including God's existence and
non-existence simultaneously. In other words, these literature
contributions discuss variants of the argument which from a classical
logical perspective do not deserve any further attention.

We give here a list of examples:
\begin{itemize}
\item Anderson and Getting's 1996, p. 168 \cite[p.168]{AndersonGettings1968}
  use essence without conjunct.
\item  Hazen 1998, p.365 \cite[p.365]{Hazen1998} states: ``G\"odel left this
  clause out in [11], but this appears to have been an oversight--it
  is included in related manuscripts''. There is not mentioning of an
  inconsistency though.
\item Look???, p. 514
% \item Oppy 1996, p.226-227 \cite[p.226/227]{Oppy1996}, Oppy 2000, p. 364
%   \cite[p.364]{Oppy200}, Oppy 2008, p. 1068 \cite[p.1068]{Oppy2008}:
%   Oppy uses: ``A is an essence of x iff for every property B, x has B
%   neces- sarily iff A entails B'' (this is from Anderson's
%   emendation). Check is this leads to inconsistency as well. In this
%   case we may write something like: 

%  Oppy presents an critical
%   assessment in order to conclude that G\"odel's argument fails to
%   convince. Oppy apparently does not see that he is assessing an
%   inconsistent axiom system (and he in fact creates adaptations that
%   suffer the same problems).
% \item 
\end{itemize}


\footnote{Fuhrmann2005: Gödel vermerkt diese Konsequenz [dass aus der Definition unmittelbar folgt, daß alle wesentlichen
Eigenschaften notwendig äquivalent sind] in einer Fußnote zur Definition. Die Definition selbst aber läßt im Definiens das Konjunkt Xx aus. Ohne das Konjunkt folgt jedoch die Konsequenz nicht. Es ist deshalb naheliegend anzunehmen, daß Gödel die Definition so beabsichtigte, wie sie hier notiert ist und wie Gödel sie selbst in früheren Notizbucheintragungen formuliert hat.}





\section{Automating HOML in HOL}
\subsection{Outline of the Embedding}
Logic textbooks commonly utilise classical higher-order logic (HOL)
\cite{Church} as a meta-language to introduce the syntax and the
semantics of object logics of interest, in which then reasoning
problems in concrete application domains can be modeled and solved
with pen and paper.  In fact, this approach can also be followed to
enable interactive and automated proof for even very challenging
objects logics with existing theorem provers for classical
higher-order logic.

% Just as commonly the case in logic textbooks, in the embeddings
% approach classical higher-order logic is utilised as a meta-language
% to encode the syntax and the semantics of object logics in which the
% proof problems of interest are then modeled in. Modulo the embedding
% state of the art automated theorem provers for classical higher-order
% logic can then be utilised as object logic reasoning tools.

For a computational analysis of G\"odel's ontological argument in this
approach the embedding of modal higher-order logics (MHOL) such as K,
KB and S5 for different domain conditions (possibilist and actualist)
is required. This idea has been successfully followed in related work
\cite{C40}. The embedding of modal higher-order logic is in fact
straightforward. Formulas in MHOL such as are lifted to predicates
over worlds, which are explicitly represented in the approach as
terms. Propositional MHOL such as $\neg \varphi$, $\varphi\vee\psi$
and $\box \varphi$ are then mapped to HOL formulas \ldots.  

It is easy
to see that this captures the standard translation \cite{Ohlbach},
which is here (intra-logically) realised in HOL by stating a set of
equations; see also Fig. ~\ref{QMLS5}. New is that also quantifiers 
can be mapped \ldots 

The approach is very flexible, KB, S5, domain conditions\ldots 


\subsection{Improved Embedding for S5}
We present an improved embedding (syntax and defintion of box) in  Fig.~\ref{QMLS5}).
\begin{figure}
\centerline{\includegraphics[width=\columnwidth]{./Images/QMLS5.png}}
\caption{Improved Embedding of S5} \label{QMLS5}
\end{figure}


\section{Full Automation of Scott's Proof}
Within the modified logic S5 Scott's proof can be fully automated. The
intermediate argumentation steps as not needed anymore to support the
prover, see Fig.~\ref{ScottS5}).
\begin{figure}
\centerline{\includegraphics[width=\columnwidth]{./Images/ScottS5.png}}
\caption{Full Automation of the Ontological Argument in S5 } \label{ScottS5}
\end{figure}




In the modified S5 setting also further results can be automatically verified, which 
failed in previous work: verification of  modal collapse in
Isabelle/HOL with Meson.

\section{Intuitive Inconsistency Argument}
\subsection{LEO-II's Inaccessible Inconsistency Proof}
Tell the story (here or earlier). Show parts of LEO proof.  Inaccessible, but relevant knowledge
is contained.
\begin{figure*}
\centerline{\includegraphics[width=\textwidth]{./Images/LEO-Proof.png}}
\caption{Primitive substitution in LEO-II generates candidates for the
empty property.} \label{LEO-Proof}
\end{figure*}
LEO-II's resolution proof is human unintuitive. However, it contained
relevant hints to the empty property (cf. Fig.~\ref{LEO-Proof}).

\subsection{Argument Reconstruction in Isabelle}
Present the argument informal and in Isabelle (see
Fig.~\ref{InconsistencyIsabelleK}). Easy to understand in 
in KB and KT, slightly harder in K.
\begin{figure}
%\centerline{\includegraphics[width=\columnwidth]{./Images/InconsistencyIsabelleK.png}}
\caption{Reconstruction of inconsistency in Isabelle/HOl..} \label{InconsistencyIsabelleK}
\end{figure}


\subsection{Mapping to Gödel's Script}
\begin{figure*}
\centerline{\includegraphics[width=\textwidth]{./Images/Manuscript2.png}}
\caption{Sources of Inconsistence in G\"{o}del's proof script.} \label{GoedelScript}
\end{figure*}


\section{Conclusion}


Good improvement (technical sense), Sledgehammer can still be improved
(LEO-II finds inconsistency when call directly in THF Syntax, but not
from within Isabelle). Without independent experiments directly in THF 
the inconsistency would thus not have been detected.

\section*{Acknowledgments}

Will be added at a later point.

%German Research Foundation DFG, Chad Brown

\appendix

\section{\LaTeX{} and Word Style Files}\label{stylefiles}

The \LaTeX{} and Word style files are available on the IJCAI--15
website, {\tt http://www.ijcai-15.org/}.
These style files implement the formatting instructions in this
document.

The \LaTeX{} files are {\tt ijcai15.sty} and {\tt ijcai15.tex}, and
the Bib\TeX{} files are {\tt named.bst} and {\tt ijcai15.bib}. The
\LaTeX{} style file is for version 2e of \LaTeX{}, and the Bib\TeX{}
style file is for version 0.99c of Bib\TeX{} ({\em not} version
0.98i). The {\tt ijcai15.sty} file is the same as the {\tt
ijcai07.sty} file used for IJCAI--07.

The Microsoft Word style file consists of a single file, {\tt
ijcai15.doc}. This template is the same as the one used for
IJCAI--07.

These Microsoft Word and \LaTeX{} files contain the source of the
present document and may serve as a formatting sample.  

Further information on using these styles for the preparation of
papers for IJCAI--15 can be obtained by contacting {\tt
pcchair15@ijcai.org}.

%% The file named.bst is a bibliography style file for BibTeX 0.99c
\bibliographystyle{named}
\bibliography{ijcai15}

\end{document}

