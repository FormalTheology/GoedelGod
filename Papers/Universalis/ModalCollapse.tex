% ------------------------------------------------------------------------
% bjourdoc.tex for birkjour.cls*******************************************
% ------------------------------------------------------------------------
%%%%%%%%%%%%%%%%%%%%%%%%%%%%%%%%%%%%%%%%%%%%%%%%%%%%%%%%%%%%%%%%%%%%%%%%%%

\documentclass{birkjour}

\usepackage[utf8]{inputenc}

\usepackage{url}
\usepackage{commands}
%
%
% THEOREM Environments (Examples)-----------------------------------------
%
 \newtheorem{thm}{Theorem}[section]
 \newtheorem{cor}[thm]{Corollary}
 \newtheorem{lem}[thm]{Lemma}
 \newtheorem{prop}[thm]{Proposition}
 \theoremstyle{definition}
 \newtheorem{defn}[thm]{Definition}
 \theoremstyle{remark}
 \newtheorem{rem}[thm]{Remark}
 \newtheorem*{ex}{Example}
 \numberwithin{equation}{section}

\def\HOML{\entity{HOML}\xspace}
\def\HOL{\entity{HOL}\xspace}


\begin{document}

%-------------------------------------------------------------------------
% editorial commands: to be inserted by the editorial office
%
%\firstpage{1} \volume{228} \Copyrightyear{2004} \DOI{003-0001}
%
%
%\seriesextra{Just an add-on}
%\seriesextraline{This is the Concrete Title of this Book\br H.E. R and S.T.C. W, Eds.}
%
% for journals:
%
%\firstpage{1}
%\issuenumber{1}
%\Volumeandyear{1 (2004)}
%\Copyrightyear{2004}
%\DOI{003-xxxx-y}
%\Signet
%\commby{inhouse}
%\submitted{March 14, 2003}
%\received{March 16, 2000}
%\revised{June 1, 2000}
%\accepted{July 22, 2000}
%
%
%
%---------------------------------------------------------------------------
%Insert here the title, affiliations and abstract:
%


\title[Modal Collapse]
 {The Ontological Modal Collapse \\ 
 as a Collapse of the Square of Opposition}



%----------Author 1
\author[Benzm\"uller]{Christoph Benzm\"uller}

\address{%
Department of Mathematics and Computer Science\\
Arnimallee 7 \\
Room 115 \\
14195 Berlin \\
Germany
}

\email{c.benzmueller@gmail.com}

%\thanks{This work was completed with the support of ToDo}


%----------Author 3
\author[Woltzenlogel-Paleo]{Bruno Woltzenlogel Paleo}
\address{ 
Favoritenstra{\ss}e 9 \\
Room HA0402 \\
1040 Wien \\
Austria
}
\email{bruno.wp@gmail.com}




%----------classification, keywords, date
\subjclass{
Prim. 03A02;  % Philosophical aspects of logic and foundations
Sec. 68T02 % Artificial Intelligence
}

\keywords{Modal Logics, Higher-Order Logics, Ontological Argument}

\date{August 30, 2014}
%----------additions
\dedicatory{ }
%%% ----------------------------------------------------------------------

\begin{abstract}
The \emph{modal collapse} that afflicts G\"odel's modal ontological 
argument for God's existence is discussed from the perspective of the 
modal square of opposition.
\end{abstract}

%%% ----------------------------------------------------------------------
\maketitle
%%% ----------------------------------------------------------------------
%\tableofcontents
\section{Introduction}

Attempts to prove the
existence (or non-existence) of God by means of abstract, ontological
arguments are an old tradition in western philosophy, with contributions by several prominent philosophers, including St. Anselm of
Canterbury, Descartes and Leibniz. Kurt G{\"o}del and Dana Scott studied and further improved this argument, bringing it to a mathematically more precise form, as a chain of axioms, lemmas and theorems in a second-order modal logic \cite{GoedelNotes,ScottNotes}, shown in Fig. \ref{fig:scott}.

\begin{figure}[t]
\noindent \framebox[\columnwidth][r]{
\begin{minipage}{.94\columnwidth}\small
\begin{itemize}
\item[\textbf{A1}] Either a property or its negation is positive, but not
  both:
  $$\hol{\allq \varphi [P(\neg \varphi) \biimp \neg P(\varphi)]}$$ 
\item[\textbf{A2}] A property necessarily implied by a
  positive property is positive:
  $$\hol{\allq \varphi \allq \psi [(P(\varphi) \wedge \nec \allq x [\varphi(x)
  \imp \psi(x)]) \imp P(\psi)]}$$
\item[\textbf{T1}] Positive properties are possibly exemplified: 
  $$\hol{\allq \varphi [P(\varphi) \imp \pos \exq x \varphi(x)]}$$ 
\item[\textbf{D1}] A \emph{God-like} being possesses all positive properties: 
  $$\hol{G(x) \equiv \forall \varphi [P(\varphi) \imp \varphi(x)]}$$ 
\item[\textbf{A3}]  The property of being God-like is positive: 
  $$\hol{P(G)}$$
\item[\textbf{C\phantom{1}}] Possibly, a God-like being exists: $$\hol{\pos \exq x G(x)}$$
\item[\textbf{A4}]  Positive properties are necessarily positive: 
  $$\hol{\allq \varphi [P(\varphi) \imp \Box \; P(\varphi)]}$$ 
\item[\textbf{D2}] An \emph{essence} of an individual is a property possessed by it and necessarily implying any of its properties: $$\hol{\ess{\varphi}{x} \equiv \varphi(x) \wedge \allq
  \psi (\psi(x) \imp \nec \allq y (\varphi(y) \imp \psi(y)))}$$ 
\item[\textbf{T2}]  Being God-like is an essence of any
  God-like being: $$\hol{\allq x [G(x) \imp \ess{G}{x}]}$$
\item[\textbf{D3}] \emph{Necessary existence} of an individual is the necessary exemplification of all its essences: 
  $$\hol{\NE(x) \equiv \allq \varphi [\ess{\varphi}{x} \imp \nec
  \exq y \varphi(y)]}$$
\item[\textbf{A5}] Necessary existence is a positive property: $$\hol{P(\NE)}$$ 
\item[\textbf{L1}] If a god-like being exists, then necessarily a god-like being exists: 
  $$\hol{\exq x G(x) \imp \nec \exq y G(y)}$$
\item[\textbf{L2}] If possibly a god-like being exists, then necessarily a god-like being exists: 
  $$\hol{\pos \exq x G(x) \imp \nec \exq y G(y)} $$
%
\item[\textbf{T3}] Necessarily, a God-like being exists: $$\hol{\nec \exq x G(x)}$$ 
\end{itemize}
\end{minipage}
} \vskip-.5em
\caption{Scott's version of G\"odel's ontological argument \cite{ScottNotes}.\label{fig:scott}} 
\end{figure}


% Ontological arguments, for or against the existence of God,
% illustrate well an essential aspect of metaphysics: some (necessary) facts
% for our existing world are deduced by purely a priori, analytical means from some
% abstract definitions and axioms. % Contingent truths are to be
% distinguished from necessary truths.


G\"{o}del defines God as a being who possesses all \emph{positive}
properties and states a few reasonable (but debatable) axioms that
such properties should satisfy.  The overall idea of G{\"o}del's proof
is in the tradition of Anselm's argument, who defined God as some
entity of which nothing greater can be conceived. Anselm argued that
existence in the actual world would make such an assumed being even
greater (more perfect); hence, by definition, God must exist. However,
for Anselm existence was treated as a predicate and the possibility of
God's existence was assumed as granted. These issues were criticized
by Kant and Leibniz, respectively, and they were addressed in the work
of G\"odel.\footnote{C.: I deleted "successfully", since I am still
  not totally convinced yet about the existence as predicate issue in
  G\"odel's work; I think we should not be conclusive at this point.}

Nevertheless, G{\"o}del's work still leaves room for criticism. In
particular, his axioms are so strong that they entail a \emph{modal
  collapse} \cite{Sobel1987,sobel2004logic}: everything that is the
case is so necessarily.  There has been an impressive body of recent
and ongoing work
(cf.~\cite{sobel2004logic,Fitting,anderson90:_some_emend_of_goedel_ontol_proof,AndersonGettings,bjordal99,fuhrmann05:_exist_notwen,Hajek2002,Hajek2008,ContemporaryBibliography}
and the references therein) proposing solutions for the modal
collapse.  The goal of this contribution is to discuss the modal
collapse from the point of view of the modal square of opposition.


\section{A Collapse of the Modal Square}

A crucial step of most ontological arguments is the claim that if
God's existence is possible, then it is necessary.  This is Lemma
\textbf{L2} in G\"odel's proof.  In the modal square of opposition
(Fig. \ref{fig:square}), this is an unusual situation in which the
\textbf{I} corner must imply and entail the \textbf{A} corner, in the
particular case when $\phi$ is $\exq x G(x)$.  G\"odel's proof shows
that his axioms are indeed strong enough to invert the direction of
entailment this choice of $\phi$.
% for the sentence at issue.  
This observation, however, immediately leads to the question whether
the axioms are eventually even strong enough to enable the inverted entailment
for arbitrary sentences $\phi$.  That is essentially the question
asked by Sobel \cite{Sobel1987}, and his proof of the modal collapse
(\textbf{MC}, cf. Fig.~\ref{fig:collapse}) provides an affirmative answer. It is possible to show
that this form of the modal collapse entails (in modal logic
\textbf{\emph{K}}) a collapse of the modal square (\textbf{MC'}),
causing the subcontraries to entail (and even imply) their respective
contraries. Normally, as shown in Fig. \ref{fig:square}, in the modal
square of opposition only the other direction of entailment holds: the
contraries entail their subcontraries, assuming the \emph{modal
  existential import} \textbf{ExImp}
\cite{WhatToCiteHere}.\footnote{C.: We need a reference here; I am still
  searching for good papers; this here is also interesting:
  \url{http://www.tandfonline.com/doi/pdf/10.1080/01445340.2013.764962}}

Moreover, in any modal logic where the axiom \textbf{T} holds
(i.e. where the accessibility relation is reflexive), 
even a total collapse of the modalities (\textbf{MC''}) 
is entailed by \textbf{MC}.
Interestingly, under this stronger form of modal
collapse, the contraries entail their subcontraries even 
without the existential import.

\begin{figure}[t]
\includegraphics[width=0.9\textwidth]{./SquarePics/ModalSquare.png}
\caption{Modal Square of Opposition.}
\label{fig:square}
\end{figure}



\begin{figure}[t]
\noindent \framebox[\columnwidth][r]{
\begin{minipage}{.94\columnwidth}\small
\begin{itemize}
\item[\textbf{MC}] Everything that is the case is so necessarily:
  $\hol{\allq \phi [\phi \imp \nec \phi]}$ \\
\item[\textbf{MC'}] Everything that is possible is necessary:
  $\hol{\allq \phi [\pos \phi \imp \nec \phi]}$  \\
\item[\textbf{T}] Everything that is necessary is the case:
  $\hol{\allq \phi [\nec \phi \imp \phi]}$ \\
\item[\textbf{ExImp}] (Modal Existential Import):
  $\hol{\pos \top}$ \\
\item[\textbf{AI}] Everything that is necessary is possible:
  $\hol{\allq \phi [\nec \phi \imp \pos \phi]}$ \\
\item[\textbf{MC''}] Modalities collapse completely:
  $\hol{\allq \phi [(\phi \biimp \nec \phi) \wedge (\pos \phi \biimp \nec \phi)]}$
\end{itemize}
\end{minipage}
} \vskip-.5em
\caption{Modal Collapse}
\label{fig:collapse} 
\end{figure}

\newcommand{\collapse}[1]{\mathit{collapse}(#1)}

Although G\"odel's axioms lead to modal collapse, 
there are several variants (e.g. \cite{anderson90:_some_emend_of_goedel_ontol_proof,AndersonGettings,bjordal99}) 
that are known to be immune to it. 
This means there must be at least one proposition $\phi$ 
such that the implication $\phi \imp \nec \phi$ 
(from now on abbreviated as $\collapse{\phi}$) is not valid 
under the axioms and definitions used by the variant. 
But if the variant is sufficiently similar to G\"odel's argument, 
following the path deriving\footnote{C.: I don't know the phrase "path deriving", please check.} Lemmas \textbf{L1} and 
\textbf{L2}, then 
$\collapse{\exq x G(x)}$ must be valid. 
Therefore, one may wonder how strong is their immunity to 
the modal collapse: is there any other proposition $\phi$ 
for which $\collapse{\phi}$ is also valid?

For Anderson's emendation \cite{anderson90:_some_emend_of_goedel_ontol_proof}, for example, a form of the modal collapse (\textbf{A:MC}), restricted to positive properties applied to god-like beings, can be derived. The proof, under the modal logic \textbf{\emph{K}}, depends only on Anderson's alternative definition of god-like being (\textbf{A:D1}). This class of propositions for which the collapse occurs is tight: weaker restrictions (\textbf{A:MC1} and \textbf{A:MC2}), which could lead to larger classes, are counter-satisfiable. These results hold under both constant and varying domain quantification, with possibilist and actualist quantifiers.


\begin{figure}[t]
\noindent \framebox[\columnwidth][r]{
\begin{minipage}{.94\columnwidth}\small
\begin{itemize}
% \item[A:A1] If a property is positive, its negation is not positive:
%   $$\hol{\allq \varphi [P(\varphi) \imp \neg P(\neg \varphi)]}$$ 
% \item[A2] A property necessarily implied by a
%   positive property is positive:
%   $$\hol{\allq \varphi \allq \psi [(P(\varphi) \wedge \nec \allq x [\varphi(x)
%   \imp \psi(x)]) \imp P(\psi)]}$$
% \item[T1] Positive properties are possibly exemplified: 
%   $$\hol{\allq \varphi [P(\varphi) \imp \pos \exq x \varphi(x)]}$$ 
\item[\textbf{A:D1}] A \emph{God-like} being necessarily possesses those and only those properties that are positive: 
  $$\hol{G_A(x) \equiv \forall \varphi [P(\varphi) \biimp \nec \varphi(x)]}$$ 
\item[\textbf{A:MC}] The modal collapse happens for any positive properties applied to any god-like being:
  $$
  \hol{\allq \varphi \allq x [(P(\varphi) \wedge G_A(x)) \imp \collapse{\varphi(x)}]}
  $$
\item[\textbf{A:MC1}] The modal collapse does \emph{not} happen for positive properties applied to arbitrary individuals (\emph{counter-satisfiable}):
  $$
  \hol{\allq \varphi \allq x [P(\varphi) \imp \collapse{\varphi(x)}]}
  $$
\item[\textbf{A:MC2}] The modal collapse does \emph{not} happen for an arbitrary properties applied to a god-like being (\emph{counter-satisfiable}):
  $$
  \hol{\allq \varphi \allq x [G_A(x) \imp \collapse{\varphi(x)}]}
  $$
% \item[A3']  The property of being God-like is positive: 
%   $$\hol{P(G_A)}$$
% \item[C\phantom{1}] Possibly, a God-like being exists: $$\hol{\pos \exq x G(x)}$$
% \item[A4]  Positive properties are necessarily positive: 
%   $$\hol{\allq \varphi [P(\varphi) \imp \Box \; P(\varphi)]}$$ 
% \item[A:D2] An \emph{essence} of an individual is a property that necessarily implies those and only those properties that the individual has necessarily: $$\hol{\essA{\varphi}{x} \equiv \allq
%   \psi [\nec \psi(x) \biimp \nec \allq y (\varphi(y) \imp \psi(y))]}$$ 
% \item[T2']  Being God-like is an essence of any
%   God-like being: $$\hol{\allq x [G_A(x) \imp \essA{G_A}{x}]}$$
% \item[D3'] \emph{Necessary existence} of an individual is the necessary exemplification of all its essences: 
%   $$\hol{\NE_A(x) \equiv \allq \varphi [\essA{\varphi}{x} \imp \nec
%   \exq y \varphi(y)]}$$
% \item[A5'] Necessary existence is a positive property: $$\hol{P(\NE_A)}$$ 
% \item[L1'] If a god-like being exists, then necessarily a god-like being exists: 
%   $$\hol{\exq x G_A(x) \imp \nec \exq y G_A(y)}$$
% \item[L2'] If possibly a god-like being exists, then necessarily a god-like being exists: 
%   $$\hol{\pos \exq x G_A(x) \imp \nec \exq y G_A(y)} $$
% %
% \item[T3'] Necessarily, a God-like being exists: $$\hol{\nec \exq x G_A(x)}$$ 
\end{itemize}
\end{minipage}
} \vskip-.5em
\caption{Restricted Collapse for Anderson's Emendation \cite{anderson90:_some_emend_of_goedel_ontol_proof}}
\label{fig:anderson}
\end{figure}


%ToDo: write a similar paragraph about Bjordal's alternative??


Independently of the variant of the ontological argument under
consideration, the following can be said about the class of collapsing
propositions:
\begin{enumerate}
\item Valid propositions are collapsing: if $\phi$ is valid, then $\collapse{\phi}$ is valid.
%
\item The class of collapsing propositions is closed under logical equivalence: if $\collapse{\phi}$ is valid and $\phi \biimp \phi'$ is valid, then $\collapse{\phi'}$ is valid.
%
\item The class of collapsing propositions is not generally closed under equi-validity: even if $\collapse{\phi}$ is valid and $\phi$ and $\phi'$ are equi-valid, $\collapse{\phi'}$ may not be valid.
%
\item The class of collapsing propositions is not generally closed under implication: even if $\collapse{\phi}$ is valid and $\phi \imp \phi'$ is valid, $\collapse{\phi'}$ may not be valid.
\end{enumerate}


%\clearpage



\section{Final Remarks}

All results announced in this note have been obtained experimentally
using interactive and automated theorem provers and model finders
\cite{LEO,Satallax,Isabelle,Coq,Nitpick}.  The source codes of the
experiments, as well as the resulting proofs and counter-models, are
available in
\url{https://github.com/FormalTheology/GoedelGod/tree/master/Formalizations/Isabelle}.\footnote{C.: Here
  we should point to the exact files (and put them in a relatively
  stable subdirectory?). We also should also use corresponding names in
  these files, so that the reader easily finds the related 
  parst. Right now this is still quite difficult.}

The technique enabling these experiments is the embedding of 
quantified modal logics into higher-order logics 
\cite{J23,B9,C36}, for which automated theorem provers exist. 
This technique has already been successfully employed in the 
verification and reconstruction of G\"odel's proof 
\cite{J28,J30,W50,J29}, and a detailed 
mathematical description is available in \cite{C40}.

The modal collapse is an interesting example of philosophical 
controversy and dispute, to which we can apply Leibniz's idea 
of a \emph{calculus ratiocinator} brought to reality in the 
form of contemporary automated theorem provers. 
A significant advantage provided by the use of computers is 
that all conditions (e.g. modal logic, domain conditions, 
semantics) under which the announced results hold must be 
explicitly specified in the source code. This reduces the need 
for interpretation and the danger of misunderstandings. 
Current technology is increasingly ready to be embraced by 
those willing to practice computer-assisted theoretical 
philosophy \cite{oppenheimera11,rushby13}.

Ongoing and future work includes the study of modal collapse in
further variants of the ontological argument, e.g.~\cite{bjordal99,fuhrmann05:_exist_notwen}.
 




% Various slightly different versions of
% axioms and definitions have been considered by G\"{o}del and by
% several philosophers who commented on his proof
% (cf.~\cite{sobel2004logic,anderson90:_some_emend_of_goedel_ontol_proof,AndersonGettings,Fitting,Adams,ContemporaryBibliography}).



% In theoretical philosophy, formal logical confrontations with such
% ontological arguments had been so far (mainly) limited to paper
% and pen.  Up to now, the use of computers was prevented, because the
% logics of the available theorem proving systems were not expressive
% enough to formalize the abstract concepts adequately. G{\"o}del's proof
% uses, for example, a complex higher-order modal logic (\HOML) to handle
% concepts such as \emph{possibility} and \emph{necessity} and to support
% quantification over individuals and properties.

% controversies, care with parameters


\subsection*{Acknowledgment}

We would like to thank Paul Weingartner, Andr\'{e} Fuhrmann and Melvin Fitting for several discussions about G\"odel's ontological argument.

ToDo: who else?

\subsection*{Note about authorship}

Alphabetic order has been used for the authors' names. The extent and
kind of contribution of each author cannot be inferred from the order.

\bibliographystyle{plain}
\bibliography{Bibliography}


%\begin{thebibliography}{1}


% \bibitem{Adams}
% R.M. Adams, `Introductory note to *1970', in {\em {Kurt G\"odel: Collected
%   Works Vol. 3: Unpubl. Essays and Letters}}, Oxford Univ. Press, (1995).

% \bibitem{AndersonGettings}
% A.C. Anderson and M.~Gettings, `G\"odel ontological proof revisited', in {\em
%   {G\"odel'96: Logical Foundations of Mathematics, Computer Science, and
%   Physics: Lecture Notes in Logic 6}},  167--172, {Springer}, (1996).

% \bibitem{anderson90:_some_emend_of_goedel_ontol_proof}
% C.A. Anderson, `Some emendations of {G{\"o}del's} ontological proof', {\em
%   Faith and Philosophy}, {\bf 7}(3), (1990).

% \bibitem{Andrews:gmae72}
% P.B. Andrews, `General models and extensionality', {\em Journal of Symbolic
%   Logic}, {\bf 37}(2),  395--397, (1972).

% \bibitem{andrewsSEP}
% P.B. Andrews, `Church's type theory', in {\em The Stanford Encyclopedia of
%   Philosophy}, ed., E.N. Zalta, spring 2014 edn., (2014).

% \bibitem{C36}
% C.~Benzm{\"u}ller, `{HOL} based universal reasoning', in {\em Handbook of the
%   4th World Congress and School on Universal Logic}, ed., J.Y. Beziau~et al.,
%   pp. 232--233, Rio de Janeiro, Brazil, (2013).

% \bibitem{B5}
% C.~Benzm{\"u}ller and D.~Miller, `Automation of higher-order logic', in {\em
%   Handbook of the History of Logic, Volume 9 --- Logic and Computation},
%   Elsevier, (2014).
% \newblock Forthcoming; preliminary version available at
%   {http://christoph-benzmueller.de/papers/B5.pdf}.

% \bibitem{C34}
% C.~Benzm{\"u}ller, J.~Otten, and Th. Raths, `Implementing and evaluating
%   provers for first-order modal logics', in {\em Proc. of the 20th European
%   Conference on Artificial Intelligence (ECAI)}, pp. 163--168, (2012).

% \bibitem{B9}
% C.~Benzm{\"u}ller and L.C. Paulson, `Exploring properties of normal multimodal
%   logics in simple type theory with {LEO-II}', in {\em {Festschrift in Honor of
%   {Peter B. Andrews} on His 70th Birthday}}, ed., C.~Benzm{\"u}ller~et al.,
%   386--406, College Publications, (2008).

% \bibitem{J23}
% C.~Benzm{\"u}ller and L.C. Paulson, `Quantified multimodal logics in simple
%   type theory', {\em Logica Universalis}, {\bf 7}(1),  7--20, (2013).

% \bibitem{LEO-II}
% C.~Benzm{\"u}ller, F.~Theiss, L.~Paulson, and A.~Fietzke, `{LEO-II} - a
%   cooperative automatic theorem prover for higher-order logic', in {\em
%   Proc.~of IJCAR 2008}, number 5195 in LNAI, pp. 162--170. Springer, (2008).

% \bibitem{J30}
% C.~Benzm{\"u}ller and B.~Woltzenlogel-Paleo, `Formalization, mechanization and
%   automation of {G{\"o}del's} proof of {God's} existence', {\em
%   arXiv:1308.4526}, (2013).

% \bibitem{J28}
% C.~Benzm\"uller and B.~Woltzenlogel-Paleo, `{G{\"o}del's God in Isabelle/HOL}',
%   {\em Archive of Formal Proofs}, (2013).

% \bibitem{W50}
% C.~Benzm\"uller and B.~Woltzenlogel-Paleo, `G\"odel's {God} on the computer',
%   in {\em Proceedings of the 10th International Workshop on the Implementation
%   of Logics}, EPiC Series. EasyChair, (2013).
% \newblock Invited abstract.

% \bibitem{Coq}
% Y.~Bertot and P.~Casteran, {\em {Interactive Theorem Proving and Program
%   Development}}, Springer, 2004.

% \bibitem{Nitpick}
% J.C. Blanchette and T.~Nipkow, `Nitpick: A counterexample generator for
%   higher-order logic based on a relational model finder', in {\em Proc. of ITP
%   2010}, number 6172 in LNCS, pp. 131--146. Springer, (2010).

% \bibitem{Satallax}
% C.E. Brown, `Satallax: An automated higher-order prover', in {\em Proc. of
%   IJCAR 2012}, number 7364 in LNAI, pp. 111 -- 117. Springer, (2012).

% \bibitem{ContemporaryBibliography}
% R.~Corazzon.
% \newblock Contemporary~bibliography~on~ontological~arguments: {\scriptsize
%   \url{http://www.ontology.co/biblio/ontological-proof-contemporary-biblio.htm}}.

% \bibitem{Fitting}
% M.~Fitting, {\em Types, Tableaux and G\"odel's God}, Kluwer, 2002.

% \bibitem{fitting98}
% M.~Fitting and R.L. Mendelsohn, {\em First-Order Modal Logic}, volume 277 of
%   {\em Synthese Library}, Kluwer, 1998.

% \bibitem{Gallin75}
% D.~Gallin, {\em Intensional and Higher-Order Modal Logic}, North-Holland, 1975.

% \bibitem{garbacz12:_prover_simpl_expal_away}
% P.~Garbacz, `{PROVER9's} simplifications explained away', {\em Australasian
%   Journal of Philosophy}, {\bf 90}(3),  585--592, (2012).

% \bibitem{GoedelNotes}
% K.~G\"odel, {\em Appx.A: Notes in Kurt G\"odel's Hand},  144--145.
% \newblock In  \cite{sobel2004logic}, 2004.

% \bibitem{Henkin50}
% L.~Henkin, `Completeness in the theory of types', {\em Journal of Symbolic
%   Logic}, {\bf 15}(2),  81--91, (1950).

% \bibitem{homl}
% R.~Muskens, `{Higher Order Modal Logic}', in {\em Handbook of Modal Logic},
%   ed., P~Blackburn~et al.,  621--653, Elsevier, Dordrecht, (2006).

% \bibitem{Isabelle}
% T.~Nipkow, L.C. Paulson, and M.~Wenzel, {\em {Isabelle/HOL: A Proof Assistant
%   for Higher-Order Logic}}, number 2283 in LNCS, Springer, 2002.

% \bibitem{oppenheimera11}
% P.E. Oppenheimer and E.N. Zalta, `A computationally-discovered simplification
%   of the ontological argument', {\em Australasian Journal of Philosophy}, {\bf
%   89}(2),  333--349, (2011).

% \bibitem{rushby13}
% J.~Rushby, `The ontological argument in {PVS}', in {\em Proc.~of CAV Workshop
%   ``Fun With Formal Methods''}, St. Petersburg, Russia,, (2013).

% \bibitem{Schulz:AICOM-2002}
% S.~Schulz, `E -- a brainiac theorem prover', {\em {AI Communications}}, {\bf
%   15}(2),  111--126, (2002).

% \bibitem{ScottNotes}
% D.~Scott, {\em Appx.B: Notes in Dana Scott's Hand},  145--146.
% \newblock In  \cite{sobel2004logic}, 2004.

% \bibitem{sobel2004logic}
% J.H. Sobel, {\em Logic and Theism: Arguments for and Against Beliefs in God},
%   Cambridge U. Press, 2004.

% \bibitem{sutcliffe2009tptp}
% G.~Sutcliffe, `The {TPTP} problem library and associated infrastructure', {\em
%   Journal of Automated Reasoning}, {\bf 43}(4),  337--362, (2009).

% \bibitem{J22}
% G.~Sutcliffe and C.~Benzm{\"u}ller, `Automated reasoning in higher-order logic
%   using the {TPTP THF} infrastructure.', {\em Journal of Formalized Reasoning},
%   {\bf 3}(1),  1--27, (2010).

% \bibitem{J29}
% B.~Woltzenlogel-Paleo and C.~Benzm{\"u}ller, `Automated verification and
%   reconstruction of {G\"odel's} proof of {God's} existence', {\em OCG J.},
%   (2013).

%\end{thebibliography}


% ------------------------------------------------------------------------
\end{document}
% ------------------------------------------------------------------------
