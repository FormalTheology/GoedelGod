% ------------------------------------------------------------------------
% bjourdoc.tex for birkjour.cls*******************************************
% ------------------------------------------------------------------------
%%%%%%%%%%%%%%%%%%%%%%%%%%%%%%%%%%%%%%%%%%%%%%%%%%%%%%%%%%%%%%%%%%%%%%%%%%

\documentclass{birkjour}
%
%
% THEOREM Environments (Examples)-----------------------------------------
%
 \newtheorem{thm}{Theorem}[section]
 \newtheorem{cor}[thm]{Corollary}
 \newtheorem{lem}[thm]{Lemma}
 \newtheorem{prop}[thm]{Proposition}
 \theoremstyle{definition}
 \newtheorem{defn}[thm]{Definition}
 \theoremstyle{remark}
 \newtheorem{rem}[thm]{Remark}
 \newtheorem*{ex}{Example}
 \numberwithin{equation}{section}


\begin{document}

%-------------------------------------------------------------------------
% editorial commands: to be inserted by the editorial office
%
%\firstpage{1} \volume{228} \Copyrightyear{2004} \DOI{003-0001}
%
%
%\seriesextra{Just an add-on}
%\seriesextraline{This is the Concrete Title of this Book\br H.E. R and S.T.C. W, Eds.}
%
% for journals:
%
%\firstpage{1}
%\issuenumber{1}
%\Volumeandyear{1 (2004)}
%\Copyrightyear{2004}
%\DOI{003-xxxx-y}
%\Signet
%\commby{inhouse}
%\submitted{March 14, 2003}
%\received{March 16, 2000}
%\revised{June 1, 2000}
%\accepted{July 22, 2000}
%
%
%
%---------------------------------------------------------------------------
%Insert here the title, affiliations and abstract:
%


\title[Modal Collapse]
 {The Ontological Modal Colapse \\ 
 as a Collapse of the Square of Opposition}



%----------Author 1
\author[Benzm\"uller]{Christoph Benzm\"uller}

\address{%
Viaduktstr. 42\\
P.O. Box 133\\
CH 4010 Basel\\
Switzerland}

\email{c.benzmueller@gmail.com}

\thanks{This work was completed with the support of ToDo}


%----------Author 2
\author[Woltzenlogel-Paleo]{Bruno Woltzenlogel Paleo}
\address{ }
\email{bruno.wp@gmail.com}
%----------classification, keywords, date
\subjclass{
Primary 03A02;  % Philosophical aspects of logic and foundations
Secondary 68T02 % Artificial Intelligence
}

\keywords{Modal Logics, Higher-Order Modal Logics}

\date{August 21, 2014}
%----------additions
\dedicatory{ }
%%% ----------------------------------------------------------------------

\begin{abstract}
This work presents.
\end{abstract}

%%% ----------------------------------------------------------------------
\maketitle
%%% ----------------------------------------------------------------------
%\tableofcontents
\section{Introduction}


\begin{defn}
This serves as environment for definitions. Note that the text
appears not in italics.
\end{defn}

\begin{equation}\label{testequation}
\text{This is a sample equation: } c^2=a^2+b^2
\end{equation}

\begin{thm}[Main Theorem]
In contrast to definitions, theorems appear typeset in italics as
it has become more or less standard in most textbooks and
monographs. Equations can be cited using the \verb+\eqref+ command which
automatically adds brackets: \verb+\eqref{testequation}+ results in \eqref{testequation}.
\end{thm}

\begin{proof}
A special environment is predefined: the \textit{proof} environment. Please use
\begin{verbatim}\begin{proof}\end{verbatim}
proof of the statement
\begin{verbatim}\end{proof}\end{verbatim}
for typesetting your proofs. The end-of-proof symbol $\Box$ will be added automatically.
\end{proof}

There are two known problems with the placement of the end-of-proof sign:

\begin{enumerate}
  \item if your proof ends with a\ \ s i n g l e\ \ displayed line, the end-of-proof sign would
be placed in the line below; if you want to avoid this, write your line in the form
\begin{verbatim}$$displayed math line \eqno\qedhere$$\end{verbatim}
which results in

\begin{proof}
$$displayed math line \eqno\qedhere$$
\end{proof}
\item if your proof ends with an aligned displayed environment, the command
\verb+\tag*{\qed}+ can be used to place the end-of-proof sign properly:
\begin{verbatim}
\begin{align*}
\alpha&=\beta+\gamma\\
&=\delta+\epsilon\tag*{\qed}
\end{align*}
\end{verbatim}
results in
\begin{align*}
\alpha&=\beta+\gamma\\
&=\delta+\epsilon\tag*{\qed}
\end{align*}
\end{enumerate}
Please try to avoid using the obsolete \verb+\eqnarray+ environment. This environment has several bugs
and has been replaced by the more flexible \AmS\ environments \verb+align, split, multline+.


\begin{rem}
Additional comments are being typeset without boldfaced entrance
word as they may be minor important.
\end{rem}

\begin{ex}
For some constructs, even no number is required.
\end{ex}

Displayed equations may be numbered like the following one:
\begin{equation}
\sqrt{1-\sin^2(x)}=|\cos(x)|.
\end{equation}


\subsection{Conclusions}
% ------------------------------------------------------------------------

% \subsection*{Acknowledgment}
% Many thanks to ...


\begin{thebibliography}{1}
\bibitem{test} A. B. C. Test, \textit{On a Test.} J. of Testing
\textbf{88} (2000), 100--120.
\bibitem{latex} G. Gr\"atzer, \textit{Math into \LaTeX.} 3rd Edition,
Birkh\"auser, 2000.
\end{thebibliography}

% ------------------------------------------------------------------------
\end{document}
% ------------------------------------------------------------------------
