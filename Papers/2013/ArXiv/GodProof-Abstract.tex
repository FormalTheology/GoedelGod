\documentclass{llncs}

\usepackage{url,amsmath,amssymb,a4wide}


\newcommand{\imp}{\rightarrow}
\newcommand{\biimp}{\leftrightarrow}
\newcommand{\all}{\forall}
\newcommand{\ex}{\exists}
\newcommand{\seq}{\vdash}
\newcommand{\nec}{\Box} % necessarily
\newcommand{\pos}{\Diamond} % possibly
\newcommand{\ess}[2]{#1 \ \mathit{ess.} \ #2}
\newcommand{\NE}{\mathit{NE}}


% \usepackage[
%     firstinits=true,
%      backend=bibtex, 
%      natbib=true,
%      isbn=false,
%      doi=true,
%      url=false,
%     style=numeric
% ]{biblatex}
% \addbibresource[label=bibs]{Bibliography.bib}
% \addbibresource[label=bibs]{chris.bib}

\title{Formalization, Mechanization and Automation % of Variants
  of \\ G\"{o}del's Proof of God's Existence\thanks{This work has been
    supported by the German Research Foundation under grant
    BE2501/9-1.}}

\author{
  Christoph Benzm\"{u}ller\inst{1} 
  \and 
  Bruno Woltzenlogel Paleo\inst{2}
}

\authorrunning{C.\~Benzm\"{u}ller \and B.\~Woltzenlogel Paleo}


\institute{
  Dahlem Center for Intelligent Systems, Freie Universit\"{a}t Berlin, Germany\\
  \email{c.benzmueller@gmail.com}
  \and 
  Theory and Logic Group, Vienna University of Technology, Austria \\
  \email{bruno@logic.at}
}

\begin{document}

\maketitle

\bigskip

\noindent
\textbf{Update (31/08/2017):} The abstract below, uploaded to arXiv on
21/08/2013, was the first communication of the computer-assisted
formalization of G\"odel's ontological proof. Since then, the
following longer papers have been published:
%\cite{B39,B40,B45,B47,B49,B52,B56,B57,B60,B61,B62,B63,B64,B65,B66}.
\cite{W50,J28,B15,C39,C40,C41,C42,C44,C46,C52,W55,J32,C55,C60,B16,B66,J35,J36,C65}.

\bigskip
\bigskip

Attempts to prove the existence (or non-existence) of God by means of
abstract ontological arguments are an old tradition in philosophy and
theology.  G\"{o}del's proof \cite{Goedel1970,GoedelNotes} is a modern culmination of
this tradition, following particularly the footsteps of Leibniz.
%
G\"{o}del defines God as a being who possesses all \emph{positive} properties.
He does not extensively discuss what positive properties are, 
but instead he states a few reasonable (but debatable) axioms that they should satisfy.
Various slightly different versions of axioms and definitions have
been considered by G\"{o}del and by several philosophers who commented
on his proof
(cf. \cite{sobel2004logic,AndersonGettings,Fitting,Adams,ContemporaryBibliography}). 

Dana Scott's version of G\"odel's proof \cite{ScottNotes} employs the
following axioms (\textbf{A}), definitions (\textbf{D}), corollaries
(\textbf{C}) and theorems (\textbf{T}), and it proceeds in the
following order:\footnote{ A1, A2, A5, D1, D3 are logically
  equivalent to, respectively, axioms 2, 5 and 4 and definitions 1 and
  3 in G\"odel's notes \cite{Goedel1970,GoedelNotes}. 
  A3 was introduced by Scott \cite{ScottNotes} and 
  could be derived from G\"odel's axiom 1 and
  D1 in a logic with infinitary conjunction. 
  A4 is a weaker form of G\"odel's axiom 3. 
  D2 has an extra conjunct $\phi(x)$ lacking in G\"odel's definition 2; 
  this is believed to have been 
  an oversight by G\"odel \cite{Hazen}.}

\allowdisplaybreaks[1] 
\begin{description}
\item[A1] Either a property or its negation is positive, but not
  both:  \hfill 
  $\all \phi [P(\neg \phi) \biimp \neg P(\phi)]$
\item[A2] A property necessarily implied \\ by a
  positive property is positive: \phantom{b} \hfill 
  $\all \phi \all \psi [(P(\phi) \wedge \nec \all x [\phi(x)
  \imp \psi(x)]) \imp P(\psi)]$ 
\item[T1] Positive properties are possibly exemplified: \hfill $\all \varphi [P(\varphi) \imp \pos \ex x \varphi(x)]$
\item[D1] A \emph{God-like} being possesses all positive properties: \hfill
  $G(x) \biimp \forall \phi [P(\phi) \to \phi(x)]$
\item[A3]  The property of being God-like is positive: \hfill   $P(G)$
\item[C\phantom{1}] Possibly, God exists: \hfill $\pos \ex x G(x)$
\item[A4]  Positive properties are necessarily positive: \hfill 
  $\all \phi [P(\phi) \to \Box \; P(\phi)]$
\item[D2] An \emph{essence} of an individual is  \\ a property possessed by it and \\ necessarily implying any of its properties:
  \phantom{b} \hfill $\ess{\phi}{x} \biimp \phi(x) \wedge \all
  \psi (\psi(x) \imp \nec \all y (\phi(y) \imp \psi(y)))$
\item[T2]  Being God-like is an essence of any
  God-like being: \hfill $\all x [G(x) \imp \ess{G}{x}]$ 
\item[D3] \emph{Necessary existence} of an individual is \\ the necessary exemplification of all its essences: 
  \phantom{b} \hfill $\NE(x) \biimp \all \phi [\ess{\phi}{x} \imp \nec \ex y \phi(y)]$
\item[A5] Necessary existence is a positive property: \hfill $P(\NE)$
\item[T3] Necessarily, God exists: \hfill $\nec \ex x G(x)$ 
\end{description}

\noindent
Scott's version of G\"{o}del's proof has now been 
analysed for the first-time
with an unprecedent degree of detail 
and formality with the help of theorem provers; cf.~\cite{FormalTheologyRepository,ComputationalPhilosophyRepository}. 
The following has been done (and in this order):
\begin{itemize}
\item A detailed natural deduction proof.
%
\item A formalization of the axioms, definitions and theorems in the TPTP THF syntax \cite{J22}.
%
\item Automatic verification of the consistency of the axioms and 
definitions with Nitpick \cite{Nitpick}.
%
\item Automatic demonstration of the theorems with the provers LEO-II \cite{LEO-II} and Satallax \cite{Satallax}.

\item A step-by-step formalization using the Coq proof assistant \cite{Coq}.

\item A formalization using the Isabelle proof assistant \cite{Isabelle}, where the theorems (and some additional lemmata) have been automated with Sledgehammer \cite{Sledgehammer} and Metis \cite{Hurd03first-orderproof}.
\end{itemize}

G\"{o}del's proof is challenging to formalize and verify because it
requires an expressive logical language with modal operators
(\emph{possibly} and \emph{necessarily}) and with
quantifiers for individuals and properties.  Our computer-assisted formalizations rely on an
embedding of the modal logic into classical higher-order logic with
Henkin semantics \cite{J23,B9}. The formalization is thus essentially
done in classical higher-order logic where  quantified modal logic is
emulated.

In our ongoing computer-assisted study of G\"odel's proof we have
obtained the following results:
\begin{itemize}
\item The basic modal logic K is sufficient for proving T1, C and T2. 
\item Modal logic S5 is not needed for proving T3; the logic KB is
  sufficient. 
\item Without the first conjunct $\phi(x)$ in D2 the set of axioms 
  and definitions would be inconsistent.
\item For proving theorem T1, only the left to right direction of
  axiom A1 is needed. However, the backward direction of A1 is
  required for proving T2.

\end{itemize}

This work attests the maturity of contemporary interactive and
automated deduction tools for classical higher-order logic and
demonstrates the elegance and practical relevance of the
embeddings-based approach.  Most importantly, our work opens new
perspectives for a computer-assisted theoretical philosophy.  The
critical discussion of the underlying concepts, definitions and axioms
remains a human responsibility, but the computer can assist in
building and checking rigorously correct logical arguments. In case of
logico-philosophical disputes, the computer can check the disputing
arguments and partially fulfill Leibniz' dictum: Calculemus --- Let us
calculate!

\footnotesize

\begin{thebibliography}{10}

\bibitem{Adams}
R.M. Adams.
\newblock Introductory note to *1970.
\newblock In {\em {Kurt G\"odel: Collected Works Vol. 3: Unpublished Essays and
  Letters}}. Oxford University Press, 1995.

\bibitem{AndersonGettings}
A.C. Anderson and M.~Gettings.
\newblock G\"odel ontological proof revisited.
\newblock In {\em {G\"odel'96: Logical Foundations of Mathematics, Computer
  Science, and Physics: Lecture Notes in Logic 6}}, pages 167--172. {Springer},
  1996.

\bibitem{B16}
Matthias Bentert, Christoph Benzm{\"u}ller, David Streit, and Bruno
  Woltzenlogel~Paleo.
\newblock Analysis of an ontological proof proposed by {Leibniz}.
\newblock In Charles Tandy, editor, {\em Death and Anti-Death, Volume 14: Four
  Decades after Michael Polanyi, Three Centuries after G.W. Leibniz}. Ria
  University Press, 2016.

\bibitem{B9}
C.~Benzm{\"u}ller and L.C. Paulson.
\newblock Exploring properties of normal multimodal logics in simple type
  theory with {LEO-II}.
\newblock In {\em {Festschrift in Honor of {Peter B. Andrews} on His 70th
  Birthday}}, pages 386--406. College Publications, 2008.

\bibitem{J23}
C.~Benzm{\"u}ller and L.C. Paulson.
\newblock Quantified multimodal logics in simple type theory.
\newblock {\em Logica Universalis (Special Issue on Multimodal Logics)},
  7(1):7--20, 2013.

\bibitem{LEO-II}
C.~Benzm{\"u}ller, F.~Theiss, L.~Paulson, and A.~Fietzke.
\newblock {LEO-II} - a cooperative automatic theorem prover for higher-order
  logic.
\newblock In {\em Proc. of IJCAR 2008}, volume 5195 of {\em LNAI}, pages
  162--170. Springer, 2008.

\bibitem{C42}
Christoph Benzm{\"u}ller.
\newblock {G\"{o}del's} ontological argument revisited -- findings from a
  computer-supported analysis (invited).
\newblock In Ricardo~Souza Silvestre and Jean-Yves B\'eziau, editors, {\em
  Handbook of the 1st World Congress on Logic and Religion, Jo\~ao Pessoa,
  Brazil}, page~13, 2015.
\newblock (Invited abstract).

\bibitem{C41}
Christoph Benzm{\"u}ller, Leon Weber, and Bruno Woltzenlogel~Paleo.
\newblock Computer-assisted analysis of the {Anderson-H\'{a}jek} ontological
  controversy.
\newblock In Ricardo~Souza Silvestre and Jean-Yves B\'eziau, editors, {\em
  Handbook of the 1st World Congress on Logic and Religion, Joao Pessoa,
  Brasil}, pages 53--54, 2015.
\newblock (superseded by 2016 article in Logica Universalis).

\bibitem{J32}
Christoph Benzm{\"u}ller, Leon Weber, and Bruno Woltzenlogel~Paleo.
\newblock Computer-assisted analysis of the {Anderson-H\'{a}jek} controversy.
\newblock {\em Logica Universalis}, 11(1):139--151, 2017.

\bibitem{J28}
Christoph Benzm\"uller and Bruno Woltzenlogel~Paleo.
\newblock {G{\"o}del's God in Isabelle/HOL}.
\newblock {\em Archive of Formal Proofs}, 2013.
\newblock (Formally verified).

\bibitem{W50}
Christoph Benzm\"uller and Bruno Woltzenlogel~Paleo.
\newblock {G\"odel's} {God} on the computer.
\newblock In S.~Schulz, G.~Sutcliffe, and B.~Konev, editors, {\em Proceedings
  of the 10th International Workshop on the Implementation of Logics}, 2013.
\newblock (Invited paper).

\bibitem{C40}
Christoph Benzm{\"u}ller and Bruno Woltzenlogel~Paleo.
\newblock Automating {G\"{o}del's} ontological proof of {God}'s existence with
  higher-order automated theorem provers.
\newblock In Torsten Schaub, Gerhard Friedrich, and Barry O'Sullivan, editors,
  {\em ECAI 2014}, volume 263 of {\em Frontiers in Artificial Intelligence and
  Applications}, pages 93 -- 98. IOS Press, 2014.

\bibitem{C39}
Christoph Benzm{\"u}ller and Bruno Woltzenlogel~Paleo.
\newblock {G\"odel's} proof of {God's} existence.
\newblock In Jean-Yves Beziau and Katarzyna Gan-Krzywoszynska, editors, {\em
  Handbook of the World Congress on the Square of Opposition IV}, pages 22--23,
  2014.
\newblock (superseded by ECAI-2014 paper).

\bibitem{C52}
Christoph Benzm{\"u}ller and Bruno Woltzenlogel~Paleo.
\newblock Experiments in computational metaphysics: {G\"odel's} proof of god's
  existence.
\newblock In Subhash~C. Mishram, Ramgopal Uppaluri, and Varun Agarwal, editors,
  {\em Science \& Spiritual Quest, Proceedings of the 9th All India Students'
  Conference, 30th October -- 1 November, 2015, IIT Kharagpur, India}, pages
  23--40. Bhaktivedanta Institute, Kolkata, \url{www.binstitute.org}, 2015.
\newblock (Invited paper, superseded by article in Savijnanam).

\bibitem{C46}
Christoph Benzm{\"u}ller and Bruno Woltzenlogel~Paleo.
\newblock Higher-order modal logics: Automation and applications.
\newblock In Adrian Paschke and Wolfgang Faber, editors, {\em Reasoning Web.
  Web Logic Rules - 11th International Summer School 2015, Berlin, Germany,
  July 31 - August 4, 2015, Tutorial Lectures}, number 9203 in LNCS, pages
  32--74, Berlin, Germany, 2015. Springer.

\bibitem{C44}
Christoph Benzm{\"{u}}ller and Bruno Woltzenlogel~Paleo.
\newblock Interacting with modal logics in the {Coq} proof assistant.
\newblock In Lev~D. Beklemishev and Daniil~V. Musatov, editors, {\em Computer
  Science - Theory and Applications - 10th International Computer Science
  Symposium in Russia, {CSR} 2015, Listvyanka, Russia, July 13-17, 2015,
  Proceedings}, volume 9139 of {\em LNCS}, pages 398--411. Springer, 2015.

\bibitem{W55}
Christoph Benzm{\"u}ller and Bruno Woltzenlogel~Paleo.
\newblock On logic embeddings and {G\"odel's} {God}.
\newblock In Mihai Codescu, Razvan Diaconescu, and Ionut Tutu, editors, {\em
  Recent Trends in Algebraic Development Techniques: 22nd International
  Workshop, WADT 2014, Sinaia, Romania, September 4-7, 2014, Revised Selected
  Papers}, number 9563 in LNCS, pages 3--6, Sinaia, Romania, 2015. Springer.
\newblock (Invited paper).

\bibitem{C55}
Christoph Benzm{\"u}ller and Bruno Woltzenlogel~Paleo.
\newblock The inconsistency in {G{\"o}del's} ontological argument: A success
  story for {AI} in metaphysics.
\newblock In Subbarao Kambhampati, editor, {\em IJCAI 2016}, volume 1-3, pages
  936--942. AAAI Press, 2016.

\bibitem{B15}
Christoph Benzm{\"u}ller and Bruno Woltzenlogel~Paleo.
\newblock The modal collapse as a collapse of the modal square of opposition.
\newblock In Jean-Yves B\'{e}ziau and Gianfranco Basti, editors, {\em The
  Square of Opposition: A Cornerstone of Thought (Collection of papers related
  to the World Congress on the Square of Opposition IV, Vatican, 2014),
  \url{http://www.springer.com/us/book/9783319450612}}, Studies in Universal
  Logic. Springer International Publishing Switzerland, 2016.

\bibitem{C60}
Christoph Benzm{\"u}ller and Bruno Woltzenlogel~Paleo.
\newblock An object-logic explanation for the inconsistency in {G\"odel's}
  ontological theory (extended abstract, sister conferences).
\newblock In Malte Helmert and Franz Wotawa, editors, {\em KI 2016: Advances in
  Artificial Intelligence, Proceedings}, LNCS, pages 244--250, Berlin, Germany,
  2016. Springer.

\bibitem{J36}
Christoph Benzm{\"u}ller and Bruno Woltzenlogel~Paleo.
\newblock Experiments in {Computational Metaphysics}: {G\"odel's} proof of
  {God's} existence.
\newblock {\em Savijnanam: scientific exploration for a spiritual paradigm.
  Journal of the Bhaktivedanta Institute}, 9:43--57, 2017.

\bibitem{Coq}
Y.~Bertot and P.~Casteran.
\newblock {\em {Interactive Theorem Proving and Program Development}}.
\newblock Springer, 2004.

\bibitem{Sledgehammer}
J.C. Blanchette, S.~B\"ohme, and L.C. Paulson.
\newblock Extending {Sledgehammer} with {SMT} solvers.
\newblock {\em Journal of Automated Reasoning}, 51(1):109--128, 2013.

\bibitem{Nitpick}
J.C. Blanchette and T.~Nipkow.
\newblock Nitpick: A counterexample generator for higher-order logic based on a
  relational model finder.
\newblock In {\em Proc. of ITP 2010}, number 6172 in LNCS, pages 131--146.
  Springer, 2010.

\bibitem{Satallax}
C.E. Brown.
\newblock Satallax: An automated higher-order prover.
\newblock In {\em Proc. of IJCAR 2012}, number 7364 in LNAI, pages 111 -- 117.
  Springer, 2012.

\bibitem{ContemporaryBibliography}
R.~Corazzon.
\newblock Contemporary bibligraphy on the ontological proof
  (\url{http://www.ontology.co/biblio/ontological-proof-contemporary-biblio.htm}).

\bibitem{Fitting}
M.~Fitting.
\newblock {\em {Types, Tableaux and G\"odel's God}}.
\newblock Kluver Academic Press, 2002.

\bibitem{C65}
David Fuenmayor and Christoph Benzm{\"u}ller.
\newblock Automating emendations of the ontological argument in intensional
  higher-order modal logic.
\newblock In {\em KI 2017: Advances in Artificial Intelligence 40th Annual
  German Conference on AI}, LNAI. Springer, 2017.

\bibitem{J35}
David Fuenmayor and Christoph Benzm\"uller.
\newblock {Types, Tableaus and G{\"o}del's God in Isabelle/HOL}.
\newblock {\em Archive of Formal Proofs}, 2017.
\newblock Formally verified with Isabelle/HOL.

\bibitem{Goedel1970}
K.~G\"odel.
\newblock Ontological proof.
\newblock In {\em {Kurt G\"odel: Collected Works Vol. 3: Unpublished Essays and
  Letters}}. Oxford University Press, 1970.

\bibitem{GoedelNotes}
K.~G\"odel.
\newblock {\em Appendix A. Notes in Kurt G\"odel's Hand}, pages 144--145.
\newblock In  \cite{sobel2004logic}, 2004.

\bibitem{Hazen}
A.P. Hazen.
\newblock On g\"odel's ontological proof.
\newblock {\em Australasian Journal of Philosophy}, 76:361--377, 1998.

\bibitem{Hurd03first-orderproof}
J.~Hurd.
\newblock First-order proof tactics in higher-order logic theorem provers.
\newblock In {\em Design and Application of Strategies/Tactics in Higher Order
  Logics, NASA Tech. Rep. NASA/CP-2003-212448}, pages 56--68, 2003.

\bibitem{Isabelle}
T.~Nipkow, L.C. Paulson, and M.~Wenzel.
\newblock {\em {Isabelle/HOL: A Proof Assistant for Higher-Order Logic}}.
\newblock Number 2283 in LNCS. Springer, 2002.

\bibitem{ScottNotes}
D.~Scott.
\newblock {\em Appendix B. Notes in Dana Scott's Hand}, pages 145--146.
\newblock In  \cite{sobel2004logic}, 2004.

\bibitem{sobel2004logic}
J.H. Sobel.
\newblock {\em Logic and Theism: Arguments for and Against Beliefs in God}.
\newblock Cambridge U. Press, 2004.

\bibitem{J22}
G.~Sutcliffe and C.~Benzm{\"u}ller.
\newblock Automated reasoning in higher-order logic using the {TPTP THF}
  infrastructure.
\newblock {\em Journal of Formalized Reasoning}, 3(1):1--27, 2010.

\bibitem{B66}
B.~Woltzenlogel~Paleo.
\newblock Leibniz's characteristica universalis and calculus ratiocinator
  today.
\newblock In Charles Tandy, editor, {\em Death and Anti-Death, Volume 14: Four
  Decades after Michael Polanyi, Three Centuries after G. W. Leibniz}. Ria
  University Press, 2016.

\bibitem{ComputationalPhilosophyRepository}
B.~Woltzenlogel~Paleo and C.~Benzm\"uller.
\newblock Computational philosophy repository
  (\url{https://gitlab.com/aossie/ComputationalPhilosophy/}).

\bibitem{FormalTheologyRepository}
B.~Woltzenlogel~Paleo and C.~Benzm\"uller.
\newblock Formal theology repository
  (\url{https://github.com/FormalTheology/GoedelGod}).

\end{thebibliography}



%\bibliographystyle{plain}
%\bibliography{Bibliography,chris}

%\printbibliography

%\footnotesize
%\bibliographystyle{plain}
%\bibliography{Bibliography,chris}
% \begin{thebibliography}{10}

% \bibitem{Adams}
% R.M. Adams.
% \newblock Introductory note to *1970.
% \newblock In {\em {Kurt G\"odel: Collected Works Vol. 3: Unpublished Essays and
%   Letters}}. Oxford University Press, 1995.

% \bibitem{AndersonGettings}
% A.C. Anderson and M.~Gettings.
% \newblock G\"odel ontological proof revisited.
% \newblock In {\em {G\"odel'96: Logical Foundations of Mathematics, Computer
%   Science, and Physics: Lecture Notes in Logic 6}}. {Springer}, 1996.

% \bibitem{B9}
% C.~Benzm{\"u}ller and L.C. Paulson.
% \newblock Exploring properties of normal multimodal logics in simple type
%   theory with {LEO-II}.
% \newblock In {\em {Festschrift in Honor of {Peter B. Andrews} on His 70th
%   Birthday}}, pages 386--406. College Publications.

% \bibitem{J23}
% C.~Benzm{\"u}ller and L.C. Paulson.
% \newblock Quantified multimodal logics in simple type theory.
% \newblock {\em Logica Universalis (Special Issue on Multimodal Logics)},
%   7(1):7--20, 2013.

% \bibitem{LEO-II}
% C.~Benzm{\"u}ller, F.~Theiss, L.~Paulson, and A.~Fietzke.
% \newblock {LEO-II} - a cooperative automatic theorem prover for higher-order
%   logic.
% \newblock In {\em Proc. of IJCAR 2008}, volume 5195 of {\em LNAI}, pages
%   162--170. Springer, 2008.

% \bibitem{Coq}
% Y.~Bertot and P.~Casteran.
% \newblock {\em {Interactive Theorem Proving and Program Development}}.
% \newblock Springer, 2004.

% \bibitem{Sledgehammer}
% J.C. Blanchette, S.~B\"ohme, and L.C. Paulson.
% \newblock Extending {Sledgehammer} with {SMT} solvers.
% \newblock {\em Journal of Automated Reasoning}, 51(1):109--128, 2013.

% \bibitem{Nitpick}
% J.C. Blanchette and T.~Nipkow.
% \newblock Nitpick: A counterexample generator for higher-order logic based on a
%   relational model finder.
% \newblock In {\em Proc. of ITP 2010}, number 6172 in LNCS, pages 131--146.
%   Springer, 2010.

% \bibitem{Satallax}
% C.E. Brown.
% \newblock Satallax: An automated higher-order prover.
% \newblock In {\em Proc. of IJCAR 2012}, number 7364 in LNAI, pages 111 -- 117.
%   Springer, 2012.

% \bibitem{ContemporaryBibliography}
% R.~Corazzon.
% \newblock Contemporary bibligraphy on the ontological proof
%   (\url{http://www.ontology.co/biblio/ontological-proof-contemporary-biblio.htm}).

% \bibitem{Fitting}
% M.~Fitting.
% \newblock {\em {Types, Tableaux and G\"odel's God}}.
% \newblock Kluver Academic Press, 2002.

% \bibitem{Goedel1970}
% K.~G\"odel.
% \newblock Ontological proof.
% \newblock In {\em {Kurt G\"odel: Collected Works Vol. 3: Unpublished Essays and
%   Letters}}. Oxford University Press, 1970.

% \bibitem{GoedelNotes}
% K.~G\"odel.
% \newblock {\em Appendix A. Notes in Kurt G\"odel's Hand}, pages 144--145.
% \newblock In  \cite{sobel2004logic}, 2004.

% \bibitem{Hazen}
% A.P. Hazen.
% \newblock On g\"odel's ontological proof.
% \newblock {\em Australasian Journal of Philosophy}, 76:361--377, 1998.

% \bibitem{Hurd03first-orderproof}
% J.~Hurd.
% \newblock First-order proof tactics in higher-order logic theorem provers.
% \newblock In {\em Design and Application of Strategies/Tactics in Higher Order
%   Logics, NASA Tech. Rep. NASA/CP-2003-212448}, 2003.

% \bibitem{Isabelle}
% T.~Nipkow, L.C. Paulson, and M.~Wenzel.
% \newblock {\em {Isabelle/HOL: A Proof Assistant for Higher-Order Logic}}.
% \newblock Number 2283 in LNCS. Springer, 2002.

% \bibitem{FormalTheologyRepository}
% B.~Woltzenlogel Paleo and C.~Benzm\"uller.
% \newblock Formal theology repository
%   (\url{https://github.com/FormalTheology/GoedelGod}).

% \bibitem{ScottNotes}
% D.~Scott.
% \newblock {\em Appendix B. Notes in Dana Scott's Hand}, pages 145--146.
% \newblock In  \cite{sobel2004logic}, 2004.

% \bibitem{sobel2004logic}
% J.H. Sobel.
% \newblock {\em Logic and Theism: Arguments for and Against Beliefs in God}.
% \newblock Cambridge U. Press, 2004.

% \bibitem{J22}
% G.~Sutcliffe and C.~Benzm{\"u}ller.
% \newblock Automated reasoning in higher-order logic using the {TPTP THF}
%   infrastructure.
% \newblock {\em Journal of Formalized Reasoning}, 3(1):1--27, 2010.

% \end{thebibliography}


\end{document}