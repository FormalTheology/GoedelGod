

\typeout{The Inconsistency in G\"odel's Ontological Argument: \\ A Success Story for AI in Metaphysics}


\documentclass{article}
\usepackage{ijcai16}

% Use the postscript times font!
\usepackage{times}

\pdfinfo{
/Title (The Inconsistency in G\"odel's Ontological Argument: A Success Story for AI in Metaphysics)
/Author (Christoph Benzm\"uller, Bruno Woltzenlogel Paleo) }

% the following package is optional:
\usepackage{latexsym} 

\usepackage{graphicx}
\usepackage{caption}
\usepackage{subcaption}

\usepackage{amsmath}
\usepackage{xspace}
\usepackage{graphicx,url}
\usepackage{txfonts} % needed for \Diamondblack
\usepackage{color}
\usepackage{xcolor}

%\usepackage{modallogics}
\newcommand{\logic}[1]{\textbf{#1}\xspace}
\newcommand{\KB}{\logic{KB}}
\newcommand{\KFour}{\logic{K4}}
\newcommand{\KFourB}{\logic{K4B}}
\newcommand{\K}{\logic{K}}
\newcommand{\KT}{\logic{KT}}
\newcommand{\SFour}{\logic{S4}}
\newcommand{\SFive}{\logic{S5}}
\newcommand{\SFiveU}{\logic{S5\textsuperscript{U}}}

\newcommand{\imp}{{\rightarrow}}
\newcommand{\biimp}{\leftrightarrow}
\newcommand{\allq}{\forall}
\newcommand{\exq}{\exists}
\newcommand{\seq}{\vdash}

\newcommand{\Dia}{\Diamond} % possibly
\newcommand{\BlackBox}{\blacksquare}
\newcommand{\BlackDia}{\Diamondblack}

\newcommand{\NE}{\mathit{NE}}
\newcommand{\ess}[2]{#1\ \mathit{ess}\ #2}
\newcommand{\nec}{\Box}
\newcommand{\pos}{\Dia}


\title{The Inconsistency in G\"odel's Ontological Argument: \\ A Success Story for AI in Metaphysics}
\author{
  \begin{minipage}{0.5\textwidth}\centering
    Christoph Benzm\"uller\thanks{This work was supported by the 
                                  German National Research Foundation (DFG) under
                                  grants BE 2501/9-2 and BE 2501/11-1.} \\
    \normalfont Freie Universit\"at Berlin \& Stanford University\\
    c.benzmueller@gmail.com
  \end{minipage}
  \begin{minipage}{0.5\textwidth}\centering
    Bruno Woltzenlogel Paleo \\
    \normalfont Australian National University\\
    bruno.wp@gmail.com
  \end{minipage}
}


\begin{document}

\maketitle

\begin{abstract}
  This paper discusses the discovery of the inconsistency in G\"odel's
  ontological argument as a success story for artificial
  intelligence. Despite the popularity of the argument since the
  appearance of G\"odel's manuscript in the early 1970's, the
  inconsistency of the axioms used in the argument remained unnoticed
  until 2013, when it was detected automatically by the higher-order
  theorem prover \textsc{Leo-II}.  Understanding and verifying the
  refutation generated by the prover turned out to be a time-consuming
  task. Its completion, as reported here, required the
  reconstruction of the refutation in the Isabelle proof assistant, and it
  also led to a novel and more efficient way of automating
  higher-order modal logic \textbf{S5} with a universal accessibility
  relation. Furthermore, the development of an improved syntactical
  hiding for the utilized logic embedding technique allows the refutation to be
  presented in a human-friendly way, suitable for non-experts in the
  technicalities of higher-order theorem proving. This brings us a
  step closer to wider adoption of logic-based artificial intelligence
  tools by philosophers.
\end{abstract}


\section{Introduction}\label{sec:introduction}
Without exaggeration Kurt G\"{o}del's ontological
argument for the existence of God \cite{GoedelNotes,ScottNotes} is
amongst the most discussed formal proofs in modern literature. A rich
body of publications -- including very recent ones -- present,
discuss, assess, criticize, modify and improve G\"{o}del's original
work (see e.g.~Sobel~\shortcite{sobel2004logic} and Oppy~\shortcite{sep-ontological-arguments} and the
references therein).  In philosophy lectures at universities the
argument is regularly presented as a masterpiece argument in
metaphysics. Since 2013, when Benzm\"uller and Woltzenlogel-Paleo~\shortcite{J30,C40} first
reported their successful initial computer-assisted
analysis of G\"odel's proof and Scott's variant,
their work has received a media repercussion on a global scale\footnote{A
  collection of news articles is available at {\scriptsize
    \url{https://github.com/FormalTheology/GoedelGod/blob/master/Press/LinksToNews.md}}},
and numerous bloggers commented on the proof
\cite{fuhrmann15:_blogg_goedel}.

The in-depth analysis presented here substantially
extends previous computer-assisted studies of G\"odel's ontological
argument. Similarly to the related work \cite{J30,C40} the analysis has
been conducted with automated theorem provers for classical
higher-order logic (HOL; cf.~\cite{andrewsSEP} and the references
therein), even though G\"odel's proof is actually formulated in
higher-order \emph{modal} logic (HOML; cf.~\cite{homl} and the
references therein). To bridge between the two logics we utilise and
further improve the logic embedding approach \cite{J23,C40}, which has
already been employed successfully in preceding related work.

The main novel contribution reported in this paper is a detailed
analysis (in various modal logics) of the inconsistency of G\"{o}del's
original version of the axioms used in his manuscript
\shortcite{GoedelNotes}. The extraction, reconstruction and
verification of an informal, human intuitive argument has been an open
problem since the first detection of this inconsistency by
Benzm\"uller and Woltzenlogel-Paleo \shortcite{C40} with the
\textsc{Leo-II} prover.  The verified refutation (discussed in
\S\ref{sec:inconsistency}) displays a surprisingly accessible
explanation of the inconsistency, which is philosophically profound
and never presented in the literature. The detection of this
inconsistency in combination with the work reported here thus
demonstrates that artificial intelligence systems -- particularly
higher-order automated theorem provers -- are capable of assisting in
the discovery and elucidation of \emph{new} and philosophically
relevant knowledge.

On the technical side, the quest for constructing a compelling
refutation, capable of convincing also human non-experts, led to an
improvement of the syntax of the embedding of modal logics in Isabelle/HOL
(as discussed in \S\ref{sec:improvedsyntax}). With the new syntax, a
(nearly) perfect match between the original pen and paper
presentations and our encoding in Isabelle/HOL is feasible. A more
user-friendly syntax, as reported here, is clearly an important
prerequisite for promoting the theorem proving technology employed
here to a wider community of philosophers, who are not necessarily
experts in automated reasoning or HOL.

Another novel contribution reported here (in
\S\ref{sec:improvedembedding}) is the implementation of an alternative
embedding for the 
% slightly stronger 
more effective modal logic \SFiveU, which is
based on a universal accessibility relation. Our experiments have
shown that the new embedding is more efficient, as the following two
previously open problems can now be solved:
\begin{itemize}
\item Automatically proving the final theorem T3 (\textit{Necessarily, there
  exists God}), directly from Scott's \shortcite{ScottNotes} (consistent) axioms
  alone, without relying on the argument's intermediate argumentation
  steps (i.e., lemmata).
\item Automatically verifying, in Isabelle/HOL, the proof of the modal
  collapse \cite{Sobel}, which is one of the most strongly criticized
  logical consequences of the argument's axioms.
\end{itemize}


\subsection{Related Work}

First successful applications of theorem proving technology in
metaphysics were reported by Fitelson, Oppenheimer and
Zalta~\shortcite{FitelsonZalta,oppenheimer11}, who coined the term \textit{Computational Metaphysics} for this new research area and employed the first-order
\textsc{Prover9} \cite{prover9-mace4} in their experiments. Later on, Rushby~\shortcite{rushby13} used the proof assistant \textsc{PVS} \cite{cade92-pvs}. Common to both
works is a significant amount of proof-hand-coding work as well as their
focus on a non-modal formalization of St. Anselm's~\shortcite{Proslogion} simpler 
and older ontological argument. In contrast, the greater complexity of G\"odel's argument requires the formalization and automation of variants of \emph{higher-order} and \emph{modal} logics.


\section{A Brief History of the Argument}\label{sec:history}

St. Anselm's ontological argument \cite{Proslogion} can be regarded as
the ancestor of modern ontological arguments such as G\"odel's. In the
millenium between Anselm and G\"odel, many philosophers modified and
arguably improved Anselm's argument. Of particular importance to
G\"odel was the work of Leibniz~\cite{Adams}.  
Although G\"odel's notion of positive property is not exactly the same as Leibniz's notion of perfection, G\"odel's manuscript
(Fig.~\ref{GoedelScript}) can be considered a translation of Leibniz's
presentation of the argument into modern modal logic. G\"odel
discussed his manuscript with Scott, who shared a slightly different
version with a larger public. Scott's version of the axioms and
definitions, formalized in Isabelle, is shown in
Fig.~\ref{Scott_S5U}. The main difference to G\"odel's version is an
extra conjunct in the definition of \emph{essence} (\emph{ess}). G\"odel's
different definition of essence can be seen either in his manuscript
(Fig.~\ref{GoedelScript}) or, in more modern notation, in the Isabelle
formalization shown in Fig.~\ref{InconsistencyIsabelleK}. For Scott,
an essential property of an individual must be possessed by
him/her. For G\"odel, this is not required. 

G\"odel's omission has been
considered inessential and merely an oversight by many. For instance,
Hazen [\citeyear[p.365]{Hazen}] states that ``G\"odel left this
clause out [\ldots] but this appears to have been an oversight''.
% -- it is
%included in related manuscripts''. 
For more than four decades, its serious consequences remained unnoticed, 
despite numerous analysis and criticisms of the
argument. Especially since the discovery by Sobel
\shortcite{Sobel} that modal collapse (MC)\footnote{The modal collapse,
  $\phi\rightarrow \Box \phi$, states that contingent truth implies
  necessary truth; it can be interpreted as \textit{everything is
    pre-determined} or even \textit{there is no free will}. } is
entailed by G\"odel's (or also Scott's) axioms, several variants have
been proposed
\cite{Anderson,AndersonGettings,Hajek1,Hajek2,Hajek3,Bjordal}
attempting to avoid the modal collapse. Many of these variants omit
the crucial conjunct in the definition of essence as well.\footnote{As
  these variants also change other axioms, on which the inconsistency
  of G\"odel's axioms depends, it is not necessarily the case that
  these variants are also inconsistent; they must be analyzed
  separately.} Opponents of the argument
(e.g. Oppy [\citeyear[p.226--227]{oppy96:_goedel_ontol_argum};
\citeyear[p.364]{oppy00:_respon_gettin};
\citeyear[p.1068]{oppy08:_higher_order_ontol_argum}]) have also proposed
parodies and other criticisms, referring to variants where the
conjunct is omitted.

% \begin{figure*}
%   \centering
%   \begin{subfigure}[t]{0.4\textwidth}
%     \includegraphics[width=\textwidth]{./Images/Scott_S5U.png}
%     \caption{Full Automation of T3 in \SFiveU; Consistency of Scott's
%       Axioms;  Automatic Verification of Modal Collapse} \label{Scott_S5U}
%   \end{subfigure}
%   ~ %add desired spacing between images, e. g. ~, \quad, \qquad, \hfill etc. 
%   % (or a blank line to force the subfigure onto a new line)
%   \begin{subfigure}[t]{0.4\textwidth}
%     \includegraphics[width=\textwidth]{./Images/QML_S5U.png}
%     \caption{Improved Embedding of \SFiveU} \label{QML_S5U}
%   \end{subfigure}
%   ~ %add desired spacing between images, e. g. ~, \quad, \qquad, \hfill etc. 
%   % (or a blank line to force the subfigure onto a new line)
%   \begin{subfigure}[t]{0.4\textwidth}
%     \includegraphics[width=\textwidth]{./Images/Inconsistency_S5U_direct.png}
%     \caption{Inconsistency  in \SFiveU} \label{Inconsistency_S5U} 
%   \end{subfigure}
%   \begin{subfigure}[t]{0.4\textwidth}
%     \centerline{\includegraphics[width=\textwidth]{./Images/InconsistencyIsabelleK.png}}
%     \caption{Refutation of G\"odel's Axioms in Isabelle/HOL.} \label{InconsistencyIsabelleK}
%   \end{subfigure}
% \end{figure*}


\begin{figure}[t]
\centerline{\includegraphics[width=1\columnwidth]{./Scott_S5U.png}}
\caption{Full Automation of T3 in \SFiveU; Consistency of Scott's
  Axioms;  Automatic Verification of Modal Collapse} \label{Scott_S5U}
\end{figure}

% There is reason to believe that the omission was more than just an oversight. As pointed out by Fuhrmann \cite{Fuhrmann2005}, ``G\"odel vermerkt diese Konsequenz [dass aus der Definition unmittelbar folgt, da\ss alle wesentlichen Eigenschaften notwendig äquivalent sind] in einer Fußnote zur Definition. Die Definition selbst aber l\"a{\ss}t im Definiens das Konjunkt $Xx$ aus. Ohne das Konjunkt folgt jedoch die Konsequenz nicht. Es ist deshalb naheliegend anzunehmen, da\ss G\"odel die Definition so beabsichtigte, wie sie hier notiert ist und wie G\"odel sie selbst in fr\"uheren Notizbucheintragungen formuliert hat.''\footnote{Our translation: ``G\"odel remarks this consequence [that the definition of essence entails that all essences are necessarily equivalent] in a footnote to the definition. The definition, however, omits the conjunct. ... ToDo''} 
% \marginpar{ToDo: check Fuhrmann's claim. He may be pointing out a second mistake by Gödel.}

However, as explained here, the extra conjunct is in
fact crucial. Without it, G\"odel's original axioms are
inconsistent. With it, Scott's axioms are consistent (cf.~Fig.~\ref{Scott_S5U}
where the model finder Nitpick \cite{Nitpick} confirms consistency).\footnote{In
  personal communication, Dana Scott confirmed that he was unaware
  at the time that G\"odel's axioms were inconsistent.}



%  Particularly interesting
% from the perspective of this paper is that many literature
% contributions unfortunately work with or refer to G\"odel's original
% definition of essence
% $$ todo $$
% This original version avoids the conjunct added by Scott expressing
% that essential properties of an individual should be possessed by the
% individual. 

% We give here a list of examples:
% \begin{itemize}
% \item Anderson and Getting's 1996, p. 168 \cite[p.168]{AndersonGettings1968}
%   use essence without conjunct.
% \item  
% \item Look???, p. 514
% % \item Oppy 1996, p.226-227 \cite[p.226/227]{Oppy1996}, Oppy 2000, p. 364
% %   \cite[p.364]{Oppy200}, Oppy 2008, p. 1068 \cite[p.1068]{Oppy2008}:
% %   Oppy uses: ``A is an essence of x iff for every property B, x has B
% %   neces- sarily iff A entails B'' (this is from Anderson's
% %   emendation). Check is this leads to inconsistency as well. In this
% %   case we may write something like: 

% %  Oppy presents an critical
% %   assessment in order to conclude that G\"odel's argument fails to
% %   convince. Oppy apparently does not see that he is assessing an
% %   inconsistent axiom system (and he in fact creates adaptations that
% %   suffer the same problems).
% % \item 
% \end{itemize}








\section{Automating HOML in HOL}\label{sec:homlinhol}

Logic textbooks % \cite{ToDo:which}
commonly utilize higher-order logic in an informal/semi-formal way as
a meta-language to introduce the syntax and the semantics of object
logics of interest, in which reasoning problems in concrete
application domains can be modeled and solved with pen and paper. In
fact, this approach can also be followed on the computer (using HOL as
a formal meta-language) for even very challenging object logics (such
as HOML) to enable interactive and automated theorem proving with
existing theorem provers for HOL.

% Just as commonly the case in logic textbooks, in the embeddings
% approach classical higher-order logic is utilised as a meta-language
% to encode the syntax and the semantics of object logics in which the
% proof problems of interest are then modeled in. Modulo the embedding
% state of the art automated theorem provers for classical higher-order
% logic can then be utilised as object logic reasoning tools.

For a computational analysis of G\"odel's ontological argument, the embedding of HOMLs such as \textbf{K},
\textbf{KB} and \textbf{S5} with various domain conditions (possibilist and actualist quantification)
is required. This idea has been successfully employed in related work
\cite{C40}. The embedding of HOML is in fact
straightforward. Formulas in HOML are \emph{lifted}, i.e., converted into predicates
over worlds, which are themselves explicitly represented as
terms. The logical constants of HOML are translated to HOL terms in such a way that, for instance, 
%$\neg \varphi$, $\varphi\vee\psi$
$\Box \varphi$ and $\Diamond \varphi$ (relative to a current world
$w_o$) are mapped, respectively, to the HOL formulas
$\forall w. (r w_0 w) \imp (\varphi w)$ and
$\exists w. (r w_0 w) \wedge (\varphi w)$. This form of embedding is
precisely the well-known standard translation
\cite{DBLP:journals/logcom/Ohlbach91}, which is here intra-logically
realized --- and extended for quantifiers --- in HOL by stating a set
of equations defining the logical constants (Fig.~\ref{QML_S5U}). The
resulting object logic is the HOML \textbf{K} with rigid terms and constant
domains (possibilist quantifiers). Other logics (e.g. \textbf{KB}, \textbf{S5}) are
embedded by adding axioms that restrict the accessibility relation
$r$. Varying domains and actualist quantifiers can be simulated by using
 an existence predicate to guard the
quantifiers. The embedding approach is, therefore, very flexible.


\subsection{Improved Embedding}\label{sec:improvedembedding}
\begin{figure}[t]
\centerline{\includegraphics[width=1\columnwidth,height=10cm]{./QML_S5U.png}}
\caption{Improved Embedding of \SFiveU} \label{QML_S5U}
\end{figure}

The modal logic \SFive requires that the accessibility relation be
reflexive, symmetric and transitive. The usual approach to embed
\SFive would be to use the standard translation for \textbf{K} described above
and to state that $r$ is an equivalence relation, e.g., by postulating
the following axioms:
\begin{itemize}
\item Reflexivity: $\forall x. (r~x~x)$
\item Symmetry: $\forall x. \forall y. (r~x~y) \rightarrow (r~y~x)$ 
\item Transitivity: $\forall x. \forall y. \forall z. (r~x~y) \wedge (r~y~z) \rightarrow (r~x~z)$
\end{itemize}
Instead, we consider here an alternative description, that we call \SFiveU, based on the following condition on $r$:
\begin{itemize}
\item Universality: $\forall x. \forall y. (r~x~y)$
\end{itemize}

It is important to note that $\vDash_{\SFive} \varphi$ iff $\vDash_{\SFiveU} \varphi$ 
\cite{Blackburn}, and therefore \SFive and \SFiveU are traditionally considered to be 
two different descriptions of the same modal logic. Nevertheless, \SFive and \SFiveU 
differ in the shapes of frames they admit:
%It is easy to see that \SFiveU is at least as strong as \SFive:
%universality entails that $r$ is an equivalence relation. \SFiveU is,
%in fact, \emph{strictly stronger} than \SFive. 
\SFiveU only admits
complete\footnote{A graph is \emph{complete} iff there is a directed
  edge connecting every ordered pair of vertices.} frames, whereas
\SFive admits non-complete frames as long as all their components are
complete. In other words, in \SFiveU we face one single equivalence
class of possible worlds, while in \SFive we may face several
disconnected equivalence classes. In fact, for this reason, \SFiveU is
considered as metaphysically more appropriate by some philosophers;
cf. \cite[p.~127]{williamson13}.

Furthermore, for \SFiveU an improved embedding is possible. Universality implies
that the guarding predicates in the definitions of $\Box$ and
$\Diamond$ always hold. Therefore, they can be omitted and the
accessibility relation can be dispensed altogether. The modal
operators can then be defined merely as:
% \begin{itemize}
% \item $\BlackBox \varphi \equiv \forall w. (\varphi w)$ 
% \item $\BlackDia \varphi \equiv \exists w. (\varphi w)$
% \end{itemize}
\[\Box \varphi \equiv \lambda w. \forall v.  \varphi(v) \qquad \text{and}
\qquad \Dia \varphi \equiv \lambda w.\exists v. \varphi(v)\]


The new embedding of \SFiveU in Isabelle/HOL is shown in
Fig.~\ref{QML_S5U}.
With this improved embedding, the final theorem T3
(\textit{Necessarily, there exists God}) can be derived from
Scott's consistent version of the axioms fully automatically.  The fully
automatic proof has been generated (in about 2.5 seconds) by the theorem prover
\textsc{Leo-II}~\cite{leo2} and subsequently verified in the proof
assistant Isabelle/HOL~\cite{NPW02}, as shown in Fig.~\ref{Scott_S5U}.
The collaboration between the two systems has been orchestrated by
Isabelle's Sledgehammer tool \cite{Sledgehammer}.

With the embedding used by Benzm\"uller and Woltzenlogel-Paleo \shortcite{C40}, the
provers still had to be given the intermediate theorem T2 and the corollary
C in order to manage to prove T3.

Another evidence that the new embedding provides a significant performance boost 
is the successful automatic verification in Isabelle/HOL (with its automatic tactic Meson) of the modal
collapse \cite{Sobel}, which is one of the most strongly criticized
`side-effects' of G\"odel's and Scott's variants of the proof. In previous work \shortcite{C40}
the modal collapse has been proven by the higher-order provers
\textsc{Satallax} \cite{Satallax} and \textsc{Leo-II}, but a fully automatic
verification in the highly trusted Isabelle/HOL system still failed
\cite{J28}.  The success with the new embedding can be seen in Fig.~\ref{Scott_S5U}.




\subsection{Improved Syntax in Isabelle}\label{sec:improvedsyntax}

Wider adoption of HOL theorem proving technology for
reasoning about and within embedded object logics, especially among
non-expert users, is still hindered by the gap between the syntax used
by people, when they write logical formulas with pen and paper, and
the syntax used by HOL theorem provers. Even when the syntax of the
underlying higher-order system is elegant (as is the case in
Isabelle/HOL), the embedding of HOML into HOL may easily expose
details of HOL that may be uncommon to the user, disturbing his/her
experience while using the system. To illustrate this point,
Fig.~\ref{UglyEssence} shows how the definition of essence looked like
in previous work \cite{J28}, where advanced syntax-sugaring features
were not used. It looks notably higher-order, and its style differs
significantly from the common style seen in works on modal logics and
the ontological argument. The following specific issues can be
enumerated:
\begin{enumerate}
\item $\lambda$-abstractions, which are typically a HOL feature, appear explicitly in places where they did not need to in a pure HOML formulation (cf. G\"odel's manuscript, Fig.~\ref{GoedelScript}).
\item Quantifiers appear as higher-order defined constants, and not as binders. This forces the user to read (and write) formulas of the form $\forall (\lambda x. A(x))$ instead of the more common $\forall x. A(x)$.
\item The lifted modal connectives are represented by prefixing the
  letter ``m'' (e.g. $m\wedge$ and $m\imp$). The prefix disturbs the
  user, as it constantly reminds him/her that there is something
  unusual about the modal connectives.
\item Higher-order parenthesis conventions for the application of a predicate to a term are used. 
Instead of reading $\psi(y)$, as he/she would expect, he/she has to read $(\psi ~ y)$. Outside niche areas in computer science, the former syntax is more widely known than the latter.
\end{enumerate}


\begin{figure}
\centerline{\includegraphics[width=1\columnwidth]{./UglyEssence.png}}
\caption{Definition of Essence using Old Syntax} \label{UglyEssence}
\end{figure}

In the embedding presented here, in Fig.~\ref{QML_S5U}, advanced
syntax-sugaring effects provided by Isabelle were used to prevent
issues as those enumerated above. The possibility to define boldface connectives allows us to drop the prefix; ``\texttt{binder}'' annotations enable modal quantifiers to be used in the standard binding way and reduce the need for explicit lambda abstractions; and a careful choice of priorities for infix connectives gives the parenthesis conventions that are more familiar to the user. As desired, the definition of essence in Fig.~\ref{Scott_S5U} is undeniably more immediately recognizable and comprehensible than the definition in Fig.~\ref{UglyEssence}. The embedding technique is now completely transparent to the user.

The syntax improvements described here render the computer-assisted analysis of ontological arguments accessible to a wider audience and ease the adoption of logic-based artificial intelligence tools by philosophers interested in topics where modal logic reasoning is required.


\section{Intuitive Inconsistency Argument} \label{sec:inconsistency}

In the typical workflow during an attempt to prove a conjecture with a theorem prover, it is customary to check the consistency of the axioms first. For if the axioms are inconsistent, anything (including the conjecture) would be trivially derivable in classical logic (\emph{ex falso quodlibet}). Surprisingly, when this routine check was performed on G\"odel's axioms \cite{C40}, the \textsc{Leo-II} prover claimed that the axioms were inconsistent. Unfortunately, the refutation generated by \textsc{Leo-II} was barely human-readable. The text file was 153 lines\footnote{Long lines with an average of 184 characters per line.} long and used machine-oriented calculus (higher-order resolution \cite{W47}) and syntax (TPTP THF \cite{J22}). Part of the file is displayed in Fig.~\ref{LEO-Proof}.

\begin{figure*}
\centerline{\includegraphics[width=\textwidth]{./LEO-Proof.png}}
\caption{Lines 115--120 of \textsc{Leo-II}'s refutation. Primitive
  substitutions (e.g. with the empty property) are highlighted. In the
red part (see $\boldsymbol{\leftarrow}$), property variable $\texttt{SV8}$ has been instantiated with the
$\lambda \texttt{SV16}_{\mu}. \lambda \texttt{SV17}_{\iota}. \bot$, i.e., the (lifted) empty property.
} \label{LEO-Proof}
\end{figure*}

Although \textsc{Leo-II}'s resolution refutation is not easy to read
for humans, it did contain relevant hints to the importance of the
empty property $\lambda x. \bot$ (also denoted $\emptyset$, as in HOL it is customary to think of unary predicates as sets).
%
Note that the terms for the empty property\footnote{An additional lambda abstraction occurs in the empty property in \textsc{Leo-II}'s proof (and also in the reconstruction in Isabelle) because the embedding approach lifts the boolean type $o$ to $\iota \imp o$.} ($\lambda x. \bot$) and for the property of self-difference ($\lambda x.  x\not=x$) have identical denotations in a logic setting
with full functional and Boolean extensionality as given
here. Nevertheless, some philosophers\footnote{Private communication with Andr\'e Fuhrmann.} may actually prefer the use of
self-difference over the empty property in
the analysis below. However, for the proof to go through it is
irrelevant which notion we use and the reader may simply replace the
empty property by self-difference.


% Philosophers
% might actually prefer using the latter variant over the
% former. However, on the given logic setting, with functional and
% Boolean extensionality, both terms have identical
% denotations. Consequently, they are mutually replacable below and the
% reader may simply use the version he prefers.

\subsection{Informal Argument} \label{sec:arg1}
Based on the hints found in \textsc{Leo-II}'s refutation, we conceived the following informal explanation for the inconsistency of G\"odel's axioms:

\begin{enumerate}
\item From G\"odel's definition of essence 
(${\ess{\phi}{x} \biimp {\allq \psi} (\psi(x)
\imp {\nec} \allq y (\phi(y) \imp \psi(y)))}$) it follows that the
empty property (or self-difference) is an essence of every individual (\textbf{Empty Essence Lemma}): 
$$\allq x\; (\ess{\emptyset}{x})$$

\item From theorem T1 (\textit{Positive properties are possibly
  exemplified}: ${\allq \phi} [P(\phi) \imp {\pos}  \exq x
  \phi(x)]$) and axiom A5 (``necessary existence'' is a positive property: $P(\NE)$ ), it follows that $\NE$ is possibly exemplified:
  $$
  \pos \exq x [\NE(x)]
  $$
 
\item Expanding the definition of ``necessary existence''
  (${\NE(x) \equiv \allq \phi [\ess{\phi}{x} \imp \nec \exq y
    \phi(y)]}$), the following is obtained:
  $$
  \pos \exq x [\allq \varphi [ \ess{\varphi}{x} \imp \nec \exq y [\varphi(y)] ] ]
  $$

\item The sentence above holds for all $\varphi$ and thus, in
  particular, for the empty property (or self-difference):
$$
\pos \exq x [ \ess{\emptyset}{x} \imp \nec \exq y [\emptyset(y)] ]
$$

\item By the Empty Essence Lemma, the antecedent of the implication above is valid. Therefore, the sentence above entails:
$$
\pos \exq x [ \nec \exq y [\emptyset(y)] ]
$$ 

\item By definition of $\emptyset$: 
$$
\pos \exq x [ \nec \bot ]
$$

\item As the existential quantifier is binding no variable within its scope, the sentence is equi-valid with:
$$\pos \nec \bot $$

\item To see that the sentence above is contradictory, we may reason semantically, thinking of possible worlds. If $w_0$ is the arbitrary current world, the $\pos$ operator forces the existence of a world $w$ accessible from $w_0$ such that $\nec \bot$ is true in $w$. But $\nec \bot$ can only be true in $w$, if there is no world $w'$ accessible from $w$. In logics with a reflexive or symmetric accessibility relation (e.g. \KB), it is easy to see that there must be a world $w'$ accessible from $w$: either $w'$ itself, in case of a reflexive relation, or $w_0$, in case of a symmetric relation. In fact, even in \K, with no accessibility condition, there must be a world $w'$ accessible from $w$. The reason is that $\pos \nec \bot$ should be \emph{valid} (true in all worlds). Therefore, it is true in $w$ as well, where the existence of an accessible world $w'$ is forced by the $\pos$ operator. As a model for $\pos \nec \bot$ (which is a consequence of G\"odel's axioms) cannot be built, G\"odel's axioms are inconsistent.
\end{enumerate}

Interestingly, the refutation automatically generated by
\textsc{Leo-II} uses a symmetric accessibility relation, and thus
requires the modal logic \KB. The informal, human-constructed
refutation described above, on the other hand, requires only the
weaker modal logic \K. In our experiments \textsc{Leo-II} (like all
other HOL provers) was still too weak to automatically prove the
inconsistency already in logic \K. Hence, this remains an open problem for automated
theorem provers.


\subsection{Argument Reconstruction in Isabelle}  \label{sec:arg2}
\begin{figure}[t]
\centerline{\includegraphics[width=1\columnwidth]{./InconsistencyIsabelleK.png}}
\caption{Inconsistency of G\"odel's Axioms in HOML \K verified in Isabelle/HOL} \label{InconsistencyIsabelleK}
\end{figure}
\begin{figure*}[t]
  \centering
  \begin{subfigure}[t]{0.715\textwidth}
    \includegraphics[width=\textwidth]{./Manuscript2.png}
    \caption{G\"{o}del's manuscript, with mutually inconsistent axioms and
      definitions highlighted (with permission from the Kurt G\"odel Papers, Shelby White and Leon Levy Archives Center, Princeton, NJ, USA, on deposit at Princeton University)} \label{GoedelScript} 
  \end{subfigure}
   \begin{subfigure}[t]{0.28\textwidth}
     \includegraphics[width=\textwidth,height=7.2cm]{./Inconsistency_S5U_direct.png}
     \caption{Inconsistency in HOML \SFiveU} \label{Inconsistency_S5U} 
   \end{subfigure}
% \includegraphics[width=.73\textwidth]{./Images/Manuscript2.png} \hfill
% \includegraphics[width=.26\textwidth]{./Images/Inconsistency_S5U_direct.png}
 \caption{The inconsistency in G\"{o}del's manuscript has been
   detected and verified by HOL ATPs} 
\end{figure*}
To verify the correctness of the informal argument explained above, it
was reconstructed in Isabelle/HOL, using Metis\footnote{Metis, unlike
  external provers such as \textsc{Leo-II} or Satallax, 
  constructs proofs in Isabelle's highly trusted kernel calculus.} to automate the
inessential parts (cf. Fig.~\ref {InconsistencyIsabelleK}). The essential use of the Empty Essence Lemma, on
the other hand, is explicitly stated, to ensure that Isabelle is
reconstructing the same argument. In fact, without the help of this
lemma, Metis is still not strong enough to refute G\"odel's
axioms.


\subsection{Mapping the Inconsistency to G\"odel}

The inconsistency verified in Fig.~\ref{InconsistencyIsabelleK} follows from the definition of
essence (ess), the definition of necessary existence (NE), the
axioms A1a and A2 (which entail theorem T1), and axiom A5. It remains to show that
these ingredients are actually present in G\"odel's manuscript in
Fig.~\ref{GoedelScript}. 


This can be easily seen: Axiom A1a in
Fig.~\ref{InconsistencyIsabelleK} is implied by Axiom Ax2 and the
highlighted footnote remark in Fig.~\ref{GoedelScript}. Axioms A2 and
A5 in Fig.~\ref{InconsistencyIsabelleK} correspond to Ax4 and Ax3 in
Fig.~\ref{GoedelScript}. The definitions of essence and necessary
existence are easy to identify. Therefore, the verified
inconsistency from Fig.~\ref{InconsistencyIsabelleK} does apply to 
G\"odel's original manuscript.


\subsection{Inconsistency of G\"odel's Axioms in \SFiveU}

Isabelle/HOL's Sledgehammer tool, which orchestrates calls to
external provers such as \textsc{Leo-II}, still
fails to detect the inconsistency of G\"odel's axioms in the standard
embedding of \SFive, while a direct modeling of the problem in TPTP THF syntax
in combination with a direct call of \textsc{Leo-II} succeeded. In
other words, without independent experiments with no mediation through Sledgehammer, the
inconsistency would not have been detected.


%\marginpar{ToDo: Are we really supposed to use \S instead of ``Sec.''?}
On the other hand (and further confirming the claims from \S\ref{sec:improvedembedding}), 
the reconstruction in Isabelle/HOL with the improved embedding for \SFiveU was more efficient: 
the inconsistency could be detected
by \textsc{Leo-II} also when called via
Sledgehammer. Moreover, the result could subsequently be verified with
Metis even without the Empty Essence Lemma (cf. Fig.~\ref{Inconsistency_S5U}). 
% Interestingly, Metis' argument is different from that of
% \S\ref{sec:arg1} and \S\ref{sec:arg2}, since axiom A1b is not used.



\section{Conclusion}\label{sec:conclusion}

The axioms and definitions in G\"odel's manuscript are inconsistent;
this was detected automatically by the prover
\textsc{Leo-II}. Here we presented a rational reconstruction and
verification of the inconsistency argument in Isabelle/HOL. This
argument is valid in all normal HOMLs including base logic \K.

We have also presented several technical improvements regarding the
semantic embedding approach. In particular, we have achieved a
nearly perfect match between pen and paper presentations in HOML and
the syntax in Isabelle/HOL. As a result, the embedding of HOML in HOL
is now fully transparent, more user-friendly and ready for wider adoption.

On the other hand, there is still room for many pragmatical
improvements in Isabelle; just one example: in default setting,
Sledgehammer does not immediately inform the user when a proof has
been found and instead silently first executes a series of
time-consuming proof analysis processes (e.g. its dependency
minimization), before it eventually reports success. For G\"odel's
theorem T3 (\textit{Necessarily, there exists God}), for example, this
phase of silence takes several minutes --- during which the user might
actually give up on the proof attempt --- even though \textsc{Leo-II} already
reported success to Sledgehammer after 2.5 seconds.

More importantly, our work reveals a challenge for automated reasoning:
the (so far partially manual) extraction of an informal argument from a formal proof. 
Without accompanying human-understandable explanations,
the proofs generated by provers such as \textsc{Leo-II} or Metis, will
presumably be only of limited value for philosophers, for whom intuitive
arguments remain crucial for the acceptance of novel results.

Another open problem that we solved in this paper is a fully automatic
proof of T3 directly from Scott's axioms. Again, this proof was
contributed by \textsc{Leo-II}. This has become possible only
after we provided a more efficient embedding for HOML \SFiveU (instead of \SFive) in HOL.


Both the automated detection of the inconsistency in G\"odel's axioms
and the fully automatic proof of T3 from Scott's axioms demonstrate
the potential of our AI technology for philosophy: this technology is,
in its current state of development, already capable of contributing novel results to
metaphysics and to conduct reasoning steps at granularity-levels
beyond common human capabilities.  
\vfill

\noindent\textbf{Acknowledgments:} We thank Chad Brown, who contributed to the rational  reconstruction
of the inconsistency argument.

\pagebreak
%A discussion between Chad Brown and Benzm\"uller
% significantly influenced the rational reconstruction of the inconsistency reported by \textsc{LEO-II}.

%\pagebreak

%German Research Foundation DFG, Chad Brown


%ToDo: reduce paper length to 7 pages

%ToDo: use \textsc for all systems. Not only for LEO-II.

%\small 
%% The file named.bst is a bibliography style file for BibTeX 0.99c

\small
\bibliographystyle{named}
%\bibliography{Bibliography}

\begin{thebibliography}{}

\bibitem[\protect\citeauthoryear{Adams}{1995}]{Adams}
R.M. Adams.
\newblock Introductory note to *1970.
\newblock In {\em {Kurt G\"odel: Collected Works Vol. 3: Unpubl. Essays and
  Letters}}. Oxford Univ. Press, 1995.

\bibitem[\protect\citeauthoryear{Anderson and
  Gettings}{1996}]{AndersonGettings}
A.C. Anderson and M.~Gettings.
\newblock {G\"odel} ontological proof revisited.
\newblock In {\em {G\"odel'96: Logical Foundations of Mathematics, Computer
  Science, and Physics: Lecture Notes in Logic 6}}, pages 167--172. {Springer},
  1996.

\bibitem[\protect\citeauthoryear{Anderson}{1990}]{Anderson}
C.A. Anderson.
\newblock Some emendations of {G{\"o}del's} ontological proof.
\newblock {\em Faith and Philosophy}, 7(3), 1990.

\bibitem[\protect\citeauthoryear{Andrews}{2014}]{andrewsSEP}
P.B. Andrews.
\newblock Church's type theory.
\newblock In E.N. Zalta, editor, {\em The Stanford Encyclopedia of Philosophy}.
  Spring 2014 edition, 2014.

\bibitem[\protect\citeauthoryear{Anselm}{1078}]{Proslogion}
St. Anselm.
\newblock Proslogion.
\newblock In M.~Charlesworth, editor, {\em St.~Anselm's Proslogion}.
  Oxford:OUP, 1078.
\newblock Republished in 1965.

\bibitem[\protect\citeauthoryear{Benzm{\"u}ller and Paulson}{2013}]{J23}
C.~Benzm{\"u}ller and L.C. Paulson.
\newblock Quantified multimodal logics in simple type theory.
\newblock {\em Logica Universalis}, 7(1):7--20, 2013.

\bibitem[\protect\citeauthoryear{Benzm{\"u}ller and
  Woltzenlogel-Paleo}{2013a}]{J30}
C.~Benzm{\"u}ller and B.~Woltzenlogel-Paleo.
\newblock Formalization, mechanization and automation of {G{\"o}del's} proof of
  {God's} existence.
\newblock {\em arXiv:1308.4526}, 2013.
\newblock Preprint available as arXiv:1308.4526.

\bibitem[\protect\citeauthoryear{Benzm\"uller and
  Woltzenlogel-Paleo}{2013b}]{J28}
C.~Benzm\"uller and B.~Woltzenlogel-Paleo.
\newblock {G{\"o}del's God in Isabelle/HOL}.
\newblock {\em Archive of Formal Proofs}, 2013, 2013.

\bibitem[\protect\citeauthoryear{Benzm{\"u}ller and
  Woltzenlogel-Paleo}{2014}]{C40}
C.~Benzm{\"u}ller and B.~Woltzenlogel-Paleo.
\newblock Automating {G\"{o}del's} ontological proof of {God}'s existence with
  higher-order automated theorem provers.
\newblock In 
  {\em ECAI 2014}, volume 263 of {\em Frontiers in Artificial Intelligence and
  Applications}, pages 93 -- 98. IOS Press, 2014.

\bibitem[\protect\citeauthoryear{Benzm{\"u}ller \bgroup \em et al.\egroup
  }{2015}]{leo2}
Christoph Benzm{\"u}ller, Lawrence~C. Paulson, Nik Sultana, and Frank
  Thei{\ss}.
\newblock The higher-order prover {LEO-II}.
\newblock {\em J. of Automated Reasoning}, 55(4):389--404, 2015.

\bibitem[\protect\citeauthoryear{Bj{\o}rdal}{1999}]{Bjordal}
F.~Bj{\o}rdal.
\newblock Understanding {G\"{o}del’s} ontological argument.
\newblock In T.~Childers, editor, {\em The Logica Yearbook 1998}. Filosofia,
  1999.

\bibitem[\protect\citeauthoryear{Blackburn \bgroup \em et al.\egroup
  }{2001}]{Blackburn}
P.~Blackburn, M.~de~Rijke, and Y.~Venema.
\newblock {\em Modal Logic}.
\newblock Cambridge University Press, 2001.

\bibitem[\protect\citeauthoryear{Blanchette and Nipkow}{2010}]{Nitpick}
J.C. Blanchette and T.~Nipkow.
\newblock Nitpick: A counterexample generator for higher-order logic based on a
  relational model finder.
\newblock In {\em ITP 2010}, number 6172 in LNCS, pages 131--146. Springer,
  2010.

\bibitem[\protect\citeauthoryear{Blanchette \bgroup \em et al.\egroup
  }{2013}]{Sledgehammer}
J.C. Blanchette, S.~B\"ohme, and L.C. Paulson.
\newblock Extending {Sledgehammer} with {SMT} solvers.
\newblock {\em J. of Automated Reasoning}, 51(1):109--128, 2013.

\bibitem[\protect\citeauthoryear{Brown}{2012}]{Satallax}
C.E. Brown.
\newblock Satallax: An automated higher-order prover.
\newblock In {\em IJCAR 2012}, number 7364 in LNAI, pages 111 -- 117. Springer,
  2012.

\bibitem[\protect\citeauthoryear{Fitelson and Zalta}{2007}]{FitelsonZalta}
Branden Fitelson and Edward~N. Zalta.
\newblock Steps toward a computational metaphysics.
\newblock {\em J. Philosophical Logic}, 36(2):227--247, 2007.

\bibitem[\protect\citeauthoryear{Fuhrmann}{2016}]{fuhrmann15:_blogg_goedel}
A.~Fuhrmann.
\newblock Blogging {G\"odel}: His ontological argument in the public eye.
\newblock In K.~\'{S}wi{e}torzecka, editor, {\em Forthcoming in
  \textit{G\"odel’s Ontological Argument - History, Modifications, and
  Controversies}}. 2016.

\bibitem[\protect\citeauthoryear{G\"odel}{1970}]{GoedelNotes}
K.~G\"odel.
\newblock {\em Appx. A: Notes in Kurt G\"odel's Hand}, pages 144--145.
\newblock In Sobel \shortcite{sobel2004logic}, 1970.

\bibitem[\protect\citeauthoryear{H\'ajek}{1996}]{Hajek1}
P.~H\'ajek.
\newblock Magari and others on {G\"odel’s} ontological proof.
\newblock In A.~Ursini and P.~Agliano, editors, {\em Logic and algebra}, page
  125–135. Dekker, New York etc., 1996.

\bibitem[\protect\citeauthoryear{H\'ajek}{2001}]{Hajek2}
P.~H\'ajek.
\newblock {Der Mathematiker und die Frage der Existenz Gottes}.
\newblock In B.~Buldt~et al., editor, {\em {Kurt G{\"o}del. Wahrheit und
  Beweisbarkeit}}, pages 325--336. öbv \& hpt, Wien, 2001.
\newblock ISBN 3-209-03835-X.

\bibitem[\protect\citeauthoryear{H{\'{a}}jek}{2002}]{Hajek3}
P.~H{\'{a}}jek.
\newblock A new small emendation of {G{\"{o}}del's} ontological proof.
\newblock {\em Studia Logica}, 71(2):149--164, 2002.

\bibitem[\protect\citeauthoryear{Hazen}{1998}]{Hazen}
A.P. Hazen.
\newblock On {G\"odel's} ontological proof.
\newblock {\em Australasian Journal of Philosophy}, 76:361--377, 1998.

\bibitem[\protect\citeauthoryear{McCune}{2010}]{prover9-mace4}
W.~McCune.
\newblock {Prover9} and {Mace4} (2005--2010).
\newblock {\small \verb|http://www.cs.unm.edu/~mccune/prover9/|}, 2010.

\bibitem[\protect\citeauthoryear{Muskens}{2006}]{homl}
R.~Muskens.
\newblock {Higher Order Modal Logic}.
\newblock In P.~Blackburn~et al., editor, {\em Handbook of Modal Logic},
  Studies in Logic and Practical Reasoning, pages 621--653. Elsevier,
  Dordrecht, 2006.

\bibitem[\protect\citeauthoryear{Nipkow \bgroup \em et al.\egroup
  }{2002}]{NPW02}
T.~Nipkow, L.~Paulson, and M.~Wenzel.
\newblock {\em {Isabelle/HOL: A Proof Assistant for Higher-Order Logic}}.
\newblock Number 2283 in LNCS. Springer, 2002.

\bibitem[\protect\citeauthoryear{Ohlbach}{1991}]{DBLP:journals/logcom/Ohlbach91}
H.J. Ohlbach.
\newblock Semantics-based translation methods for modal logics.
\newblock {\em J. Log. Comput.}, 1(5):691--746, 1991.

\bibitem[\protect\citeauthoryear{Oppenheimer and Zalta}{2011}]{oppenheimer11}
P.E. Oppenheimer and E.N. Zalta.
\newblock A computationally-discovered simplification of the ontological
  argument.
\newblock {\em Australasian J. of Philosophy}, 89(2):333--349, 2011.

\bibitem[\protect\citeauthoryear{Oppy}{1996}]{oppy96:_goedel_ontol_argum}
G.~Oppy.
\newblock G{\"o}delian ontological arguments.
\newblock {\em Analysis}, 56(4):226--230, 1996.

\bibitem[\protect\citeauthoryear{Oppy}{2000}]{oppy00:_respon_gettin}
G.~Oppy.
\newblock Response to {Gettings}.
\newblock {\em Analysis}, 60(4):363--367, 2000.

\bibitem[\protect\citeauthoryear{Oppy}{2008}]{oppy08:_higher_order_ontol_argum}
G.~Oppy.
\newblock Higher-order ontological arguments.
\newblock {\em Philosophy Compass}, 3(5):1066--1078, 2008.

\bibitem[\protect\citeauthoryear{Oppy}{2015}]{sep-ontological-arguments}
G.~Oppy.
\newblock Ontological arguments.
\newblock In Edward~N. Zalta, editor, {\em The Stanford Encyclopedia of
  Philosophy}. Spring 2015 edition, 2015.

\bibitem[\protect\citeauthoryear{Owre \bgroup \em et al.\egroup
  }{1992}]{cade92-pvs}
{S.} Owre, {J.}~{M.} Rushby, and {N.} Shankar.
\newblock {PVS:} {A} prototype verification system.
\newblock In Deepak Kapur, editor, {\em CADE}, volume 607 of {\em LNAI}, pages
  748--752, Saratoga, {NY}, jun 1992. Springer.

\bibitem[\protect\citeauthoryear{Rushby}{2013}]{rushby13}
J.~Rushby.
\newblock The ontological argument in {PVS}.
\newblock In {\em Proc.~of CAV Workshop ``Fun With Formal Methods''}, St.
  Petersburg, Russia, 2013.

\bibitem[\protect\citeauthoryear{Scott}{1972}]{ScottNotes}
D.~Scott.
\newblock {\em Appx. B: Notes in Dana Scott's Hand}, pages 145--146.
\newblock In Sobel \shortcite{sobel2004logic}, 1972.

\bibitem[\protect\citeauthoryear{Sobel}{1987}]{Sobel}
J.H. Sobel.
\newblock G\"odel's ontological proof.
\newblock In {\em {On Being and Saying. Essays for Richard Cartwright}}, pages
  241--261. {MIT Press}, 1987.

\bibitem[\protect\citeauthoryear{Sobel}{2004}]{sobel2004logic}
J.H. Sobel.
\newblock {\em Logic and Theism: Arguments for and Against Beliefs in God}.
\newblock Cambridge U. Press, 2004.

\bibitem[\protect\citeauthoryear{Sultana and Benzm{\"u}ller}{2013}]{W47}
N.~Sultana and C.~Benzm{\"u}ller.
\newblock Understanding {LEO-II's} proofs.
\newblock In Konstantin Korovin, Stephan Schulz, and Eugenia Ternovska,
  editors, {\em IWIL 2012}, volume~22 of {\em EPiC Series}, pages 33--52,
  Merida, Venezuela, 2013. EasyChair.

\bibitem[\protect\citeauthoryear{Sutcliffe and Benzm{\"u}ller}{2010}]{J22}
G.~Sutcliffe and C.~Benzm{\"u}ller.
\newblock Automated reasoning in higher-order logic using the {TPTP THF}
  infrastructure.
\newblock {\em J. of Formalized Reasoning}, 3(1):1--27, 2010.

\bibitem[\protect\citeauthoryear{Williamson}{2013}]{williamson13}
T.~Williamson.
\newblock {\em Modal Logic as Metaphysics}.
\newblock Oxford:OUP, 2013.

\end{thebibliography}


\end{document}

