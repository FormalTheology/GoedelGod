\documentclass{article}

\usepackage{latexsym}
\usepackage{bussproofs}
\EnableBpAbbreviations
\newcommand{\rl}[1]{\RightLabel{#1}}

% Logical symbols
\newcommand{\imp}{\rightarrow}
\newcommand{\biimp}{\leftrightarrow}
\newcommand{\all}{\forall}
\newcommand{\ex}{\exists}
\newcommand{\seq}{\vdash}
\newcommand{\nec}{\Box} % necessarily
\newcommand{\pos}{\Diamond} % possibly

\author{Christoph Benzm\"{u}ller and Bruno Woltzenlogel Paleo}

\title{Formalizations of G\"{o}del's Ontological Proof of God's Existence}

\begin{document}

\maketitle

Das Logikgenie Kurt Gödel (1906-1978) hat vor seinem Tod ein
Argumentationsformalismus für den Beweis der Existenz Gottes
vorgeschlagen. Nun ist es Christoph Benzmüller, Wissenschaftler an der FU
Berlin im Dahlem Center for Intelligent Systems, in Kooperation mit
Bruno Woltzenlogel-Paleo von der TU Wien gelungen, Kurt Gödel's
Gottesbeweis auf dem Computer abzubilden, den Formalismus zu
verifizieren und zu automatisieren. Der Computer hat dann von Gödels
Prämissen ausgehend bewiesen: es gäbe Gott. "Theorembeweisen" heisst
die mathematische Disziplin bei der solche automatisierten Systemen
entwickelt werden.

Versuche sich der Existenz (oder Nichtexistenz) Gottes mithilfe
abstrakter, ontologischer Argumentationen zu nähern haben eine lange
Tradition in der Philosophie. Vor Gödel haben schon St. Anselm,
Descartes und Leibniz ihre jeweilige Beweise vorgelegt. Was Gödel als
Logiker antrieb war die Frage, ob sich ausgehend von einer kleinen
Anzahl (debattierbarer) grundlegender Axiome und Begriffsdefinitionen
die notwendige Existenz Gottes mithilfe einer mathematisch präzisen,
formal-logischen Argumentationskette nachweisen lässt.

In dem Computer wurde die Kernaussage, die notwendige Existenz Gottes,
wie von Gödel vorgeschlagen in vier Einzelargumente zerlegt und die
Beweisführung und Verifikation jedes dieser vier Einzelargumente wurde
dann mithilfe von Theorembeweisern automatisch erzeugt. Auch die
Widerspruchsfreiheit der grundlegenden Axiome und Definitionen konnte
mithilfe des Computers analysiert werden.

In der theoretischen Philosophie erfolgt die formal-logische
Auseinandersetzung mit solchen Gottesbeweisen bisher ausschließlich
mit Papier und Bleistift. Der Einsatz von Computern scheiterte bisher,
weil es den "Logiken" der verfügbaren Systeme noch an Ausdruckstärke
fehlte um die abstrakten Begriffe adäquat zu erfassen. Gödels
Gottesbeweis verwendet beispielsweise eine komplexe höherstufige
sogenannte Modallogik, die mit Begriffen wie "möglich" und
"notwendig" arbeitet.

Aktuelle Arbeiten von Christoph Benzmüller und Larry Paulson
(Cambridge University, UK) zeigen jedoch, dass sich viele
ausdrucksstarke Logiken, einschliesslich höherstufige Modallogiken, in
die klassische höherstufige Logik einbetten lassen. Letztere kann
also, mit einigen Einschränkungen, als eine universelle Logik
angesehen werden. Für die klassische höherstufige Logik wurden aber in
den vergangenen Jahren leistungsfähige
Theorembeweiser entwickelt und solche Systeme kommen nun in den
Arbeiten von Benzmüller und Woltzenlogel-Paleo zum Einsatz.

Die von den Wissenschaftlern gewählte Herangehensweise eröffnet damit
neue Perspektiven für eine computer-assistierte theoretische
Philosophie. Dabei bleibt die kritische Auseinandersetzung mit
grundlegenden Begriffen und Axiomen dem Menschen vorbehalten. Der
Computer kann aber helfen, Fehler in den formal-logischen Anteilen zu
minimieren: der Computer kann die Argumentationskette überprüfen und so
teilweise Leibniz Diktum bei logischen Streiterien einsetzen:
Calculemus --- Lasst es uns berechnen!

\end{document}