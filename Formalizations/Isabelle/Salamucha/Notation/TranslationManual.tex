\documentclass[english,ngerman,paper=a4,]{scrartcl}%
\usepackage[T1]{fontenc}% Für Vektorschriften
\usepackage[utf8]{inputenc}%Eingabekodierung für Sonderzeichen und Umlaute etc.
\usepackage{kpfonts}
%\usepackage{libertine}% Alternative Schrift
\usepackage{microtype}
\usepackage{babel} %Sprachumschaltung
\usepackage{xcolor}
%%%%%%%%%%%%%%% LAYOUT %%%%%%%%%%%%%%%5
\usepackage{geometry}%Alternativ typearea (KOMA)
\geometry{lmargin=2cm,tmargin=2cm,rmargin=1.5cm,bmargin=2cm}
\usepackage{marvosym}
\setlength{\parindent}{0pt}
\usepackage{parskip}
\usepackage{subscript}


\usepackage{tcolorbox}
\usepackage{array} %Neue Spaltentrennungen
\usepackage{ragged2e}%Flattersatz mit Trennungen
\newcolumntype{L}[1]{>{\RaggedRight}p{#1}}% Großuchstaben für Trennungen
\newcolumntype{C}[1]{>{\Centering}p{#1}}% Großuchstaben für Trennungen
\newcolumntype{R}[1]{>{\RaggedLeft}p{#1}}% Großuchstaben für Trennungen
\usepackage{tabularx}% Tabellen bestimmter Breite
\usepackage{longtable}
\usepackage{multicol}
\usepackage{caption} % BILD UND TABELLENBESCHRIFTUNGEN
\usepackage{graphicx}
\usepackage{multido} %Schleifen
\usepackage{siunitx}
\usepackage{csquotes}


\usepackage{amsmath}





\usepackage{xspace}

\usepackage{pdfpages}


%\usepackage{abkuerzungen}

\usepackage{jurabib}

\jurabibsetup{
	commabeforerest,
	ibidem=strict,
	citefull=first,
	see,
	titleformat={all,colonsep},
}

\renewcommand*{\jbauthorfont}{\textsc}
\renewcommand*{\biblnfont}{\scshape\textbf}
\renewcommand*{\bibfnfont}{\normalfont\textbf}

\AddTo\bibsgerman{%
	\renewcommand*{\ibidemname}{Ebd.}
	\renewcommand*{\ibidemmidname}{Ebd.}
}






\usepackage{varioref} %% Kluge Verweise %%



\newcommand\Makro[1]{\texttt{\textbackslash#1}}
\newcommand\Merksatz[1]{%
	\begin{tabularx}{\linewidth}{
	!{\vrule width 2mm}
	X
	!{\vrule width 2mm}    }
	#1
	\end{tabularx}
}

%\renewcommand\thefootnote{\alph{footnote}}


%\usepackage[colorlinks]{hyperref} %%%% HYPERREF SOLLTE BIS AUF AUSNAHMEN IMMER DAS LETZTE PAKET SEIN %%%%
%%%%%%%%%%%%%%%%%%PRÄAMBEL ENDE%%%%%%%%%%%%%%%%%%%%%%%%%%%%%%%
%
\usepackage%
%[hidelinks]%
{hyperref} %%%% HYPERREF SOLLTE BIS AUF AUSNAHMEN IMMER DAS LETZTE PAKET SEIN %%%%
%
%%%%%%%%%%%%%%%%%%%%%%%%%%%%%%%%%%
%%%%%%%%%%%%%%%%%%%%%%%%%%%%%%%%%%
\begin{document}%
%
%
%
%
%
%
\renewcommand{\baselinestretch}{1.50}\normalsize%
%\tableofcontents\newpage%
%
%%%%%%%%%%%%%%%%%%%%%

\makeatletter
% Remove \@date and spacing following it from \@maketitle
\patchcmd{\@maketitle}% <cmd>
  {{\usekomafont{date}{\@date \par}}%
    \vskip \z@ \@plus 1em}% <search>
  {}% <replace>
  {}{}% <success><failure>
\makeatother
%
\title{Translation Manual for Salamuchas Formalization of the Ex Motu Argument}
\maketitle
\section{Logical Notation}
%%
%

\begin{tabular}{|c|c|c|c|}
\hline 
\textbf{Name of Operator} & \textbf{Samalamucha first variant} & \textbf{Second variant} & \textbf{\enquote{Standard notation}} \\ 
\hline 
Conjunction & $p . q$ & $p \wedge q$ & $p \wedge q$ \\ 
\hline 
Disjunction & $p \vee q$ & $p \vee q$  & $p \vee q$ \\ 
\hline 
Negation & $\sim p$ & $\sim p$ & $\neg p$ \\ 
\hline 
(Material) Implication & $p \supset q$ & $p \rightarrow q$ & $p \rightarrow q$ \\ 
\hline 
Biconditional & $p \equiv q$ & $p \equiv q$ & $p \leftrightarrow q$ \\ 
\hline 
Universal Quantifier & $[x]. \phi (x)$ & $\wedge x \phi (x)$ & $\forall x. \phi (x)$ \\ 
\hline 
Existential Quantifier & $[\exists x]. \phi (x)$& $\vee x \phi (x)$ & $\exists x. \phi (x)$ \\ 
\hline 
\end{tabular} 

\section{Definitions of relations, predicates, etc.}
\begin{tabular}{|c|c|c|}
\hline 
\textbf{Notation} & \textbf{Name/Description} & \textbf{Definition} \\ 
\hline 
$x R y$ & Relation (at first!) & the usual \\ 
\hline 
$C'R$& Set of all elements of a relation & $x \in C'R \Leftrightarrow \exists t. (t R x \vee x R t)$ \\ 
\hline 
$K(R)$ & Ordering relation & transitive, irreflexive, connected relation \\ 
\hline 
$f x$ & x is in motion &  - \\ 
\hline 
$x R y$ & x moves y&  - \\ 
\hline 
$M_{x}(a)$ or $a M x$ & a is the proper part of x [sic] & -  \\ 
\hline
$xA_{s}y$ & x is in aspect S \emph{in actu} to y [emph. orig.] & -  \\ 
\hline 
$C x$ & x is a body & -  \\ 
\hline 
 $t_{i}Fx$  & $t_i$ is the duration of movement of x [sic] & -  \\ 
\hline 
 $F(t_i)$ or $Ht_i$ & $t_i$ is the finite period of time [sic]& -  \\ 
\hline 
\end{tabular}
%
%%%%%%%%%%%%%%%%%%%%%%
\bibliographystyle{jurabib}
\bibliography{/home/knork/Dokumente/Latex/Philosophie.bib}


\end{document}