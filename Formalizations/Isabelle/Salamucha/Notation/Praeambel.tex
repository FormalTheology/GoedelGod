\usepackage[T1]{fontenc}% Für Vektorschriften
\usepackage[utf8]{inputenc}%Eingabekodierung für Sonderzeichen und Umlaute etc.
\usepackage{kpfonts}
%\usepackage{libertine}% Alternative Schrift
\usepackage{microtype}
\usepackage{babel} %Sprachumschaltung
\usepackage{xcolor}
%%%%%%%%%%%%%%% LAYOUT %%%%%%%%%%%%%%%5
\usepackage{geometry}%Alternativ typearea (KOMA)
\geometry{lmargin=2cm,tmargin=2cm,rmargin=1.5cm,bmargin=2cm}
\usepackage{marvosym}
\setlength{\parindent}{0pt}
\usepackage{parskip}
\usepackage{subscript}


\usepackage{tcolorbox}
\usepackage{array} %Neue Spaltentrennungen
\usepackage{ragged2e}%Flattersatz mit Trennungen
\newcolumntype{L}[1]{>{\RaggedRight}p{#1}}% Großuchstaben für Trennungen
\newcolumntype{C}[1]{>{\Centering}p{#1}}% Großuchstaben für Trennungen
\newcolumntype{R}[1]{>{\RaggedLeft}p{#1}}% Großuchstaben für Trennungen
\usepackage{tabularx}% Tabellen bestimmter Breite
\usepackage{longtable}
\usepackage{multicol}
\usepackage{caption} % BILD UND TABELLENBESCHRIFTUNGEN
\usepackage{graphicx}
\usepackage{multido} %Schleifen
\usepackage{siunitx}
\usepackage{csquotes}


\usepackage{amsmath}





\usepackage{xspace}

\usepackage{pdfpages}


%\usepackage{abkuerzungen}

\usepackage{jurabib}

\jurabibsetup{
	commabeforerest,
	ibidem=strict,
	citefull=first,
	see,
	titleformat={all,colonsep},
}

\renewcommand*{\jbauthorfont}{\textsc}
\renewcommand*{\biblnfont}{\scshape\textbf}
\renewcommand*{\bibfnfont}{\normalfont\textbf}

\AddTo\bibsgerman{%
	\renewcommand*{\ibidemname}{Ebd.}
	\renewcommand*{\ibidemmidname}{Ebd.}
}






\usepackage{varioref} %% Kluge Verweise %%



\newcommand\Makro[1]{\texttt{\textbackslash#1}}
\newcommand\Merksatz[1]{%
	\begin{tabularx}{\linewidth}{
	!{\vrule width 2mm}
	X
	!{\vrule width 2mm}    }
	#1
	\end{tabularx}
}

%\renewcommand\thefootnote{\alph{footnote}}


%\usepackage[colorlinks]{hyperref} %%%% HYPERREF SOLLTE BIS AUF AUSNAHMEN IMMER DAS LETZTE PAKET SEIN %%%%
%%%%%%%%%%%%%%%%%%PRÄAMBEL ENDE%%%%%%%%%%%%%%%%%%%%%%%%%%%%%%