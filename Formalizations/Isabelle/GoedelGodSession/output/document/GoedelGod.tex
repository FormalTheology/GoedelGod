%
\begin{isabellebody}%
\def\isabellecontext{GoedelGod}%
%
\isadelimtheory
%
\endisadelimtheory
%
\isatagtheory
%
\endisatagtheory
{\isafoldtheory}%
%
\isadelimtheory
%
\endisadelimtheory
%
\isamarkupsection{Introduction%
}
\isamarkuptrue%
%
\begin{isamarkuptext}%
A formalization and (partial) automation of Dana Scott's version \cite{ScottNotes}
 of Goedel's ontological argument \cite{GoedelNotes} in quantified modal logic KB (QML KB) is presented. 
 QML KB is in turn modeled as a fragment of classical higher-order logic (HOL). 
 Thus, the formalization is essentially a formalization in HOL. The employed embedding 
 of QML KB in HOL is adapting the work of Benzm\"uller and Paulson \cite{J23,B9}.
 Note that the QML KB formalization employs quantification over individuals and 
 quantification over sets of individuals (poperties).

 The formalization presented here has been carried and formally verified in the Isabelle/HOL 
 proof assistant; for more information on this system see the textbook by Nipkow, 
 Paulson, and Wenzel \cite{Isabelle}. More recent tutorials on Isabelle can be found 
 at the Isabelle homepage: \url{http://isabelle.in.tum.de}.
 

 Some further notes: \sloppy
 \begin{enumerate}
 \item This LaTeX text document has been produced automatically from the Isabelle source
 code document at 
 \url{https://github.com/FormalTheology/GoedelGod/tree/master/Formalizations/Isabelle/GoedelGodSession} 
 with the Isabelle build tool.
 \item The formalization presented here is related to the THF \cite{J22} and 
    Coq \cite{Coq} formalizations available at
    \url{https://github.com/FormalTheology/GoedelGod/tree/master/Formalizations/}.
 \item All reasoning gaps in Scott's proof script have been automated 
    with Sledgehammer \cite{Sledgehammer} performing remote calls to the higher-order automated
    theorem prover LEO-II \cite{LEO-II}. These calls then suggested respective 
    Metis \cite{Metis} calls as given below. The Metis proofs are then verified in Isabelle/HOL.
 \item For consistency checking, the Nitpick model finder \cite{Nitpick} has been employed.
 \end{enumerate}%
\end{isamarkuptext}%
\isamarkuptrue%
%
\isamarkupsection{An Embedding of QML KB in HOL%
}
\isamarkuptrue%
%
\begin{isamarkuptext}%
The types \isa{i} for possible worlds (or states) and \isa{mu} for individuals 
are introduced.%
\end{isamarkuptext}%
\isamarkuptrue%
\ \ \isacommand{typedecl}\isamarkupfalse%
\ i\ \ \ \ \ %
\isamarkupcmt{the type for possible worlds%
}
\ \isanewline
\ \ \isacommand{typedecl}\isamarkupfalse%
\ mu\ \ \ \ %
\isamarkupcmt{the type for indiviuals%
}
%
\begin{isamarkuptext}%
Possible worlds are connected by an accessibility relation .%
\end{isamarkuptext}%
\isamarkuptrue%
\ \ \isacommand{consts}\isamarkupfalse%
\ r\ {\isacharcolon}{\isacharcolon}\ {\isachardoublequoteopen}i\ {\isasymRightarrow}\ i\ {\isasymRightarrow}\ bool{\isachardoublequoteclose}\ {\isacharparenleft}\isakeyword{infixr}\ {\isachardoublequoteopen}r{\isachardoublequoteclose}\ {\isadigit{7}}{\isadigit{0}}{\isacharparenright}\ \ \ \ %
\isamarkupcmt{accessibility relation r%
}
%
\begin{isamarkuptext}%
The \isa{B} axiom (symmetry) for relation r is stated. \isa{B} is needed only 
for proving theorem T3.%
\end{isamarkuptext}%
\isamarkuptrue%
\ \ \isacommand{axiomatization}\isamarkupfalse%
\ \isakeyword{where}\ sym{\isacharcolon}\ {\isachardoublequoteopen}x\ r\ y\ {\isasymlongrightarrow}\ y\ r\ x{\isachardoublequoteclose}%
\begin{isamarkuptext}%
QML formulas are identified with certain HOL terms of type \isa{i\ {\isasymRightarrow}\ bool}. 
This type will be abbreviated in the remainder as \isa{{\isasymsigma}}%
\end{isamarkuptext}%
\isamarkuptrue%
\ \ \isacommand{type{\isacharunderscore}synonym}\isamarkupfalse%
\ {\isasymsigma}\ {\isacharequal}\ {\isachardoublequoteopen}{\isacharparenleft}i\ {\isasymRightarrow}\ bool{\isacharparenright}{\isachardoublequoteclose}%
\begin{isamarkuptext}%
The classical connectives $\neg, \wedge, \Rightarrow$, and $\forall$
(for individuals and over sets of individuals) and $\exists$ (over individuals) are
lifted to type $\sigma$. Further connectives could be introduced analogously. \isa{definition} 
could be used instead of \isa{abbreviation}; the latter are always fully expanded/rewritten,
which is fine here, where the focus has been on proof automation, but which would lead to 
overly complex proof tasks in a purely interactive session.%
\end{isamarkuptext}%
\isamarkuptrue%
\ \ \isacommand{abbreviation}\isamarkupfalse%
\ mnot\ {\isacharcolon}{\isacharcolon}\ {\isachardoublequoteopen}{\isasymsigma}\ {\isasymRightarrow}\ {\isasymsigma}{\isachardoublequoteclose}\ {\isacharparenleft}{\isachardoublequoteopen}m{\isasymnot}{\isachardoublequoteclose}{\isacharparenright}\ \isakeyword{where}\ {\isachardoublequoteopen}m{\isasymnot}\ {\isasymphi}\ {\isasymequiv}\ {\isacharparenleft}{\isasymlambda}w{\isachardot}\ {\isasymnot}\ {\isasymphi}\ w{\isacharparenright}{\isachardoublequoteclose}\ \ \ \ \isanewline
\ \ \isacommand{abbreviation}\isamarkupfalse%
\ mand\ {\isacharcolon}{\isacharcolon}\ {\isachardoublequoteopen}{\isasymsigma}\ {\isasymRightarrow}\ {\isasymsigma}\ {\isasymRightarrow}\ {\isasymsigma}{\isachardoublequoteclose}\ {\isacharparenleft}\isakeyword{infixr}\ {\isachardoublequoteopen}m{\isasymand}{\isachardoublequoteclose}\ {\isadigit{7}}{\isadigit{9}}{\isacharparenright}\ \isakeyword{where}\ {\isachardoublequoteopen}{\isasymphi}\ m{\isasymand}\ {\isasympsi}\ {\isasymequiv}\ {\isacharparenleft}{\isasymlambda}w{\isachardot}\ {\isasymphi}\ w\ {\isasymand}\ {\isasympsi}\ w{\isacharparenright}{\isachardoublequoteclose}\ \ \ \isanewline
\ \ \isacommand{abbreviation}\isamarkupfalse%
\ mimplies\ {\isacharcolon}{\isacharcolon}\ {\isachardoublequoteopen}{\isasymsigma}\ {\isasymRightarrow}\ {\isasymsigma}\ {\isasymRightarrow}\ {\isasymsigma}{\isachardoublequoteclose}\ {\isacharparenleft}\isakeyword{infixr}\ {\isachardoublequoteopen}m{\isasymRightarrow}{\isachardoublequoteclose}\ {\isadigit{7}}{\isadigit{4}}{\isacharparenright}\ \isakeyword{where}\ {\isachardoublequoteopen}{\isasymphi}\ m{\isasymRightarrow}\ {\isasympsi}\ {\isasymequiv}\ {\isacharparenleft}{\isasymlambda}w{\isachardot}\ {\isasymphi}\ w\ {\isasymlongrightarrow}\ {\isasympsi}\ w{\isacharparenright}{\isachardoublequoteclose}\ \ \isanewline
\ \ \isacommand{abbreviation}\isamarkupfalse%
\ mforall{\isacharunderscore}ind\ {\isacharcolon}{\isacharcolon}\ {\isachardoublequoteopen}{\isacharparenleft}mu\ {\isasymRightarrow}\ {\isasymsigma}{\isacharparenright}\ {\isasymRightarrow}\ {\isasymsigma}{\isachardoublequoteclose}\ {\isacharparenleft}{\isachardoublequoteopen}{\isasymforall}i{\isachardoublequoteclose}{\isacharparenright}\ \isakeyword{where}\ {\isachardoublequoteopen}{\isasymforall}i\ {\isasymPhi}\ {\isasymequiv}\ {\isacharparenleft}{\isasymlambda}w{\isachardot}\ {\isasymforall}x{\isachardot}\ {\isasymPhi}\ x\ w{\isacharparenright}{\isachardoublequoteclose}\ \ \ \isanewline
\ \ \isacommand{abbreviation}\isamarkupfalse%
\ mexists{\isacharunderscore}ind\ {\isacharcolon}{\isacharcolon}\ {\isachardoublequoteopen}{\isacharparenleft}mu\ {\isasymRightarrow}\ {\isasymsigma}{\isacharparenright}\ {\isasymRightarrow}\ {\isasymsigma}{\isachardoublequoteclose}\ {\isacharparenleft}{\isachardoublequoteopen}{\isasymexists}i{\isachardoublequoteclose}{\isacharparenright}\ \isakeyword{where}\ {\isachardoublequoteopen}{\isasymexists}i\ {\isasymPhi}\ {\isasymequiv}\ {\isacharparenleft}{\isasymlambda}w{\isachardot}\ {\isasymexists}x{\isachardot}\ {\isasymPhi}\ x\ w{\isacharparenright}{\isachardoublequoteclose}\isanewline
\ \ \isacommand{abbreviation}\isamarkupfalse%
\ mforall{\isacharunderscore}indset\ {\isacharcolon}{\isacharcolon}\ {\isachardoublequoteopen}{\isacharparenleft}{\isacharparenleft}mu\ {\isasymRightarrow}\ {\isasymsigma}{\isacharparenright}\ {\isasymRightarrow}\ {\isasymsigma}{\isacharparenright}\ {\isasymRightarrow}\ {\isasymsigma}{\isachardoublequoteclose}\ {\isacharparenleft}{\isachardoublequoteopen}{\isasymforall}p{\isachardoublequoteclose}{\isacharparenright}\ \isakeyword{where}\ {\isachardoublequoteopen}{\isasymforall}p\ P\ {\isasymequiv}\ {\isacharparenleft}{\isasymlambda}w{\isachardot}\ {\isasymforall}x{\isachardot}\ P\ x\ w{\isacharparenright}{\isachardoublequoteclose}\isanewline
\ \ \isacommand{abbreviation}\isamarkupfalse%
\ mbox\ {\isacharcolon}{\isacharcolon}\ {\isachardoublequoteopen}{\isasymsigma}\ {\isasymRightarrow}\ {\isasymsigma}{\isachardoublequoteclose}\ {\isacharparenleft}{\isachardoublequoteopen}{\isasymbox}{\isachardoublequoteclose}{\isacharparenright}\ \isakeyword{where}\ {\isachardoublequoteopen}{\isasymbox}\ {\isasymphi}\ {\isasymequiv}\ {\isacharparenleft}{\isasymlambda}w{\isachardot}\ {\isasymforall}v{\isachardot}\ {\isasymnot}\ w\ r\ v\ {\isasymor}\ {\isasymphi}\ v{\isacharparenright}{\isachardoublequoteclose}\isanewline
\ \ \isacommand{abbreviation}\isamarkupfalse%
\ mdia\ {\isacharcolon}{\isacharcolon}\ {\isachardoublequoteopen}{\isasymsigma}\ {\isasymRightarrow}\ {\isasymsigma}{\isachardoublequoteclose}\ {\isacharparenleft}{\isachardoublequoteopen}{\isasymdiamond}{\isachardoublequoteclose}{\isacharparenright}\ \isakeyword{where}\ {\isachardoublequoteopen}{\isasymdiamond}\ {\isasymphi}\ {\isasymequiv}\ {\isacharparenleft}{\isasymlambda}w{\isachardot}\ {\isasymexists}v{\isachardot}\ w\ r\ v\ {\isasymand}\ {\isasymphi}\ v{\isacharparenright}{\isachardoublequoteclose}%
\begin{isamarkuptext}%
For the grounding of lifted formulas the meta-predicate \isa{valid} is introduced.%
\end{isamarkuptext}%
\isamarkuptrue%
\ \ \isacommand{abbreviation}\isamarkupfalse%
\ valid\ {\isacharcolon}{\isacharcolon}\ {\isachardoublequoteopen}{\isasymsigma}\ {\isasymRightarrow}\ bool{\isachardoublequoteclose}\ {\isacharparenleft}{\isachardoublequoteopen}{\isacharbrackleft}{\isacharunderscore}{\isacharbrackright}{\isachardoublequoteclose}{\isacharparenright}\ \isakeyword{where}\ {\isachardoublequoteopen}{\isacharbrackleft}p{\isacharbrackright}\ {\isasymequiv}\ {\isasymforall}w{\isachardot}\ p\ w{\isachardoublequoteclose}%
\begin{isamarkuptext}%
The model finder Nitpick confirms that the axioms and definitions above are consistent. 
Unfortunately, the respective command syntax for Nitpick is not very intuitive.%
\end{isamarkuptext}%
\isamarkuptrue%
\ \ \isacommand{lemma}\isamarkupfalse%
\ True\ \isacommand{nitpick}\isamarkupfalse%
\ {\isacharbrackleft}satisfy{\isacharcomma}\ user{\isacharunderscore}axioms{\isacharcomma}\ expect\ {\isacharequal}\ genuine{\isacharbrackright}%
\isadelimproof
\ %
\endisadelimproof
%
\isatagproof
\isacommand{oops}\isamarkupfalse%
%
\endisatagproof
{\isafoldproof}%
%
\isadelimproof
%
\endisadelimproof
%
\begin{isamarkuptext}%
Constant symbol \isa{P} (G\"odel's "Positive") is introduced.%
\end{isamarkuptext}%
\isamarkuptrue%
\ \ \isacommand{consts}\isamarkupfalse%
\ P\ {\isacharcolon}{\isacharcolon}\ {\isachardoublequoteopen}{\isacharparenleft}mu\ {\isasymRightarrow}\ {\isasymsigma}{\isacharparenright}\ {\isasymRightarrow}\ {\isasymsigma}{\isachardoublequoteclose}%
\begin{isamarkuptext}%
The meaning of \isa{P} is restricted by axioms \isa{A{\isadigit{1}}{\isacharparenleft}a{\isacharslash}b{\isacharparenright}}: $\all \phi 
[P(\neg \phi) \biimp \neg P(\phi)]$ (Either a property or its negation is positive, but not both.) 
and \isa{A{\isadigit{2}}}: $\all \phi \all \psi [(P(\phi) \wedge \nec \all x [\phi(x) \imp \psi(x)]) 
\imp P(\psi)]$ (A property necessarily implied by a positive property is positive.).%
\end{isamarkuptext}%
\isamarkuptrue%
\ \ \isacommand{axiomatization}\isamarkupfalse%
\ \isakeyword{where}\isanewline
\ \ \ \ A{\isadigit{1}}a{\isacharcolon}\ {\isachardoublequoteopen}{\isacharbrackleft}{\isasymforall}p\ {\isacharparenleft}{\isasymlambda}{\isasymPhi}{\isachardot}\ P\ {\isacharparenleft}{\isasymlambda}x{\isachardot}\ m{\isasymnot}\ {\isacharparenleft}{\isasymPhi}\ x{\isacharparenright}{\isacharparenright}\ m{\isasymRightarrow}\ m{\isasymnot}\ {\isacharparenleft}P\ {\isasymPhi}{\isacharparenright}{\isacharparenright}{\isacharbrackright}{\isachardoublequoteclose}\ \isakeyword{and}\isanewline
\ \ \ \ A{\isadigit{1}}b{\isacharcolon}\ {\isachardoublequoteopen}{\isacharbrackleft}{\isasymforall}p\ {\isacharparenleft}{\isasymlambda}{\isasymPhi}{\isachardot}\ m{\isasymnot}\ {\isacharparenleft}P\ {\isasymPhi}{\isacharparenright}\ m{\isasymRightarrow}\ P\ {\isacharparenleft}{\isasymlambda}x{\isachardot}\ m{\isasymnot}\ {\isacharparenleft}{\isasymPhi}\ x{\isacharparenright}{\isacharparenright}{\isacharparenright}{\isacharbrackright}{\isachardoublequoteclose}\ \isakeyword{and}\isanewline
\ \ \ \ A{\isadigit{2}}{\isacharcolon}\ \ {\isachardoublequoteopen}{\isacharbrackleft}{\isasymforall}p\ {\isacharparenleft}{\isasymlambda}{\isasymPhi}{\isachardot}\ {\isasymforall}p\ {\isacharparenleft}{\isasymlambda}{\isasympsi}{\isachardot}\ {\isacharparenleft}P\ {\isasymPhi}\ m{\isasymand}\ {\isasymbox}\ {\isacharparenleft}{\isasymforall}i\ {\isacharparenleft}{\isasymlambda}X{\isachardot}\ {\isasymPhi}\ X\ m{\isasymRightarrow}\ {\isasympsi}\ X{\isacharparenright}{\isacharparenright}{\isacharparenright}\ m{\isasymRightarrow}\ P\ {\isasympsi}{\isacharparenright}{\isacharparenright}{\isacharbrackright}{\isachardoublequoteclose}%
\begin{isamarkuptext}%
We prove theorem T1: $\all \varphi [P(\varphi) \imp \pos \ex x \varphi(x)]$ (Positive 
properties are possibly exemplified). T1 is proved directly by Sledghammer with command \isa{sledgehammer\ {\isacharbrackleft}provers\ {\isacharequal}\ remote{\isacharunderscore}leo{\isadigit{2}}\ remote{\isacharunderscore}satallax{\isacharbrackright}}. This successful attempt then suggest to 
instead try the Metis call in the line below. This Metis call generates a proof object that is 
verified in Isabelle/HOL's kernel.%
\end{isamarkuptext}%
\isamarkuptrue%
\ \ \isacommand{theorem}\isamarkupfalse%
\ T{\isadigit{1}}{\isacharcolon}\ {\isachardoublequoteopen}{\isacharbrackleft}{\isasymforall}p\ {\isacharparenleft}{\isasymlambda}{\isasymPhi}{\isachardot}\ P\ {\isasymPhi}\ m{\isasymRightarrow}\ {\isasymdiamond}\ {\isacharparenleft}{\isasymexists}i\ {\isasymPhi}{\isacharparenright}{\isacharparenright}{\isacharbrackright}{\isachardoublequoteclose}\ \ \isanewline
\ \ \isacommand{sledgehammer}\isamarkupfalse%
\ {\isacharbrackleft}provers\ {\isacharequal}\ remote{\isacharunderscore}leo{\isadigit{2}}{\isacharbrackright}\isanewline
%
\isadelimproof
\ \ %
\endisadelimproof
%
\isatagproof
\isacommand{using}\isamarkupfalse%
\ A{\isadigit{2}}\ A{\isadigit{1}}a\ \isacommand{by}\isamarkupfalse%
\ metis%
\endisatagproof
{\isafoldproof}%
%
\isadelimproof
%
\endisadelimproof
%
\begin{isamarkuptext}%
Next, the symbol \isa{G}, for "God-like", is introduced and defined 
as $G(x) \biimp \forall \phi [P(\phi) \to \phi(x)]$ (A God-like being possesses 
all positive properties:).%
\end{isamarkuptext}%
\isamarkuptrue%
\ \ \isacommand{definition}\isamarkupfalse%
\ G\ {\isacharcolon}{\isacharcolon}\ {\isachardoublequoteopen}mu\ {\isasymRightarrow}\ {\isasymsigma}{\isachardoublequoteclose}\ \isakeyword{where}\ {\isachardoublequoteopen}G\ {\isacharequal}\ {\isacharparenleft}{\isasymlambda}x{\isachardot}\ {\isasymforall}p\ {\isacharparenleft}{\isasymlambda}{\isasymPhi}{\isachardot}\ P\ {\isasymPhi}\ m{\isasymRightarrow}\ {\isasymPhi}\ x{\isacharparenright}{\isacharparenright}{\isachardoublequoteclose}%
\begin{isamarkuptext}%
Axiom \isa{A{\isadigit{3}}} is added: $P(G)$ (The property of being God-like is positive.).
Sledgehammer and Metis then prove corollary \isa{C}: $\pos \ex x G(x)$ 
(Possibly, God exists.).%
\end{isamarkuptext}%
\isamarkuptrue%
\ \ \isacommand{axiomatization}\isamarkupfalse%
\ \isakeyword{where}\ A{\isadigit{3}}{\isacharcolon}\ \ {\isachardoublequoteopen}{\isacharbrackleft}P\ G{\isacharbrackright}{\isachardoublequoteclose}\ \isanewline
\isanewline
\ \ \isacommand{corollary}\isamarkupfalse%
\ C{\isacharcolon}\ {\isachardoublequoteopen}{\isacharbrackleft}{\isasymdiamond}\ {\isacharparenleft}{\isasymexists}i\ G{\isacharparenright}{\isacharbrackright}{\isachardoublequoteclose}\ \isanewline
\ \ \isacommand{sledgehammer}\isamarkupfalse%
\ {\isacharbrackleft}provers\ {\isacharequal}\ remote{\isacharunderscore}leo{\isadigit{2}}{\isacharbrackright}\isanewline
%
\isadelimproof
\ \ %
\endisadelimproof
%
\isatagproof
\isacommand{using}\isamarkupfalse%
\ A{\isadigit{3}}\ T{\isadigit{1}}\ \isacommand{by}\isamarkupfalse%
\ metis%
\endisatagproof
{\isafoldproof}%
%
\isadelimproof
%
\endisadelimproof
%
\begin{isamarkuptext}%
We add axiom \isa{A{\isadigit{4}}}: $\all \phi [P(\phi) \to \Box \; P(\phi)]$ 
(Positive properties are necessarily positive).%
\end{isamarkuptext}%
\isamarkuptrue%
\ \ \isacommand{axiomatization}\isamarkupfalse%
\ \isakeyword{where}\ A{\isadigit{4}}{\isacharcolon}\ \ {\isachardoublequoteopen}{\isacharbrackleft}{\isasymforall}p\ {\isacharparenleft}{\isasymlambda}{\isasymPhi}{\isachardot}\ P\ {\isasymPhi}\ m{\isasymRightarrow}\ {\isasymbox}\ {\isacharparenleft}P\ {\isasymPhi}{\isacharparenright}{\isacharparenright}{\isacharbrackright}{\isachardoublequoteclose}%
\begin{isamarkuptext}%
Symbol \isa{ess}, for "Essence", is introduced and defined as 
$\ess{\phi}{x} \biimp \phi(x) \wedge \all \psi (\psi(x) \imp \nec \all y (\phi(y) 
\imp \psi(y)))$ (An \emph{essence} of an individual is a property possessed by it 
and necessarily implying any of its properties.).%
\end{isamarkuptext}%
\isamarkuptrue%
\ \ \isacommand{definition}\isamarkupfalse%
\ ess\ {\isacharcolon}{\isacharcolon}\ {\isachardoublequoteopen}{\isacharparenleft}mu\ {\isasymRightarrow}\ {\isasymsigma}{\isacharparenright}\ {\isasymRightarrow}\ mu\ {\isasymRightarrow}\ {\isasymsigma}{\isachardoublequoteclose}\ {\isacharparenleft}\isakeyword{infixr}\ {\isachardoublequoteopen}ess{\isachardoublequoteclose}\ {\isadigit{8}}{\isadigit{5}}{\isacharparenright}\ \isakeyword{where}\isanewline
\ \ \ \ {\isachardoublequoteopen}{\isasymPhi}\ ess\ x\ {\isacharequal}\ {\isasymPhi}\ x\ m{\isasymand}\ {\isasymforall}p\ {\isacharparenleft}{\isasymlambda}{\isasympsi}{\isachardot}\ {\isasympsi}\ x\ m{\isasymRightarrow}\ {\isasymbox}\ {\isacharparenleft}{\isasymforall}i\ {\isacharparenleft}{\isasymlambda}y{\isachardot}\ {\isasymPhi}\ y\ m{\isasymRightarrow}\ {\isasympsi}\ y{\isacharparenright}{\isacharparenright}{\isacharparenright}{\isachardoublequoteclose}%
\begin{isamarkuptext}%
Next, Sledgehammer and Metis prove theorem \isa{T{\isadigit{2}}}: $\all x [G(x) \imp \ess{G}{x}]$ 
(Being God-like is an essence of any God-like being).%
\end{isamarkuptext}%
\isamarkuptrue%
\ \ \isacommand{theorem}\isamarkupfalse%
\ T{\isadigit{2}}{\isacharcolon}\ {\isachardoublequoteopen}{\isacharbrackleft}{\isasymforall}i\ {\isacharparenleft}{\isasymlambda}x{\isachardot}\ G\ x\ m{\isasymRightarrow}\ G\ ess\ x{\isacharparenright}{\isacharbrackright}{\isachardoublequoteclose}\isanewline
\ \ \isacommand{sledgehammer}\isamarkupfalse%
\ {\isacharbrackleft}provers\ {\isacharequal}\ remote{\isacharunderscore}leo{\isadigit{2}}{\isacharbrackright}\isanewline
%
\isadelimproof
\ \ %
\endisadelimproof
%
\isatagproof
\isacommand{by}\isamarkupfalse%
\ {\isacharparenleft}metis\ {\isacharparenleft}lifting{\isacharparenright}\ A{\isadigit{1}}b\ A{\isadigit{4}}\ G{\isacharunderscore}def\ ess{\isacharunderscore}def{\isacharparenright}%
\endisatagproof
{\isafoldproof}%
%
\isadelimproof
%
\endisadelimproof
%
\begin{isamarkuptext}%
Symbol \isa{NE}, for "Necessary Existence", is introduced and
defined as $\NE(x) \biimp \all \phi [\ess{\phi}{x} \imp \nec \ex y \phi(y)]$ (Necessary 
existence of an individual is the necessary exemplification of all its essences.).%
\end{isamarkuptext}%
\isamarkuptrue%
\ \ \isacommand{definition}\isamarkupfalse%
\ NE\ {\isacharcolon}{\isacharcolon}\ {\isachardoublequoteopen}mu\ {\isasymRightarrow}\ {\isasymsigma}{\isachardoublequoteclose}\ \isakeyword{where}\ {\isachardoublequoteopen}NE\ {\isacharequal}\ {\isacharparenleft}{\isasymlambda}x{\isachardot}\ {\isasymforall}p\ {\isacharparenleft}{\isasymlambda}{\isasymPhi}{\isachardot}\ {\isasymPhi}\ ess\ x\ m{\isasymRightarrow}\ {\isasymbox}\ {\isacharparenleft}{\isasymexists}i\ {\isasymPhi}{\isacharparenright}{\isacharparenright}{\isacharparenright}{\isachardoublequoteclose}%
\begin{isamarkuptext}%
Moreover, axiom \isa{A{\isadigit{5}}} is added: $P(\NE)$ (Necessary existence is a positive 
property.).%
\end{isamarkuptext}%
\isamarkuptrue%
\ \ \isacommand{axiomatization}\isamarkupfalse%
\ \isakeyword{where}\ A{\isadigit{5}}{\isacharcolon}\ \ {\isachardoublequoteopen}{\isacharbrackleft}P\ NE{\isacharbrackright}{\isachardoublequoteclose}%
\begin{isamarkuptext}%
Finally, Sledgehammer and Metis prove the main theorem \isa{T{\isadigit{3}}}: $\nec \ex x G(x)$ 
(Necessarily, God exists).%
\end{isamarkuptext}%
\isamarkuptrue%
\ \ \isacommand{theorem}\isamarkupfalse%
\ T{\isadigit{3}}{\isacharcolon}\ {\isachardoublequoteopen}{\isacharbrackleft}{\isasymbox}\ {\isacharparenleft}{\isasymexists}i\ G{\isacharparenright}{\isacharbrackright}{\isachardoublequoteclose}\ \isanewline
\ \ \isacommand{sledgehammer}\isamarkupfalse%
\ {\isacharbrackleft}provers\ {\isacharequal}\ remote{\isacharunderscore}leo{\isadigit{2}}{\isacharbrackright}\isanewline
%
\isadelimproof
\ \ %
\endisadelimproof
%
\isatagproof
\isacommand{using}\isamarkupfalse%
\ A{\isadigit{5}}\ C\ T{\isadigit{2}}\ sym\ G{\isacharunderscore}def\ NE{\isacharunderscore}def\ \isacommand{by}\isamarkupfalse%
\ metis%
\endisatagproof
{\isafoldproof}%
%
\isadelimproof
\isanewline
%
\endisadelimproof
\isanewline
\ \ \isacommand{corollary}\isamarkupfalse%
\ T{\isadigit{4}}{\isacharcolon}\ {\isachardoublequoteopen}{\isacharbrackleft}{\isasymexists}i\ G{\isacharbrackright}{\isachardoublequoteclose}\ \isanewline
\ \ \isacommand{sledgehammer}\isamarkupfalse%
\ {\isacharbrackleft}provers\ {\isacharequal}\ remote{\isacharunderscore}leo{\isadigit{2}}{\isacharbrackright}\isanewline
%
\isadelimproof
\ \ %
\endisadelimproof
%
\isatagproof
\isacommand{using}\isamarkupfalse%
\ T{\isadigit{1}}\ T{\isadigit{3}}\ sym\ G{\isacharunderscore}def\ \isacommand{by}\isamarkupfalse%
\ metis%
\endisatagproof
{\isafoldproof}%
%
\isadelimproof
%
\endisadelimproof
%
\begin{isamarkuptext}%
Finally, the consistency of the entire theory is checked with Nitpick.%
\end{isamarkuptext}%
\isamarkuptrue%
\ \ \isacommand{lemma}\isamarkupfalse%
\ True\ \isacommand{nitpick}\isamarkupfalse%
\ {\isacharbrackleft}satisfy{\isacharcomma}\ user{\isacharunderscore}axioms{\isacharcomma}\ expect\ {\isacharequal}\ genuine{\isacharbrackright}%
\isadelimproof
\ %
\endisadelimproof
%
\isatagproof
\isacommand{oops}\isamarkupfalse%
%
\endisatagproof
{\isafoldproof}%
%
\isadelimproof
%
\endisadelimproof
%
\begin{isamarkuptext}%
\paragraph{Acknowledgments:} Nik Sultana, Jasmin Blanchette and Larry Paulson provided 
very important help wrt consistency checking in Isabelle. Jasmin Blanchette instructed us on how to 
produce latex documents from Isabelle sources, and he pointed to many useful tricks in Isabelle.%
\end{isamarkuptext}%
\isamarkuptrue%
%
\isadelimtheory
%
\endisadelimtheory
%
\isatagtheory
%
\endisatagtheory
{\isafoldtheory}%
%
\isadelimtheory
%
\endisadelimtheory
\ \end{isabellebody}%
%%% Local Variables:
%%% mode: latex
%%% TeX-master: "root"
%%% End:
