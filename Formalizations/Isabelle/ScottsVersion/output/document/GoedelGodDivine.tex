%
\begin{isabellebody}%
\def\isabellecontext{GoedelGodDivine}%
%
\isadelimtheory
%
\endisadelimtheory
%
\isatagtheory
%
\endisatagtheory
{\isafoldtheory}%
%
\isadelimtheory
%
\endisadelimtheory
%
\begin{isamarkuptext}%
\newpage%
\end{isamarkuptext}%
\isamarkuptrue%
%
\isamarkupsection{What does G\"odel mean with 'Positive' properties? And what not?%
}
\isamarkuptrue%
%
\begin{isamarkuptext}%
In order to better illustrate G\"odel's notion of 'Positive' properties, we reformulate the
entire theory and use 'Divine' instead of 'Positive'. Then we introduce orthogonal predicates 
'positive' and 'negative' and we show that God-like beings may well have 'positive' and 
'negative' properties as long as all these properties are divine properties.%
\end{isamarkuptext}%
\isamarkuptrue%
%
\begin{isamarkuptext}%
The types \isa{i} for possible worlds.%
\end{isamarkuptext}%
\isamarkuptrue%
\ \ \isacommand{typedecl}\isamarkupfalse%
\ i\ \ \ \ %
\isamarkupcmt{the type for possible worlds%
}
\ \isanewline
\ \ \isacommand{typedecl}\isamarkupfalse%
\ {\isasymmu}\ \ \ \ %
\isamarkupcmt{the type for indiviuals%
}
%
\begin{isamarkuptext}%
Accessibility relation \isa{r}.%
\end{isamarkuptext}%
\isamarkuptrue%
\ \ \isacommand{consts}\isamarkupfalse%
\ r\ {\isacharcolon}{\isacharcolon}\ {\isachardoublequoteopen}i\ {\isasymRightarrow}\ i\ {\isasymRightarrow}\ bool{\isachardoublequoteclose}\ {\isacharparenleft}\isakeyword{infixr}\ {\isachardoublequoteopen}r{\isachardoublequoteclose}\ {\isadigit{7}}{\isadigit{0}}{\isacharparenright}\ \ \ \ %
\isamarkupcmt{accessibility relation r%
}
%
\begin{isamarkuptext}%
The \isa{B} axiom (symmetry).%
\end{isamarkuptext}%
\isamarkuptrue%
\ \ \isacommand{axiomatization}\isamarkupfalse%
\ \isakeyword{where}\ sym{\isacharcolon}\ {\isachardoublequoteopen}x\ r\ y\ {\isasymlongrightarrow}\ y\ r\ x{\isachardoublequoteclose}%
\begin{isamarkuptext}%
QML formulas are identified with certain HOL terms of type \isa{i\ {\isasymRightarrow}\ bool}.%
\end{isamarkuptext}%
\isamarkuptrue%
\ \ \isacommand{type{\isacharunderscore}synonym}\isamarkupfalse%
\ {\isasymsigma}\ {\isacharequal}\ {\isachardoublequoteopen}{\isacharparenleft}i\ {\isasymRightarrow}\ bool{\isacharparenright}{\isachardoublequoteclose}%
\begin{isamarkuptext}%
The classical connectives $\neg, \wedge, \rightarrow$, and $\forall$
(over individuals and over sets of individuals) and $\exists$ (over individuals) are
lifted to type $\sigma$.%
\end{isamarkuptext}%
\isamarkuptrue%
\ \ \isacommand{abbreviation}\isamarkupfalse%
\ mnot\ {\isacharcolon}{\isacharcolon}\ {\isachardoublequoteopen}{\isasymsigma}\ {\isasymRightarrow}\ {\isasymsigma}{\isachardoublequoteclose}\ {\isacharparenleft}{\isachardoublequoteopen}m{\isasymnot}{\isachardoublequoteclose}{\isacharparenright}\ \isakeyword{where}\ {\isachardoublequoteopen}m{\isasymnot}\ {\isasymphi}\ {\isasymequiv}\ {\isacharparenleft}{\isasymlambda}w{\isachardot}\ {\isasymnot}\ {\isasymphi}\ w{\isacharparenright}{\isachardoublequoteclose}\ \ \ \ \isanewline
\ \ \isacommand{abbreviation}\isamarkupfalse%
\ mand\ {\isacharcolon}{\isacharcolon}\ {\isachardoublequoteopen}{\isasymsigma}\ {\isasymRightarrow}\ {\isasymsigma}\ {\isasymRightarrow}\ {\isasymsigma}{\isachardoublequoteclose}\ {\isacharparenleft}\isakeyword{infixr}\ {\isachardoublequoteopen}m{\isasymand}{\isachardoublequoteclose}\ {\isadigit{7}}{\isadigit{9}}{\isacharparenright}\ \isakeyword{where}\ {\isachardoublequoteopen}{\isasymphi}\ m{\isasymand}\ {\isasympsi}\ {\isasymequiv}\ {\isacharparenleft}{\isasymlambda}w{\isachardot}\ {\isasymphi}\ w\ {\isasymand}\ {\isasympsi}\ w{\isacharparenright}{\isachardoublequoteclose}\ \ \ \isanewline
\ \ \isacommand{abbreviation}\isamarkupfalse%
\ mimplies\ {\isacharcolon}{\isacharcolon}\ {\isachardoublequoteopen}{\isasymsigma}\ {\isasymRightarrow}\ {\isasymsigma}\ {\isasymRightarrow}\ {\isasymsigma}{\isachardoublequoteclose}\ {\isacharparenleft}\isakeyword{infixr}\ {\isachardoublequoteopen}m{\isasymRightarrow}{\isachardoublequoteclose}\ {\isadigit{7}}{\isadigit{4}}{\isacharparenright}\ \isakeyword{where}\ {\isachardoublequoteopen}{\isasymphi}\ m{\isasymRightarrow}\ {\isasympsi}\ {\isasymequiv}\ {\isacharparenleft}{\isasymlambda}w{\isachardot}\ {\isasymphi}\ w\ {\isasymlongrightarrow}\ {\isasympsi}\ w{\isacharparenright}{\isachardoublequoteclose}\isanewline
\ \ \isacommand{abbreviation}\isamarkupfalse%
\ mor\ {\isacharcolon}{\isacharcolon}\ {\isachardoublequoteopen}{\isasymsigma}\ {\isasymRightarrow}\ {\isasymsigma}\ {\isasymRightarrow}\ {\isasymsigma}{\isachardoublequoteclose}\ {\isacharparenleft}\isakeyword{infixr}\ {\isachardoublequoteopen}m{\isasymor}{\isachardoublequoteclose}\ {\isadigit{7}}{\isadigit{8}}{\isacharparenright}\ \isakeyword{where}\ {\isachardoublequoteopen}{\isasymphi}\ m{\isasymor}\ {\isasympsi}\ {\isasymequiv}\ {\isacharparenleft}{\isasymlambda}w{\isachardot}\ {\isasymphi}\ w\ {\isasymor}\ {\isasympsi}\ w{\isacharparenright}{\isachardoublequoteclose}\isanewline
\ \ \isacommand{abbreviation}\isamarkupfalse%
\ mequiv\ {\isacharcolon}{\isacharcolon}\ {\isachardoublequoteopen}{\isasymsigma}\ {\isasymRightarrow}\ {\isasymsigma}\ {\isasymRightarrow}\ {\isasymsigma}{\isachardoublequoteclose}\ {\isacharparenleft}\isakeyword{infixr}\ {\isachardoublequoteopen}m{\isasymequiv}{\isachardoublequoteclose}\ {\isadigit{7}}{\isadigit{7}}{\isacharparenright}\ \isakeyword{where}\ {\isachardoublequoteopen}{\isasymphi}\ m{\isasymequiv}\ {\isasympsi}\ {\isasymequiv}\ {\isacharparenleft}{\isasymlambda}w{\isachardot}\ {\isacharparenleft}{\isasymphi}\ w\ {\isasymlongrightarrow}\ {\isasympsi}\ w{\isacharparenright}\ {\isasymand}\ {\isacharparenleft}{\isasympsi}\ w\ {\isasymlongrightarrow}\ {\isasymphi}\ w{\isacharparenright}{\isacharparenright}{\isachardoublequoteclose}\ \ \ \isanewline
\ \ \isacommand{abbreviation}\isamarkupfalse%
\ mforall{\isacharunderscore}ind\ {\isacharcolon}{\isacharcolon}\ {\isachardoublequoteopen}{\isacharparenleft}{\isasymmu}\ {\isasymRightarrow}\ {\isasymsigma}{\isacharparenright}\ {\isasymRightarrow}\ {\isasymsigma}{\isachardoublequoteclose}\ {\isacharparenleft}{\isachardoublequoteopen}{\isasymforall}{\isachardoublequoteclose}{\isacharparenright}\ \isakeyword{where}\ {\isachardoublequoteopen}{\isasymforall}\ {\isasymPhi}\ {\isasymequiv}\ {\isacharparenleft}{\isasymlambda}w{\isachardot}\ {\isasymforall}x{\isachardot}\ {\isasymPhi}\ x\ w{\isacharparenright}{\isachardoublequoteclose}\ \ \ \isanewline
\ \ \isacommand{abbreviation}\isamarkupfalse%
\ mforall{\isacharunderscore}indset\ {\isacharcolon}{\isacharcolon}\ {\isachardoublequoteopen}{\isacharparenleft}{\isacharparenleft}{\isasymmu}\ {\isasymRightarrow}\ {\isasymsigma}{\isacharparenright}\ {\isasymRightarrow}\ {\isasymsigma}{\isacharparenright}\ {\isasymRightarrow}\ {\isasymsigma}{\isachardoublequoteclose}\ {\isacharparenleft}{\isachardoublequoteopen}{\isasymPi}{\isachardoublequoteclose}{\isacharparenright}\ \isakeyword{where}\ {\isachardoublequoteopen}{\isasymPi}\ P\ {\isasymequiv}\ {\isacharparenleft}{\isasymlambda}w{\isachardot}\ {\isasymforall}x{\isachardot}\ P\ x\ w{\isacharparenright}{\isachardoublequoteclose}\isanewline
\ \ \isacommand{abbreviation}\isamarkupfalse%
\ mexists{\isacharunderscore}ind\ {\isacharcolon}{\isacharcolon}\ {\isachardoublequoteopen}{\isacharparenleft}{\isasymmu}\ {\isasymRightarrow}\ {\isasymsigma}{\isacharparenright}\ {\isasymRightarrow}\ {\isasymsigma}{\isachardoublequoteclose}\ {\isacharparenleft}{\isachardoublequoteopen}{\isasymexists}{\isachardoublequoteclose}{\isacharparenright}\ \isakeyword{where}\ {\isachardoublequoteopen}{\isasymexists}\ {\isasymPhi}\ {\isasymequiv}\ {\isacharparenleft}{\isasymlambda}w{\isachardot}\ {\isasymexists}x{\isachardot}\ {\isasymPhi}\ x\ w{\isacharparenright}{\isachardoublequoteclose}\isanewline
\ \ \isacommand{abbreviation}\isamarkupfalse%
\ mbox\ {\isacharcolon}{\isacharcolon}\ {\isachardoublequoteopen}{\isasymsigma}\ {\isasymRightarrow}\ {\isasymsigma}{\isachardoublequoteclose}\ {\isacharparenleft}{\isachardoublequoteopen}{\isasymbox}{\isachardoublequoteclose}{\isacharparenright}\ \isakeyword{where}\ {\isachardoublequoteopen}{\isasymbox}\ {\isasymphi}\ {\isasymequiv}\ {\isacharparenleft}{\isasymlambda}w{\isachardot}\ {\isasymforall}v{\isachardot}\ {\isasymnot}\ w\ r\ v\ {\isasymor}\ {\isasymphi}\ v{\isacharparenright}{\isachardoublequoteclose}\isanewline
\ \ \isacommand{abbreviation}\isamarkupfalse%
\ mdia\ {\isacharcolon}{\isacharcolon}\ {\isachardoublequoteopen}{\isasymsigma}\ {\isasymRightarrow}\ {\isasymsigma}{\isachardoublequoteclose}\ {\isacharparenleft}{\isachardoublequoteopen}{\isasymdiamond}{\isachardoublequoteclose}{\isacharparenright}\ \isakeyword{where}\ {\isachardoublequoteopen}{\isasymdiamond}\ {\isasymphi}\ {\isasymequiv}\ {\isacharparenleft}{\isasymlambda}w{\isachardot}\ {\isasymexists}v{\isachardot}\ w\ r\ v\ {\isasymand}\ {\isasymphi}\ v{\isacharparenright}{\isachardoublequoteclose}%
\begin{isamarkuptext}%
The meta-predicate \isa{valid} is introduced.%
\end{isamarkuptext}%
\isamarkuptrue%
\ \ \isacommand{abbreviation}\isamarkupfalse%
\ valid\ {\isacharcolon}{\isacharcolon}\ {\isachardoublequoteopen}{\isasymsigma}\ {\isasymRightarrow}\ bool{\isachardoublequoteclose}\ {\isacharparenleft}{\isachardoublequoteopen}{\isacharbrackleft}{\isacharunderscore}{\isacharbrackright}{\isachardoublequoteclose}{\isacharparenright}\ \isakeyword{where}\ {\isachardoublequoteopen}{\isacharbrackleft}p{\isacharbrackright}\ {\isasymequiv}\ {\isasymforall}w{\isachardot}\ p\ w{\isachardoublequoteclose}%
\begin{isamarkuptext}%
Constant symbol \isa{Divine} (G\"odel's `Positive') is declared.%
\end{isamarkuptext}%
\isamarkuptrue%
\ \ \isacommand{consts}\isamarkupfalse%
\ Divine\ {\isacharcolon}{\isacharcolon}\ {\isachardoublequoteopen}{\isacharparenleft}{\isasymmu}\ {\isasymRightarrow}\ {\isasymsigma}{\isacharparenright}\ {\isasymRightarrow}\ {\isasymsigma}{\isachardoublequoteclose}%
\begin{isamarkuptext}%
The meaning of \isa{Divine} is restricted by axioms \isa{A{\isadigit{1}}{\isacharparenleft}a{\isacharslash}b{\isacharparenright}}: $\all \phi 
[Divine(\neg \phi) \biimp \neg Divine(\phi)]$ (Either a property or its negation is divine, but not both.) 
and \isa{A{\isadigit{2}}}: $\all \phi \all \psi [(Divine(\phi) \wedge \nec \all x [\phi(x) \imp \psi(x)]) 
\imp Divine(\psi)]$ (A property necessarily implied by a divine property is divine).%
\end{isamarkuptext}%
\isamarkuptrue%
\ \ \isacommand{axiomatization}\isamarkupfalse%
\ \isakeyword{where}\isanewline
\ \ \ \ A{\isadigit{1}}a{\isacharcolon}\ {\isachardoublequoteopen}{\isacharbrackleft}{\isasymPi}\ {\isacharparenleft}{\isasymlambda}{\isasymPhi}{\isachardot}\ Divine\ {\isacharparenleft}{\isasymlambda}x{\isachardot}\ m{\isasymnot}\ {\isacharparenleft}{\isasymPhi}\ x{\isacharparenright}{\isacharparenright}\ m{\isasymRightarrow}\ m{\isasymnot}\ {\isacharparenleft}Divine\ {\isasymPhi}{\isacharparenright}{\isacharparenright}{\isacharbrackright}{\isachardoublequoteclose}\ \isakeyword{and}\isanewline
\ \ \ \ A{\isadigit{1}}b{\isacharcolon}\ {\isachardoublequoteopen}{\isacharbrackleft}{\isasymPi}\ {\isacharparenleft}{\isasymlambda}{\isasymPhi}{\isachardot}\ m{\isasymnot}\ {\isacharparenleft}Divine\ {\isasymPhi}{\isacharparenright}\ m{\isasymRightarrow}\ Divine\ {\isacharparenleft}{\isasymlambda}x{\isachardot}\ m{\isasymnot}\ {\isacharparenleft}{\isasymPhi}\ x{\isacharparenright}{\isacharparenright}{\isacharparenright}{\isacharbrackright}{\isachardoublequoteclose}\ \isakeyword{and}\isanewline
\ \ \ \ A{\isadigit{2}}{\isacharcolon}\ \ {\isachardoublequoteopen}{\isacharbrackleft}{\isasymPi}\ {\isacharparenleft}{\isasymlambda}{\isasymPhi}{\isachardot}\ {\isasymPi}\ {\isacharparenleft}{\isasymlambda}{\isasympsi}{\isachardot}\ {\isacharparenleft}Divine\ {\isasymPhi}\ m{\isasymand}\ {\isasymbox}\ {\isacharparenleft}{\isasymforall}\ {\isacharparenleft}{\isasymlambda}x{\isachardot}\ {\isasymPhi}\ x\ m{\isasymRightarrow}\ {\isasympsi}\ x{\isacharparenright}{\isacharparenright}{\isacharparenright}\ m{\isasymRightarrow}\ Divine\ {\isasympsi}{\isacharparenright}{\isacharparenright}{\isacharbrackright}{\isachardoublequoteclose}%
\begin{isamarkuptext}%
We prove theorem T1: $\all \varphi [Divine(\varphi) \imp \pos \ex x \varphi(x)]$ (Divine 
properties are possibly exemplified). T1 is proved directly by Sledghammer with command \isa{sledgehammer\ {\isacharbrackleft}provers\ {\isacharequal}\ remote{\isacharunderscore}leo{\isadigit{2}}{\isacharbrackright}}. This successful attempt then suggests to 
instead try the Metis call in the line below. This Metis call generates a proof object 
that is verified in Isabelle/HOL's kernel.%
\end{isamarkuptext}%
\isamarkuptrue%
\ \ \isacommand{theorem}\isamarkupfalse%
\ T{\isadigit{1}}{\isacharcolon}\ {\isachardoublequoteopen}{\isacharbrackleft}{\isasymPi}\ {\isacharparenleft}{\isasymlambda}{\isasymPhi}{\isachardot}\ Divine\ {\isasymPhi}\ m{\isasymRightarrow}\ {\isasymdiamond}\ {\isacharparenleft}{\isasymexists}\ {\isasymPhi}{\isacharparenright}{\isacharparenright}{\isacharbrackright}{\isachardoublequoteclose}\ \ \isanewline
\ \ \isacommand{sledgehammer}\isamarkupfalse%
\ {\isacharbrackleft}provers\ {\isacharequal}\ remote{\isacharunderscore}leo{\isadigit{2}}{\isacharbrackright}\ \isanewline
%
\isadelimproof
\ \ %
\endisadelimproof
%
\isatagproof
\isacommand{by}\isamarkupfalse%
\ {\isacharparenleft}metis\ A{\isadigit{1}}a\ A{\isadigit{2}}{\isacharparenright}%
\endisatagproof
{\isafoldproof}%
%
\isadelimproof
%
\endisadelimproof
%
\begin{isamarkuptext}%
Next, the symbol \isa{G} for `God-like'  is introduced and defined 
as $G(x) \biimp \forall \phi [Divine(\phi) \to \phi(x)]$ (A God-like being possesses 
all divine properties).%
\end{isamarkuptext}%
\isamarkuptrue%
\ \ \isacommand{definition}\isamarkupfalse%
\ G\ {\isacharcolon}{\isacharcolon}\ {\isachardoublequoteopen}{\isasymmu}\ {\isasymRightarrow}\ {\isasymsigma}{\isachardoublequoteclose}\ \isakeyword{where}\ {\isachardoublequoteopen}G\ {\isacharequal}\ {\isacharparenleft}{\isasymlambda}x{\isachardot}\ {\isasymPi}\ {\isacharparenleft}{\isasymlambda}{\isasymPhi}{\isachardot}\ Divine\ {\isasymPhi}\ m{\isasymRightarrow}\ {\isasymPhi}\ x{\isacharparenright}{\isacharparenright}{\isachardoublequoteclose}%
\begin{isamarkuptext}%
Axiom \isa{A{\isadigit{3}}} is added: $Divine(G)$ (The property of being God-like is divine).
Sledgehammer and Metis then prove corollary \isa{C}: $\pos \ex x G(x)$ 
(Possibly, God exists).%
\end{isamarkuptext}%
\isamarkuptrue%
\ \ \isacommand{axiomatization}\isamarkupfalse%
\ \isakeyword{where}\ A{\isadigit{3}}{\isacharcolon}\ \ {\isachardoublequoteopen}{\isacharbrackleft}Divine\ G{\isacharbrackright}{\isachardoublequoteclose}\ \isanewline
\isanewline
\ \ \isacommand{corollary}\isamarkupfalse%
\ C{\isacharcolon}\ {\isachardoublequoteopen}{\isacharbrackleft}{\isasymdiamond}\ {\isacharparenleft}{\isasymexists}\ G{\isacharparenright}{\isacharbrackright}{\isachardoublequoteclose}\ \isanewline
\ \ \isacommand{sledgehammer}\isamarkupfalse%
\ {\isacharbrackleft}provers\ {\isacharequal}\ remote{\isacharunderscore}leo{\isadigit{2}}{\isacharbrackright}%
\isadelimproof
\ %
\endisadelimproof
%
\isatagproof
\isacommand{by}\isamarkupfalse%
\ {\isacharparenleft}metis\ A{\isadigit{3}}\ T{\isadigit{1}}{\isacharparenright}%
\endisatagproof
{\isafoldproof}%
%
\isadelimproof
%
\endisadelimproof
%
\begin{isamarkuptext}%
Axiom \isa{A{\isadigit{4}}} is added: $\all \phi [Divine(\phi) \to \Box \; Divine(\phi)]$ 
(Divine properties are necessarily divine).%
\end{isamarkuptext}%
\isamarkuptrue%
\ \ \isacommand{axiomatization}\isamarkupfalse%
\ \isakeyword{where}\ A{\isadigit{4}}{\isacharcolon}\ \ {\isachardoublequoteopen}{\isacharbrackleft}{\isasymPi}\ {\isacharparenleft}{\isasymlambda}{\isasymPhi}{\isachardot}\ Divine\ {\isasymPhi}\ m{\isasymRightarrow}\ {\isasymbox}\ {\isacharparenleft}Divine\ {\isasymPhi}{\isacharparenright}{\isacharparenright}{\isacharbrackright}{\isachardoublequoteclose}%
\begin{isamarkuptext}%
Symbol \isa{ess} for `Essence' is introduced and defined as 
$\ess{\phi}{x} \biimp \phi(x) \wedge \all \psi (\psi(x) \imp \nec \all y (\phi(y) 
\imp \psi(y)))$ (An \emph{essence} of an individual is a property possessed by it 
and necessarily implying any of its properties).%
\end{isamarkuptext}%
\isamarkuptrue%
\ \ \isacommand{definition}\isamarkupfalse%
\ ess\ {\isacharcolon}{\isacharcolon}\ {\isachardoublequoteopen}{\isacharparenleft}{\isasymmu}\ {\isasymRightarrow}\ {\isasymsigma}{\isacharparenright}\ {\isasymRightarrow}\ {\isasymmu}\ {\isasymRightarrow}\ {\isasymsigma}{\isachardoublequoteclose}\ {\isacharparenleft}\isakeyword{infixr}\ {\isachardoublequoteopen}ess{\isachardoublequoteclose}\ {\isadigit{8}}{\isadigit{5}}{\isacharparenright}\ \isakeyword{where}\isanewline
\ \ \ \ {\isachardoublequoteopen}{\isasymPhi}\ ess\ x\ {\isacharequal}\ {\isasymPhi}\ x\ m{\isasymand}\ {\isasymPi}\ {\isacharparenleft}{\isasymlambda}{\isasympsi}{\isachardot}\ {\isasympsi}\ x\ m{\isasymRightarrow}\ {\isasymbox}\ {\isacharparenleft}{\isasymforall}\ {\isacharparenleft}{\isasymlambda}y{\isachardot}\ {\isasymPhi}\ y\ m{\isasymRightarrow}\ {\isasympsi}\ y{\isacharparenright}{\isacharparenright}{\isacharparenright}{\isachardoublequoteclose}%
\begin{isamarkuptext}%
Next, Sledgehammer and Metis prove theorem \isa{T{\isadigit{2}}}: $\all x [G(x) \imp \ess{G}{x}]$ 
(Being God-like is an essence of any God-like being).%
\end{isamarkuptext}%
\isamarkuptrue%
\ \ \isacommand{theorem}\isamarkupfalse%
\ T{\isadigit{2}}{\isacharcolon}\ {\isachardoublequoteopen}{\isacharbrackleft}{\isasymforall}\ {\isacharparenleft}{\isasymlambda}x{\isachardot}\ G\ x\ m{\isasymRightarrow}\ G\ ess\ x{\isacharparenright}{\isacharbrackright}{\isachardoublequoteclose}\isanewline
\ \ \isacommand{sledgehammer}\isamarkupfalse%
\ {\isacharbrackleft}provers\ {\isacharequal}\ remote{\isacharunderscore}leo{\isadigit{2}}{\isacharbrackright}%
\isadelimproof
\ %
\endisadelimproof
%
\isatagproof
\isacommand{by}\isamarkupfalse%
\ {\isacharparenleft}metis\ A{\isadigit{1}}b\ A{\isadigit{4}}\ G{\isacharunderscore}def\ ess{\isacharunderscore}def{\isacharparenright}%
\endisatagproof
{\isafoldproof}%
%
\isadelimproof
%
\endisadelimproof
%
\begin{isamarkuptext}%
Symbol \isa{NE}, for `Necessary Existence', is introduced and
defined as $\NE(x) \biimp \all \phi [\ess{\phi}{x} \imp \nec \ex y \phi(y)]$ (Necessary 
existence of an individual is the necessary exemplification of all its essences).%
\end{isamarkuptext}%
\isamarkuptrue%
\ \ \isacommand{definition}\isamarkupfalse%
\ NE\ {\isacharcolon}{\isacharcolon}\ {\isachardoublequoteopen}{\isasymmu}\ {\isasymRightarrow}\ {\isasymsigma}{\isachardoublequoteclose}\ \isakeyword{where}\ {\isachardoublequoteopen}NE\ {\isacharequal}\ {\isacharparenleft}{\isasymlambda}x{\isachardot}\ {\isasymPi}\ {\isacharparenleft}{\isasymlambda}{\isasymPhi}{\isachardot}\ {\isasymPhi}\ ess\ x\ m{\isasymRightarrow}\ {\isasymbox}\ {\isacharparenleft}{\isasymexists}\ {\isasymPhi}{\isacharparenright}{\isacharparenright}{\isacharparenright}{\isachardoublequoteclose}%
\begin{isamarkuptext}%
Moreover, axiom \isa{A{\isadigit{5}}} is added: $Divine(\NE)$ (Necessary existence is a divine 
property).%
\end{isamarkuptext}%
\isamarkuptrue%
\ \ \isacommand{axiomatization}\isamarkupfalse%
\ \isakeyword{where}\ A{\isadigit{5}}{\isacharcolon}\ \ {\isachardoublequoteopen}{\isacharbrackleft}Divine\ NE{\isacharbrackright}{\isachardoublequoteclose}%
\begin{isamarkuptext}%
Finally, Sledgehammer and Metis prove the main theorem \isa{T{\isadigit{3}}}: $\nec \ex x G(x)$ 
(Necessarily, God exists).%
\end{isamarkuptext}%
\isamarkuptrue%
\ \ \isacommand{theorem}\isamarkupfalse%
\ T{\isadigit{3}}{\isacharcolon}\ {\isachardoublequoteopen}{\isacharbrackleft}{\isasymbox}\ {\isacharparenleft}{\isasymexists}\ G{\isacharparenright}{\isacharbrackright}{\isachardoublequoteclose}\ \isanewline
\ \ \isacommand{sledgehammer}\isamarkupfalse%
\ {\isacharbrackleft}provers\ {\isacharequal}\ remote{\isacharunderscore}leo{\isadigit{2}}{\isacharbrackright}%
\isadelimproof
\ %
\endisadelimproof
%
\isatagproof
\isacommand{by}\isamarkupfalse%
\ {\isacharparenleft}metis\ A{\isadigit{5}}\ C\ T{\isadigit{2}}\ sym\ G{\isacharunderscore}def\ NE{\isacharunderscore}def{\isacharparenright}%
\endisatagproof
{\isafoldproof}%
%
\isadelimproof
%
\endisadelimproof
\isanewline
\isanewline
\ \ \isacommand{corollary}\isamarkupfalse%
\ C{\isadigit{2}}{\isacharcolon}\ {\isachardoublequoteopen}{\isacharbrackleft}{\isasymexists}\ G{\isacharbrackright}{\isachardoublequoteclose}\ \isanewline
\ \ \isacommand{sledgehammer}\isamarkupfalse%
\ {\isacharbrackleft}provers\ {\isacharequal}\ remote{\isacharunderscore}leo{\isadigit{2}}{\isacharbrackright}{\isacharparenleft}T{\isadigit{1}}\ T{\isadigit{3}}\ G{\isacharunderscore}def\ sym{\isacharparenright}%
\isadelimproof
\ %
\endisadelimproof
%
\isatagproof
\isacommand{by}\isamarkupfalse%
\ {\isacharparenleft}metis\ T{\isadigit{1}}\ T{\isadigit{3}}\ G{\isacharunderscore}def\ sym{\isacharparenright}%
\endisatagproof
{\isafoldproof}%
%
\isadelimproof
%
\endisadelimproof
%
\begin{isamarkuptext}%
The consistency of the entire theory is checked with Nitpick.%
\end{isamarkuptext}%
\isamarkuptrue%
\ \ \isacommand{lemma}\isamarkupfalse%
\ True\ \isacommand{nitpick}\isamarkupfalse%
\ {\isacharbrackleft}satisfy{\isacharcomma}\ user{\isacharunderscore}axioms{\isacharcomma}\ expect\ {\isacharequal}\ genuine{\isacharbrackright}%
\isadelimproof
\ %
\endisadelimproof
%
\isatagproof
\isacommand{oops}\isamarkupfalse%
%
\endisatagproof
{\isafoldproof}%
%
\isadelimproof
%
\endisadelimproof
%
\begin{isamarkuptext}%
It has been critisized that G\"odel's ontological argument implies what is called the 
modal collapse. The prover Satallax \cite{Satallax} can indeed show this, but verification with 
Metis still fails.%
\end{isamarkuptext}%
\isamarkuptrue%
\ \ \isacommand{lemma}\isamarkupfalse%
\ MC{\isacharcolon}\ {\isachardoublequoteopen}{\isacharbrackleft}p\ m{\isasymRightarrow}\ {\isacharparenleft}{\isasymbox}\ p{\isacharparenright}{\isacharbrackright}{\isachardoublequoteclose}\isanewline
%
\isadelimproof
\ \ %
\endisadelimproof
%
\isatagproof
\isacommand{using}\isamarkupfalse%
\ T{\isadigit{2}}\ T{\isadigit{3}}\ ess{\isacharunderscore}def\ sym%
\endisatagproof
{\isafoldproof}%
%
\isadelimproof
%
\endisadelimproof
\ \isacommand{sledgehammer}\isamarkupfalse%
\ {\isacharbrackleft}provers\ {\isacharequal}\ remote{\isacharunderscore}satallax{\isacharbrackright}%
\isadelimproof
\ %
\endisadelimproof
%
\isatagproof
\isacommand{oops}\isamarkupfalse%
%
\endisatagproof
{\isafoldproof}%
%
\isadelimproof
%
\endisadelimproof
%
\begin{isamarkuptext}%
We now introduce some orthogonal predicates 'positive' and 'negative'.%
\end{isamarkuptext}%
\isamarkuptrue%
\ \ \isacommand{consts}\isamarkupfalse%
\ positive\ {\isacharcolon}{\isacharcolon}\ {\isachardoublequoteopen}{\isacharparenleft}{\isasymmu}\ {\isasymRightarrow}\ {\isasymsigma}{\isacharparenright}\ {\isasymRightarrow}\ {\isasymsigma}{\isachardoublequoteclose}\ \isanewline
\ \ \isacommand{consts}\isamarkupfalse%
\ negative\ {\isacharcolon}{\isacharcolon}\ {\isachardoublequoteopen}{\isacharparenleft}{\isasymmu}\ {\isasymRightarrow}\ {\isasymsigma}{\isacharparenright}\ {\isasymRightarrow}\ {\isasymsigma}{\isachardoublequoteclose}\ \isanewline
\isanewline
\ \ \isacommand{axiomatization}\isamarkupfalse%
\ \isakeyword{where}\isanewline
\ \ \ \ axTest{\isadigit{1}}\ \ {\isacharcolon}\ {\isachardoublequoteopen}{\isacharbrackleft}positive{\isacharparenleft}{\isasymphi}{\isacharparenright}\ m{\isasymor}\ negative{\isacharparenleft}{\isasymphi}{\isacharparenright}{\isacharbrackright}{\isachardoublequoteclose}\ \isakeyword{and}\isanewline
\ \ \ \ axTest{\isadigit{2}}\ {\isacharcolon}\ {\isachardoublequoteopen}{\isacharbrackleft}positive{\isacharparenleft}{\isasymphi}{\isacharparenright}\ m{\isasymequiv}\ m{\isasymnot}\ {\isacharparenleft}negative{\isacharparenleft}{\isasymphi}{\isacharparenright}{\isacharparenright}{\isacharbrackright}{\isachardoublequoteclose}\ \isakeyword{and}\isanewline
\ \ \ \ axTest{\isadigit{3}}\ \ {\isacharcolon}\ {\isachardoublequoteopen}{\isacharbrackleft}m{\isasymnot}\ {\isacharparenleft}positive{\isacharparenleft}{\isasymphi}{\isacharparenright}{\isacharparenright}\ m{\isasymequiv}\ {\isacharparenleft}positive\ {\isacharparenleft}{\isasymlambda}x\ {\isachardot}\ m{\isasymnot}\ {\isacharparenleft}{\isasymphi}\ x{\isacharparenright}{\isacharparenright}{\isacharparenright}{\isacharbrackright}{\isachardoublequoteclose}\ \isakeyword{and}\isanewline
\ \ \ \ axTest{\isadigit{4}}\ \ {\isacharcolon}\ {\isachardoublequoteopen}{\isacharbrackleft}m{\isasymnot}\ {\isacharparenleft}negative{\isacharparenleft}{\isasymphi}{\isacharparenright}{\isacharparenright}\ m{\isasymequiv}\ {\isacharparenleft}negative\ {\isacharparenleft}{\isasymlambda}x\ {\isachardot}\ m{\isasymnot}\ {\isacharparenleft}{\isasymphi}\ x{\isacharparenright}{\isacharparenright}{\isacharparenright}{\isacharbrackright}{\isachardoublequoteclose}%
\begin{isamarkuptext}%
We model a concrete God-like being called \isa{god{\isadigit{1}}}. \isa{god{\isadigit{1}}} is omniscient, 
punitive, and a fan of the Bayern Munich soccer team. Omniscience is modeled as a positive property 
and the other two properties are declared as negative.%
\end{isamarkuptext}%
\isamarkuptrue%
\ \ \isacommand{consts}\isamarkupfalse%
\ god{\isadigit{1}}\ {\isacharcolon}{\isacharcolon}\ {\isachardoublequoteopen}{\isasymmu}{\isachardoublequoteclose}\isanewline
\ \ \isacommand{consts}\isamarkupfalse%
\ omniscient\ {\isacharcolon}{\isacharcolon}\ {\isachardoublequoteopen}{\isasymmu}\ {\isasymRightarrow}\ {\isasymsigma}{\isachardoublequoteclose}\isanewline
\ \ \isacommand{consts}\isamarkupfalse%
\ fanOfBayernMunich\ {\isacharcolon}{\isacharcolon}\ {\isachardoublequoteopen}{\isasymmu}\ {\isasymRightarrow}\ {\isasymsigma}{\isachardoublequoteclose}\isanewline
\ \ \isacommand{consts}\isamarkupfalse%
\ punitive\ {\isacharcolon}{\isacharcolon}\ {\isachardoublequoteopen}{\isasymmu}\ {\isasymRightarrow}\ {\isasymsigma}{\isachardoublequoteclose}\isanewline
\ \ \isanewline
\ \ \isacommand{axiomatization}\isamarkupfalse%
\ \isakeyword{where}\isanewline
\ \ \ \ axTest{\isadigit{5}}\ {\isacharcolon}\ {\isachardoublequoteopen}{\isacharbrackleft}positive{\isacharparenleft}omniscient{\isacharparenright}\ m{\isasymand}\ negative{\isacharparenleft}punitive{\isacharparenright}\ m{\isasymand}\ negative{\isacharparenleft}fanOfBayernMunich{\isacharparenright}{\isacharbrackright}{\isachardoublequoteclose}\ \isakeyword{and}\isanewline
\ \ \ \ axTest{\isadigit{6}}\ {\isacharcolon}\ {\isachardoublequoteopen}{\isacharbrackleft}omniscient{\isacharparenleft}god{\isadigit{1}}{\isacharparenright}\ m{\isasymand}\ punitive{\isacharparenleft}god{\isadigit{1}}{\isacharparenright}\ m{\isasymand}\ fanOfBayernMunich{\isacharparenleft}god{\isadigit{1}}{\isacharparenright}{\isacharbrackright}{\isachardoublequoteclose}\ \isakeyword{and}\isanewline
\ \ \ \ axTest{\isadigit{7}}\ {\isacharcolon}\ {\isachardoublequoteopen}{\isacharbrackleft}G\ god{\isadigit{1}}{\isacharbrackright}{\isachardoublequoteclose}%
\begin{isamarkuptext}%
Nitpick confirms that these assumptions are consistent.%
\end{isamarkuptext}%
\isamarkuptrue%
\ \ \isacommand{lemma}\isamarkupfalse%
\ True\ \isacommand{nitpick}\isamarkupfalse%
\ {\isacharbrackleft}satisfy{\isacharcomma}\ user{\isacharunderscore}axioms{\isacharcomma}\ expect\ {\isacharequal}\ genuine{\isacharbrackright}%
\isadelimproof
\ %
\endisadelimproof
%
\isatagproof
\isacommand{oops}\isamarkupfalse%
%
\endisatagproof
{\isafoldproof}%
%
\isadelimproof
%
\endisadelimproof
%
\begin{isamarkuptext}%
We prove that the properties of \isa{god{\isadigit{1}}} are all divine properties.%
\end{isamarkuptext}%
\isamarkuptrue%
\ \ \isacommand{lemma}\isamarkupfalse%
\ DivineProps\ {\isacharcolon}\ {\isachardoublequoteopen}{\isacharbrackleft}Divine{\isacharparenleft}omniscient{\isacharparenright}\ m{\isasymand}\ Divine{\isacharparenleft}punitive{\isacharparenright}\ m{\isasymand}\ Divine{\isacharparenleft}fanOfBayernMunich{\isacharparenright}{\isacharbrackright}{\isachardoublequoteclose}\isanewline
\ \ \isacommand{sledgehammer}\isamarkupfalse%
\ {\isacharbrackleft}provers\ {\isacharequal}\ remote{\isacharunderscore}satallax{\isacharbrackright}\isanewline
%
\isadelimproof
\ \ %
\endisadelimproof
%
\isatagproof
\isacommand{by}\isamarkupfalse%
\ {\isacharparenleft}metis\ A{\isadigit{1}}b\ G{\isacharunderscore}def\ axTest{\isadigit{6}}\ axTest{\isadigit{7}}{\isacharparenright}%
\endisatagproof
{\isafoldproof}%
%
\isadelimproof
%
\endisadelimproof
%
\begin{isamarkuptext}%
\newpage%
\end{isamarkuptext}%
\isamarkuptrue%
%
\isadelimtheory
%
\endisadelimtheory
%
\isatagtheory
%
\endisatagtheory
{\isafoldtheory}%
%
\isadelimtheory
%
\endisadelimtheory
\ \end{isabellebody}%
%%% Local Variables:
%%% mode: latex
%%% TeX-master: "root"
%%% End:
