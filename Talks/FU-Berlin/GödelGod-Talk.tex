\documentclass[9pt]{beamer}
\mode<presentation>
{
  \usetheme{Warsaw}
  \usecolortheme{crane}
  \setbeamercovered{transparent}
}

\usepackage[english]{babel}
\usepackage[latin1]{inputenc}
\usepackage{times}
\usepackage[T1]{fontenc}
\usepackage{verbatim}

\usepackage{amsmath}
\usepackage{amssymb}
\usepackage{amsfonts}
\usepackage{pxfonts}
\usepackage{amsthm}
\usepackage{graphicx}



\usepackage{bussproofs}
\usepackage{proof}

\usepackage{fancybox}
\usepackage{fancyvrb}

%\usepackage{rotating}

% Sequent Calculus Proof Settings
\EnableBpAbbreviations
\def\fCenter{\mbox{\ $\vdash$\ }}


\newcommand{\mypause}{\pause}
%\newcommand{\mypause}{}

\title[On G\"{o}del's Proof of God's
Existence]{Formalization, Mechanization and Automation of \\ G\"{o}del's
  Proof of God's Existence}

\author{Christoph Benzm\"{u}ller and Bruno Woltzenlogel Paleo}

\institute[]{
  \inst{}%
}

\date[1.11.2013]
{November 1, 2013}

\begin{document}

\begin{frame}
  \titlepage
\end{frame}


\begin{frame}{Introduction} \small
\begin{minipage}{.56\textwidth} 
\includegraphics[width=\textwidth]{spiegel1} 
\vskip1em
Germany \\
- Telepolis \& Heise \\
- Spiegel Online \\
- FAZ \\
- Die Welt \\
- Berliner Morgenpost \\
- Hamburger Abendpost \\
- \ldots \\
\end{minipage} \hfill
%
\begin{minipage}{.3\textwidth}
Austria \\
- myscience.at \\
- Wiener Zeitung \\
- ORF \\
- \ldots \\

Italy \\
- Repubblica \\
- Today.it \\
- Ilsussidario \\
- \ldots \\

% Russia \\
% - \ldots \\

India \\
- DNA India \\
- Delhi Daily News \\
- Indoa Today \\
- \ldots \\

International \\
- Spiegel International \\
- Yahoo Finance \\
- CNET \\
- United Press Intl. \\
- \ldots
\end{minipage}
\end{frame}


\begin{frame}{Introduction}\large
%Ontological argument: Conclude that God exists from premises by pure, a priori reasoning.
\begin{block}{Def: \textcolor{blue}{Ontological Argument/Proof}}
* deductive argument \\

* for the existence of god \\

* starting from premises, which are justified by pure reasoning,
i.e. they do not depend on observation in the world.
\end{block}

\vfill \pause
Existence of God: different types of arguments/proofs\\[.2em]
\begin{itemize}
\item[---]a posteriori (use experience/observation in the world)
  \begin{itemize}
  \item[------]teleological
  \item[------]cosmological
  \item[------]moral
  \item[------] \ldots
  \end{itemize}  
\item[---]a priori (based on pure reasoning, independent)
  \begin{itemize}
  \item[------]\textcolor{blue}{ontological argument}
    \begin{itemize}
    \item[------]definitional 
    \item[------]modal 
    \item[------] \ldots
    \end{itemize}
  \item[------]other a priori arguments
  \end{itemize}
\end{itemize}
\end{frame}

\begin{frame}{Introduction}

\emph{\huge Wohl eine jede Philosophie kreist um den ontologischen
  Gottesbeweis} \\[2em]
(Adorno, Th. W.: Negative Dialektik. Frankfurt a. M. 1966, p.378)
\vfill
\ldots \hfill
\includegraphics[height=2cm]{buch3.jpg} \hfill
\includegraphics[height=2cm]{buch2.jpg} \hfill 
\includegraphics[height=2cm]{buch4.jpg} \hfill
\includegraphics[height=2cm]{buch5.jpg} \hfill
\includegraphics[height=2cm]{buch6.jpg} \hfill
\includegraphics[height=2cm]{buch1.jpg} \hfill
\ldots
\end{frame}


\begin{frame}{Introduction} \Large
{Rich history on ontological arguments (\textcolor{blue}{pros} and \textcolor{red}{cons})}\\[1em]

\ldots\rotatebox[origin = bl,width = 0mm]{65}{\textcolor{blue}{Anselm v. C.}} \hskip-2.3em
\ldots  \rotatebox[origin = bl]{65}{\textcolor{red}{Th. Aquinas}}  \hskip-2.3em
\ldots\ldots  \rotatebox[origin = bl]{65}{\textcolor{blue}{Spinoza}} \hskip-1.3em
                \rotatebox[origin = bl]{65}{\textcolor{blue}{Leibniz}}  \hskip-1.2em
\ldots  \rotatebox[origin = bl]{65}{\textcolor{red}{Hume}}  \hskip-1em
          \rotatebox[origin = bl]{65}{\textcolor{red}{Kant}}  \hskip-.8em
\ldots  \rotatebox[origin = bl]{65}{\textcolor{blue}{Hegel}}  \hskip-1.3em
\ldots  \rotatebox[origin = bl]{65}{\textcolor{red}{Frege}}  \hskip-1.3em
\ldots  \rotatebox[origin = bl]{65}{\textcolor{blue}{Hartshorne}} \hskip-1.9em
          \rotatebox[origin = bl]{65}{\textcolor{blue}{Malcolm}}  \hskip-1.4em
          \rotatebox[origin = bl]{65}{\textcolor{red}{Lewis}}  \hskip-1em
          \rotatebox[origin = bl]{65}{\textcolor{blue}{Plantinga}}  \hskip-1.6em
          \rotatebox[origin = bl]{65}{\textcolor{blue}{G\"odel}}   \hskip-1.2em
\ldots \\[1em]

\pause
\vfill
Anselm's notion of God:\\
\,\hfill \emph{``God is that, than which nothing greater can be
  conceived.''} \\[1em]

G\"odel's notion of God:\\
\,\hfill \emph{``A God-like being possesses all `positive' properties.''} \\[1em]

To show by logical reasoning: \\
\,\hfill \emph{``(Necessarily) God exists.''} \\[1em]

% \rnode{n2}{}\emph{}
% \ncline[nodesep=2pt,linecolor=black,linewidth=1pt]{->}{n1}{n2}
\end{frame}


\begin{frame}{Introduction} \Large
Different Interests in Ontological Arguments: \\[1em]
\begin{itemize}
\item \textcolor{blue}{Philosophical:} Boundaries of Mataphysics \& Epistemology
  \begin{itemize}
  \item We talk about a metaphysical concept (God), 
  \item but we want to draw
      a conclusion for the real world. \\[2em]
  \end{itemize} 
\item \textcolor{blue}{Theistic:} Successful argument should convince atheists. \\[2em]
\item \textcolor{red}{Our:} Can computers (theorem provers) be used
  \begin{itemize}
  \item to formalize the definitions and axioms?
  \item to verify the arguments step-by-step?
  \item to fully automate (sub-)arguments? \\[1em]
  \end{itemize}
  \textcolor{red}{\emph{``Computer-assisted Theoretical Philosophy''}}
\end{itemize}
\end{frame}

\begin{frame}{Introduction} \large
Main Challenge: \hfill No theorem provers for \emph{Second-order Modal
  Logic} \\[1em]

Our Idea: \hfill Exploit an embedding in \emph{Classical Higher-order
  Logic} \\
\,\hfill {\small [Benzm\"ullerPaulson, Logica Universalis, 2013]} \\[2em]

What we did \textcolor{blue}{(rough outline for remaining
  presentation!)}: \\

\begin{itemize}
\item[A:] pen and paper: detailed natural deduction proof \hfill Bruno
\item[B:] formalization: axioms, defs, thms  in \textsc{TPTP THF}
\item[C:] consistency: automatic verification with \textsc{Nitpick} 
\item[D:] proof automation: theorems provers Leo-II and \textsc{Satallax} 
\item[E:] step-by-step verification: proof assistant \textsc{Coq} 
\item[F:] automation \& verification: proof assistant
  \textsc{Isabelle}
\item 
\end{itemize}



\end{frame}

\end{document}
