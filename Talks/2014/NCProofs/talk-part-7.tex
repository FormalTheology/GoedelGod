
\begin{transitionframe}{Images/Transitions/TrifidNebula(Nasa)(PublicDomain).jpg}{black}
\textbf{Conclusions and Future Work}
\end{transitionframe}



% \begin{frame}{Summary of Results} \large

% The (\alert{new}) insights we gained from experiments include:\\[.5em]
% \begin{itemize}
% \item Logic K sufficient for T1, C and T2 
% \item Logic S5 \emph{not} needed for T3
% \item \alert{Logic KB sufficient for T3 (not well known)}
% \item \alert{We found a simpler new proof of C}
% \item \alert{G\"odel's axioms (without conjunct $\phi(x)$ in D2) are inconsistent}
% \item Scott's axioms are consistent
% \item For T1, only half of A1 (A1a) is needed 
% \item For T2, the other half (A1b) is needed
% \end{itemize}
% \end{frame}


\begin{frame}{Conclusions} \large

\begin{itemize}
\item Contributions: \\[.5em]
\begin{itemize}
\item Powerful infrastructure for reasoning with QML
\item A (new?) natural deduction calculus for \\ higher-order (intuitionistic) modal logic
\item Non-trivial new benchmark problems for HOL provers
\item Verification of existing results about G\"odel's proof (e.g. Modal Collapse)
\item New results about G\"odel's proof (e.g. Inconsistency)
%\item Huge media attention
\end{itemize}

%\item Interesting bridge between CS, Philosophy and Theology

\item Major step towards \alert{Computer-assisted Theoretical Philosophy}
 \begin{itemize}
  \item see also Ed Zalta's \emph{Computational Metaphysics} project at Stanford University and John Rushby's formalization of Anselm's proof using PVS
  %\item remember Leibniz' dictum --- \emph{Calculemus!}
  \end{itemize}
\end{itemize}
\end{frame}

\newcommand\hol[1]{\boldsymbol{#1}}
\newcommand\lift[1]{\lceil #1 \rceil}
\newcommand\llift[1]{\dot{#1}}

\begin{frame}{Future Work}{From a Non-Classical Proof-Theoretic Perspective} \large

\begin{itemize}
\item Embedding of other non-classical logics
\begin{itemize}
\item Quantified Conditional Logics, \\
      Modal Logics based on Neighbourhood Semantics, \ldots
\pause
\item Other Suggestions?
\pause
\item Human Resources?
\end{itemize}

\pause

\item New non-classical calculi via HOL embedding: \\
(when does this approach work?)
\pause
\begin{enumerate}
  \item Embed your favorite logic $L$ into HOL
  \pause
  \item Play with Coq and invent new tactics to encapsulate the embedding and make it transparent to the user
  \pause
  \item Interpret the tactics as new rules for a new natural deduction calculus $C$
  \pause
  \item Prove soundness and completeness indirectly: \\
  \[
  \models_L s_o \quad \text{iff} \quad \hol{\models} \hol{\textrm{valid}(}\lceil s_o \rceil\hol{)}
  \qquad\qquad
  \hol{\vdash} \hol{\textrm{valid}(}\lceil s_o \rceil\hol{)} \quad \text{iff} \quad \vdash_C s_o
  \]

\end{enumerate}

%\pause

%\item Why is it so hard to prove T3 without intermediary steps?

%\pause

%\item Provocative question: \pause life is short! So why take the long structural proof-theoretical road, if the ``lazier'' HOL embedding approach gives us efficient theorem provers in less than 100 lines of declarative code?

\end{itemize}
\end{frame}


% \begin{frame}{Ongoing and Future Work} \large
% \begin{itemize}
% \item Formalize and verify other ontological arguments
%   \begin{itemize}
%   \item \ldots particularly the criticisms and improvements
%   \end{itemize}
% \item Experiment with other embeddings (e.g. varying domains)
% \item Eliminate and introduce cuts
% \end{itemize}
% \end{frame}


% \begin{frame}{Some Comments and Reactions}
% \colorbox{gray}{\includegraphics[width=.8\textwidth]{Images/Comments/Comment1}}\\[.7em]

% \, \hfill \colorbox{gray}{\includegraphics[width=.7\textwidth]{Images/Comments/Comment2}}\\[.7em]

% \colorbox{gray}{\includegraphics[width=.8\textwidth]{Images/Comments/Comment3}}\\[1em]

% \, \hfill \ldots find more on the internet \ldots
% \end{frame}

\begin{frame}[plain]
\colorbox{black}{\includegraphics[width=\textwidth]{Images/Transitions/GodComputerC}}
\end{frame}

% \begin{frame}{Licenses} \centering
% \includegraphics[scale=0.5]{Images/CC-BY-SA.png}

% \bigskip
% \bigskip

% \begin{center}
% The following images used in these slides were obtained in commons.wikimedia.org and are licensed as follows:

% \bigskip

% CC-BY-SA:

% ReligiousSymbols, PaganReligiousSymbols.

% \bigskip

% Public Domain: 

% TrifidNebula
% \end{center}
% \end{frame}
