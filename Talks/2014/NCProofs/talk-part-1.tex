
% \begin{frame}{} \small
% \vskip1em
% \begin{minipage}{.56\textwidth} 
% \colorbox{gray}{\includegraphics[width=\textwidth]{Images/News/spiegel1}} 
% \vskip1em
% Germany \\
% - Telepolis \& Heise \\
% - Spiegel Online \\
% - FAZ \\
% - Die Welt \\
% - Berliner Morgenpost \\
% - Hamburger Abendpost \\

% \end{minipage} \hfill
% %
% \begin{minipage}{.35\textwidth}
% Austria \\
% - Die Presse \\
% - Wiener Zeitung \\
% - ORF \\

% International \\
% - Spiegel International \\
% - Yahoo Finance \\
% % - CNET \\
% - United Press Intl. \\

% India \\
% - DNA India \\
% - Delhi Daily News \\
% - India Today \\

% US \\
% - ABC News \\

% Italy \\
% - Repubblica \\

% Spain, Russia, Brazil, Bulgaria \\

% \ldots

% \end{minipage}
% \end{frame}


% \begin{frame}{Motivations} \Large
% \begin{itemize}
% \item \textcolor{blue}{Philosophical:} 
%   \begin{itemize}
%   \item Limits of metaphysics \& epistemology
%   \item \emph{Metaphysical} versus \emph{logical} necessary existence \\[2em]
%   \end{itemize} 
% \item \textcolor{blue}{Theological:} 
%   \begin{itemize}
%   \item Investigations of the nature of God
%   \item Arguments to convince atheists \\[2em]
%   \end{itemize}
% \item \textcolor{red}{Computational:} can automated reasoners be used \ldots
%   \begin{itemize}
%   \item \ldots to formalize the definitions, axioms and theorems?
%   \item \ldots to verify the arguments step-by-step?
%   \item \ldots to fully automate (sub-)arguments? \\[2em]
%   \end{itemize}
% \end{itemize}

% \begin{center}
%   \textcolor{red}{\emph{``Computer-assisted Theoretical Philosophy''}}
% \end{center}
% \end{frame}

\begin{frame}{G\"odel's Manuscript: 1930's, 1941, 1946-1955, 1970}
\bigskip

\begin{changemargin}{-1.2cm}{-1.2cm}
\includegraphics[width=13cm]{Images/Manuscript.png}
\end{changemargin}
\end{frame}


\begin{frame}{A Long History}{\textcolor{blue}{pros} and \textcolor{red}{cons}} \Large

\hskip-.5em
\ldots\rotatebox[origin = bl,width = 0mm]{65}{\textcolor{blue}{Anselm v. C.}} \hskip-2.3em
          \rotatebox[origin = bl]{65}{\textcolor{red}{Gaunilo}} \hskip-1.3em
\ldots  \rotatebox[origin = bl]{65}{\textcolor{red}{Th. Aquinas}}  \hskip-2.3em
\ldots\ldots   \rotatebox[origin = bl]{65}{\textcolor{blue}{Descartes}} \hskip-1.7em
               \rotatebox[origin = bl]{65}{\textcolor{blue}{Spinoza}} \hskip-1.3em
               \rotatebox[origin = bl]{65}{\textcolor{blue}{Leibniz}}  \hskip-1.2em
\ldots  \rotatebox[origin = bl]{65}{\textcolor{red}{Hume}}  \hskip-1em
          \rotatebox[origin = bl]{65}{\textcolor{red}{Kant}}  \hskip-.8em
\ldots  \rotatebox[origin = bl]{65}{\textcolor{blue}{Hegel}}  \hskip-1.3em
\ldots  \rotatebox[origin = bl]{65}{\textcolor{red}{Frege}}  \hskip-1.3em
\ldots  \rotatebox[origin = bl]{65}{\textcolor{blue}{Hartshorne}} \hskip-1.9em
          \rotatebox[origin = bl]{65}{\textcolor{blue}{Malcolm}}  \hskip-1.4em
          \rotatebox[origin = bl]{65}{\textcolor{red}{Lewis}}  \hskip-1em
          \rotatebox[origin = bl]{65}{\textcolor{blue}{Plantinga}}  \hskip-1.6em
          \rotatebox[origin = bl]{65}{\textcolor{blue}{G\"odel}}   \hskip-1.2em
\ldots \\[1em]

\pause
\vfill
Anselm's notion of God:\\
\,\hfill \emph{``God is that, than which nothing greater can be
  conceived.''} \\[1em]

G\"odel's notion of God:\\
\,\hfill \emph{``A God-like being possesses all `positive' properties.''} \\[1em]

%To show by logical reasoning: \\
%\,\hfill \emph{``(Necessarily) God exists.''} \\[1em]

\end{frame}



\begin{frame}{The Ontological Proof Today}
\vskip1em
% \emph{\huge Wohl eine jede Philosophie kreist um den ontologischen
%   Gottesbeweis} \\[1.5em]
% (Adorno, Th. W.: Negative Dialektik. Frankfurt a. M. 1966, p.378)
% \vfill
\begin{center}
\fcolorbox{gray}{white}{
\begin{minipage}{0.9\textwidth}
\hfill
\includegraphics[height=2.2cm]{Images/Books/buch3.jpg} \hfill
\includegraphics[height=2.2cm]{Images/Books/buch2.jpg} \hfill 
\includegraphics[height=2.2cm]{Images/Books/buch4.jpg} \hfill
\hfill

\vspace{0.5 cm}

\hfill
\includegraphics[height=2.2cm]{Images/Books/buch7.jpg} \hfill
\includegraphics[height=2.2cm]{Images/Books/buch5.jpg} \hfill
\includegraphics[height=2.2cm]{Images/Books/buch6.jpg} \hfill
\includegraphics[height=2.2cm]{Images/Books/buch1.jpg} \hfill
\hfill
\end{minipage}
}

\end{center}
\end{frame}

