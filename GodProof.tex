\documentclass{article}

\usepackage{latexsym}
\usepackage{bussproofs}
\EnableBpAbbreviations
\newcommand{\rl}[1]{\RightLabel{#1}}

% Logical symbols
\newcommand{\imp}{\rightarrow}
\newcommand{\biimp}{\leftrightarrow}
\newcommand{\all}{\forall}
\newcommand{\ex}{\exists}
\newcommand{\seq}{\vdash}
\newcommand{\nec}{\Box} % necessarily
\newcommand{\pos}{\Diamond} % possibly

\author{Bruno Woltzenlogel Paleo}

\title{G\"{o}del's Ontological Proof of God's Existence }

\begin{document}

\maketitle

\newcommand{\ess}[2]{#1 \ \mathit{ess} \ #2}


\noindent
``There is a scientific (exact) philosophy and theology, 
which deals with concepts of the highest abstractness; and this is also most highly fruitful for science. [\ldots] Religions are, for the most part, bad; but religion is not.'' - Kurt G\"{o}del


\section{Possible witnessing of positive properties}

\textbf{Axioms:}
\begin{itemize}
\item \textbf{(1)} Properties necessarily entailed by \emph{positive} properties are also positive: 
$$
\all \varphi. \all \psi.[(P(\varphi) \wedge \nec \all x.[\varphi(x) \imp \psi(x)]) \imp P(\psi)]
$$
%
\item \textbf{(2)} A property's negation is positive iff the property is not positive: 
$$ 
\all \varphi. [P(\neg \varphi) \biimp \neg P(\varphi)]
$$
\end{itemize}

\noindent
\textbf{Theorem 1:} Positive properties possibly have a witness:
$$
\all \varphi. [P(\varphi) \imp \pos \ex x.\varphi(x)]
$$


\noindent
\textbf{Formal proof:}

\begin{prooftree}
        \AXC{$ \all \varphi. \all \psi.[(P(\varphi) \wedge \nec \all x.[\varphi(x) \imp \psi(x)]) \imp P(\psi)]$}
        \UIC{$ \all \psi.[(P(\varphi') \wedge \nec \all x.[\varphi'(x) \imp \psi(x)]) \imp P(\psi)]$}
        \UIC{$(P(\varphi') \wedge \nec \all x.[\varphi'(x) \imp \neg \varphi'(x)]) \imp P(\neg \varphi')$} \doubleLine
        \UIC{$(P(\varphi') \wedge \nec \all x.[\neg \varphi'(x)]) \imp P(\neg \varphi')$}
                        \AXC{$\all \varphi.[ P(\neg \varphi) \biimp \neg P(\varphi) ]$}
                        \UIC{$ P(\neg \varphi') \biimp \neg P(\varphi') $} \doubleLine
                 \BIC{$ (P(\varphi') \wedge \nec \all x.[\neg \varphi'(x)]) \imp \neg P(\varphi') $} \doubleLine
                 \UIC{$ P(\varphi') \imp \pos \ex x.\varphi'(x) $}
                 \UIC{$\all \varphi.[ P(\varphi) \imp \pos \ex x.\varphi(x) ] $}
\end{prooftree}



\section{Possible existence of a God}

\textbf{Axioms:}
\begin{itemize}
\item \textbf{(3)} Being God is a positive property: 
$$
P(G)
$$
\end{itemize}

\noindent
\textbf{Theorem 2:} It is possible that a God exists:
$$
\pos \ex x. G(x)
$$

\noindent
\textbf{Formal proof:}

\begin{prooftree}
\AXC{$P(G)$}
                 \AXC{$ $} \dashedLine \RightLabel{Th. 1}
                 \UIC{$\all \varphi.[ P(\varphi) \imp \pos \ex x.\varphi(x) ]$}
                 \UIC{$ P(G) \imp \pos \ex x.G(x) $}
    \BIC{$\pos \ex x. G(x)$}
\end{prooftree}


\section{Essentiality of being God}

\textbf{Definitions:}
\begin{itemize}
\item A property is \emph{essential} for an individual if and only if it holds for that inidividual and necessarily entails every other property that holds for that individual: $\ess{\varphi}{x} \biimp \varphi(x) \wedge \all \psi. (\psi(x) \imp \nec \all x. (\varphi(x) \imp \psi(x)))$
\end{itemize}

\noindent
\textbf{Axioms:}
\begin{itemize}
\item \textbf{(4)} Positive properties are necessarily positive: 
$$
\all \varphi.[P(\varphi) \to \Box \; P(\varphi)]
$$
\end{itemize}

\noindent
\textbf{Theorem 3}: If an individual is a God, then being God is an essential property for that individual:
$$
\all y.[G(y) \imp \ess{G}{y}]
$$

\noindent
\textbf{Formal proof:}

\begin{prooftree}
\AXC{$ $} \RightLabel{1}
\UIC{$ G(x) $}
        \AXC{$ $} \RightLabel{2}
        \UIC{$\psi(x)$}
        \UIC{$G(x) \imp \psi(x)$}
        \UIC{$\all x.[G(x) \imp \psi(x)]$} \RightLabel{Necessitation}
        \UIC{$\nec \all x.[G(x) \imp \psi(x)]$} \RightLabel{2}
        \UIC{$\psi(x) \imp \nec \all x.[G(x) \imp \psi(x)]$}
        \UIC{$\all \psi.[\psi(x) \imp \nec \all x.[G(x) \imp \psi(x)]]$}
    \BIC{$G(x) \wedge \all \psi.[\psi(x) \imp \nec \all x.[G(x) \imp \psi(x)]]$} \dottedLine
    \UIC{$\ess{G}{x}$} \RightLabel{1}
    \UIC{$G(x) \imp \ess{G}{x}$}
    \UIC{$\all y.[G(y) \imp \ess{G}{y}]$}
\end{prooftree}

\noindent
\textbf{Note:} the formal proof above uses the necessitation rule of the basic modal logic \textbf{K}, instead of using Axiom 4, which corresponds to a restricted form of necessitation. I currently do not know how to formally prove theorem 3 using axiom 4 and not relying on necessitation.


\section{Necessity of God's existence}

\textbf{Definitions:}
\begin{itemize}
\item An individual is a \emph{God} if and only if he possesses all positive properties: 
$$
G(x) \biimp \forall \varphi. [P(\varphi) \to \varphi(x)]
$$
\item An individual \emph{necessarily exists} if and only if all its essential properties are necessarily witnessed: 
$$ 
E(x) \biimp \all \varphi.[\ess{\varphi}{x} \imp \nec \ex x.\varphi(x)] 
$$
\end{itemize}

\noindent
\textbf{Axioms:}
\begin{itemize}
\item \textbf{(5)} Necessary existence is a positive property: 
$$
P(E)
$$
\end{itemize}

\noindent
\textbf{Theorem A:} If there is a God, then there necessarily exists a God:
$$
\ex z. G(z) \imp \nec \ex x. G(x)
$$


\noindent
\textbf{Formal proof:}

\begin{prooftree}
\AXC{$ $} \RightLabel{1}
\UIC{$\ex z. G(z)$}
\UIC{$G(g)$}
\end{prooftree}

\begin{prooftree}
\AXC{$ $} \dashedLine
\UIC{$G(g)$}
        \AXC{$ $} \dashedLine \RightLabel{Th. 3}
        \UIC{$\all y.[G(y) \imp \ess{G}{y}]$}
        \UIC{$G(g) \imp \ess{G}{g}$}
    \BIC{$\ess{G}{g}$}
                \AXC{$P(E)$}
                        \AXC{$ $} \dashedLine
                        \UIC{$G(g)$} \dottedLine
                        \UIC{$\all \varphi.[P(\varphi) \imp \varphi(g)]$}
                        \UIC{$P(E) \imp E(g)$}
                     \BIC{$E(g)$} \dottedLine
                     \UIC{$ \all \varphi.[\ess{\varphi}{g} \imp \nec \ex x.\varphi(x)] $}
                     \UIC{$ \ess{G}{g} \imp \nec \ex x. G(x) $}
        \BIC{$\nec \ex x. G(x)$} \RightLabel{1}
        \UIC{$\ex z. G(z) \imp \nec \ex x. G(x)$}
\end{prooftree}

\noindent
\textbf{Note:} Theorem A could be proved more quickly using the necessitation rule. Interestingly, the proof above shows that, by using the given axioms and definitions of god and necessary existence, theorem A can be derived even without the necessitation rule.


\section{Necessary existence of a God}


\noindent
\textbf{Theorem 4:} The existence of a God is necessary:
$$
\nec \ex x. G(x)
$$


\noindent
\textbf{Formal proof:}

\begin{small}
\begin{prooftree}
\AXC{$ $} \dashedLine \RightLabel{\textbf{S5}}
\UIC{$ \all \varphi. [\pos \ldots \pos \nec \varphi \biimp \nec \varphi] $}
\UIC{$ \pos \nec  \ex x. G(x) \biimp \nec \ex x. G(x) $}
        \AXC{$ $} \dashedLine  \RightLabel{Th. 2}
        \UIC{$\pos \ex x. G(x)$}
                \AXC{$ $} \dashedLine \RightLabel{Th. A}
                \UIC{$\ex z. G(z) \imp \nec \ex x. G(x)$}
            \BIC{$\pos \nec \ex x. G(x)$}
    \BIC{$\nec \ex x. G(x)$}
\end{prooftree}
\end{small}

\noindent
\textbf{Note:} The proof above relies on a theorem of the modal logic \textbf{S5}, which is a quite strong modal logic. It would be interesting to try to derive theorem 4 with weaker modal axioms. 


\section{God's existence}


\noindent
\textbf{Axioms:}
\begin{itemize}
\item \textbf{(M)} What is necessary is the case:
$$
\all \varphi. [\nec \varphi \imp \varphi]
$$
\end{itemize}

\noindent
\textbf{Theorem:} There exists a God:
$$
\ex x. G(x)
$$

\noindent
\textbf{Formal proof:}


\begin{prooftree}
\AXC{$ $} \dashedLine \RightLabel{Th. 4}
\UIC{$ \nec \ex x. G(x)$}
        \AXC{$ \all \varphi. [\nec \varphi \imp \varphi]$}
        \UIC{$\nec \ex x. G(x) \imp \ex x. G(x)$}
    \BIC{$\ex x. G(x)$}
\end{prooftree}



\end{document}